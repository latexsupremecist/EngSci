\documentclass[12pt]{article}
\usepackage{../../template}
\title{Lecture 5}
\author{niceguy}
\begin{document}
\maketitle

\section{More Linear Algebra}

For any ket $|\psi\rangle$, we can write it as a column vector

$$\begin{pmatrix} \langle v_1|\psi\rangle \\ \langle v_2|\psi\rangle \\ \vdots \end{pmatrix}$$

Similarly for any bra $\langle\phi|$,

$$\begin{pmatrix} \langle \phi|v_1\rangle \\ \langle \phi|v_2\rangle \\ \dots \end{pmatrix}$$

\begin{align*}
    \langle \phi | A | \psi \rangle &= \langle \phi IAI | \psi \rangle \\
                                    &= \sum_{ij} \langle \phi | u_i \rangle \langle u_i | A | u_j \rangle \langle u_j | \psi \rangle
\end{align*}

If we have $C=AB$, then the inner term becomes

$$\langle u_i | AB | u_j \rangle = \sum_k \langle u_i | A | u_k \rangle \langle u_k | B | u_j \rangle$$

\section{Change of Variables}

In different bases, we can express the same ket as

$$|\psi\rangle = \sum_i c_i|u_i\rangle = \sum_i |u_i\rangle\langle u_i |\psi \rangle$$
and
$$|\psi\rangle = \sum_k c_k'|t_k\rangle = \sum_k |t_k\rangle\langle t_k |\psi \rangle$$
Then
\begin{align*}
    \langle u_m | \psi \rangle &= \sum_k \langle u_m|t_k \rangle\langle t_k | \psi \rangle \\
                               &= \sum_k S_{mk} \langle t_k|\psi \rangle \\
    c_m &= \sum_k S_{mk}c_k'
\end{align*}

We can get the same relation in reverse by taking the adjoint of $S$. Applying this a few times, we can do this for a matrix too.

$$A_{kl} = \sum_{ij} S^*_{ki} A_{ij} S_{jl}$$

\section{Eigenkets}

\begin{defn}[Degeneracy]
    We say an eigenket is degenerate if the eigenspace has more than 1 dimension.
\end{defn}

\end{document}
