\documentclass[12pt]{article}
\usepackage{../../template}
\title{Lecture 4}
\author{niceguy}
\begin{document}
\maketitle

\section{Bras and Kets}

Not every bra has a corresponding ket. We avoid this by defining a set of generalised kets which is isomorphic with the bras. This comes at the cost of generalised kets not being normalisable.

\subsection{Linear Operators}

Let $A$ and $B$ be linear operators. Then we can define products $AB$ where $AB|\psi\rangle = A\left(B|\psi\rangle\right)$ is also linear. Note that in general,
$$[A,B] = AB - BA \neq 0$$

\begin{ex}
    If $\langle \psi | \psi \rangle =$, we can define a linear operator
    $$P_\psi = |\psi\rangle\langle\psi|$$
    This is called a projection for obvious reasons. Note that it is equal to its square, which is how we define projections.
\end{ex}

\begin{ex}
    Let
    $$P_g = \sum_{i=1}^g |\phi_i\rangle\langle\phi_i|$$
    Where $\phi_i$ is orthonormal. We can then show $P_g^2 = P_g$, so we can write
    $$P_g|x\rangle = \sum_{i=1}^g |\phi_i\rangle\langle\phi_i|x\rangle$$
\end{ex}

We can also define Hermitian conjugation, which is the adjoint, as in
$$\langle Tx,y \rangle = \langle x,T^*y \rangle$$

We say an operator is hermitian if it is equal to its adjoint. Projections are hermitian.

\begin{defn}[Orthnormal Discrete Basis]
    If we have a set $\{|u_i\rangle\}$ such that $\forall |\psi\rangle$,
    $$|\psi\rangle = \sum_i c_i |u_i\rangle$$
    and that
    $$\langle u_i|u_j \rangle = \delta_{ij}$$
    we say that $\{|u_i\rangle\}$ form an orthonormal discrete basis.
\end{defn}

\begin{defn}[Orthnormal Continuous Basis]
    Likewise, for $\{|w_\alpha\rangle\}$ such that $\forall |\psi\rangle$,
    $$|\psi\rangle = \int c(\alpha) |w_\alpha\rangle d\alpha$$
    and that
    $$\langle w_\alpha|w_{\alpha'}\rangle = \delta(\alpha - \alpha')$$
    We say $\{|w_\alpha\rangle\}$ form an orthonormal continuous basis.
\end{defn}

Using a discrete basis,

\begin{align*}
    |\psi\rangle &= \sum_i c_i|u_i\rangle \\
    \langle u_j|\psi\rangle &= \sum_i \langle u_j|u_i \rangle c_i = c_j
\end{align*}

So

\begin{align*}
    |\psi\rangle &= \sum_i |u_i\rangle\langle u_i|\psi\rangle \\
                 &= I|\psi\rangle
\end{align*}

where

$$I = \sum_i |u_i\rangle\langle u_i |$$

For a continuous basis, we similarly have
$$c(\alpha) = \langle w_\alpha|\psi\rangle$$
where
$$I = \int|w_\alpha\rangle\langle w_\alpha| d\alpha$$

\end{document}
