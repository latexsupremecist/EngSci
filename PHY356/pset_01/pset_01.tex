\documentclass[answers]{exam}
\usepackage{../../template}
\author{niceguy}
\title{Problem Set 1}
\begin{document}
\maketitle

\begin{questions}

\question{Find the group velocity of a wavepacket associated with a relativistic particle in terms of the velocity of the associated particle.}

\begin{solution}
    Note that
    \begin{equation}
        E = \frac{mc^2}{\sqrt{1 - \frac{v^2}{c^2}}}
    \end{equation}
    and
    \begin{equation}
        p = \frac{mv}{\sqrt{1 - \frac{v^2}{c^2}}}
    \end{equation}
    Then squaring both,

    \begin{align}\label{yeet}
        E^2 &= \frac{p^2c^4}{v^2} \\
            &= c^2p^2 \times \frac{c^2}{v^2} \\
            &= c^2p^2 \left(1 + \frac{c^2-v^2}{v^2}\right) \\
            &= c^2p^2 + \frac{m^2c^2(c^2-v^2)}{1 - \frac{v^2}{c^2}} \\
            &= c^2p^2 + m^2c^4
    \end{align}

    From particle properties,
    $$E = h\nu = \hbar \omega$$
    and
    $$p = \frac{h}{\lambda} = \hbar k$$
    Then
    \begin{align*}
        V &= \frac{d\omega}{dk} \\
          &= \frac{\frac{1/\hbar}dE}{\frac{1/\hbar}dp} \\
          &= \frac{dE}{dp}
    \end{align*}
    This hold regardless if the particle is relativistic or not. Hence for a relativistic particle, differentiating both sides of \ref{yeet} with respect to $p$, noting that $m$ and $c$ are constants,
    \begin{align*}
        2E \times \frac{dE}{dp} &= 2c^2p \\
        \frac{dE}{dp} &= c^2 \times \frac{p}{E} \\
        V &= v
    \end{align*}
\end{solution}

\question{Prove the given identity.}

\begin{solution}
    We define $f(k)$ such that
    $$\psi(x,t) = \int f(k)dk$$
    for convenience. Then the left hand side of the identity becomes
    \begin{align*}
        i\hbar \del{\psi(x,t)}{t} &= i\hbar \int -i\omega(k)f(k)dk \\
                                  &= \frac{\hbar^2}{2m} \int k^2f(k)dk
    \end{align*}
    And the right hand side becomes
    \begin{align*}
        -\frac{\hbar^2}{2m}\del{^2\psi(x,t)}{x^2} &= -\frac{\hbar^2}{2m} \int -k^2f(k)dk \\
                                                  &= \frac{\hbar^2}{2m} \int k^2f(k)dk
    \end{align*}
    which is equal to the right hand side.
\end{solution}

\question{Prove the given identity.}

\begin{solution}
    First note that
    \begin{equation}
        \nabla \cdot (f\vec g) = f(\nabla \cdot \vec g) + \vec g \cdot \nabla f
    \end{equation}
    Also note that differential operators and the complex conjugate function commute. Now starting from the second term,
    \begin{align*}
        \nabla \cdot j &= -\frac{i\hbar}{2m} \left[\psi^*(\nabla^2\psi) + (\nabla\psi)\cdot(\nabla^2\psi)^* - \psi(\nabla^2\psi)^* - (\nabla\psi)^*\cdot(\nabla\psi)\right] \\
                       &= -\frac{i\hbar}{2m} \left[\psi^*(\nabla^2\psi) - \psi(\nabla^2\psi)^*\right]
    \end{align*}
    Now rearranging the given equation on top, the Laplacian of the wavefunction is given by
    \begin{equation}
        -\frac{2mi}{\hbar} \del{\psi}{t} + \frac{2m}{\hbar^2} V\psi
    \end{equation}
    Substituting, the second term equals
    \begin{align*}
        -\frac{i\hbar}{2m} \left[\psi^*(\nabla^2\psi) - \psi(\nabla^2\psi)^*\right] &= -\frac{i\hbar}{2m} \left[-\frac{2mi}{\hbar}\psi^*\del{\psi}{t} + \frac{2m}{\hbar^2}V|\psi|^2 - \frac{2mi}{\hbar} \psi\del{\psi^2}{t} - \frac{2m}{\hbar^2} V|\psi|^2\right] \\
                                                                                    &= -\frac{i\hbar}{2m} \times \frac{2mi}{\hbar} \left(\psi\del{\psi^*}{t} + \psi^*\del{\psi}{t}\right) \\
                                                                                    &= -\left(\psi\del{\psi^*}{t} + \psi^*\del{\psi}{t}\right) \\
                                                                                    &= -\del{\psi^*\psi}{t} \\
                                                                                    &= -\del{\rho}{t}
    \end{align*}
    Hence the sum of the two terms vanish.
\end{solution}

\end{questions}

\end{document}
