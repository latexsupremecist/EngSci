\documentclass[12pt]{article}
\usepackage{../../template}
\title{Lecture 2}
\author{niceguy}
\begin{document}
\maketitle

\section{Superpositions of Waves}

\begin{equation}
    \psi(x,t) = \int_{-\infty}^\infty \frac{g(k)}{\sqrt{2\pi}}\exp(i(kx - \omega(k)t))
\end{equation}

We showed that for a nonrelatavistic particle, the group velocity is equal to that of a classical particle. The phase velocity is $\frac{v}{2}$ for a nonrelatavistic particle, and $\frac{c^2}{v}$ otherwise.

\begin{defn}[Group Velocity Dispersion]
    The group velocity dispersion is defined as
    $$\left(\frac{d^2\omega}{dk^2}\right)_{k_0}$$
\end{defn}

In 3D, we know the wavefunction satisfies

\begin{equation}
    i\hbar \del{\psi(\vec r,t)}{t} = -\frac{\hbar^2}{2m} \nabla^2 \psi(\vec r,t)
\end{equation}

More generally,

\begin{equation}
    i\hbar \del{\psi(\vec r,t)}{t} = -\frac{\hbar^2}{2m} \nabla^2 \psi(\vec r,t) + v(\vec r)\psi(\vec r,t)
\end{equation}

We have stationary solutions of the form

$$\psi(\vec r,t) = \phi(\vec r)\chi(t)$$

\begin{align*}
    \phi(\vec r)\frac{d\chi(t)}{dt} &= \left[-\frac{\hbar^2}{2m}\nabla^2\phi(\vec r) + v(\vec r)\phi(\vec r)\right] \chi(t) \\
    \frac{d\chi(t)}{dt}\frac{1}{\chi(t)} &= \left[-\frac{\hbar^2}{2m}\nabla^2\phi(\vec r) + v(\vec r)\phi(\vec r)\right] \frac{1}{\phi(\vec r)}
\end{align*}

Since both sides are functions of $t$ and $\vec r$ respectively, they must be constants. Calling that constant $E$ and integrating,

$$\chi(t) = A\exp\left(-\frac{iEt}{\hbar}\right)$$

Note that $E$ is also an eigenvalue of the operator $H$, where

\begin{equation}
    H = -\frac{\hbar^2}{2m}\nabla^2 + v(\vec r)
\end{equation}

This is obtained by letting RHS be equal to $E$. \\

In some problems, $E$s are discrete, and we have eigenfunctions

\begin{equation}
    H\phi_n(\vec r) = E_n\phi_n(\vec r)
\end{equation}

Writing $\omega_n = \frac{E_n}{\hbar}$, we get

\begin{equation}
    \phi_n(\vec r,t) = \phi_n(\vec r) \exp(-i\omega_nt)
\end{equation}

and the general wavefunction becomes

\begin{equation}
    \psi(\vec r,t) = \sum_n c_n\phi_n(\vec r)e^{-i\omega_nt}
\end{equation}

\section{The Born Rule}

For a normalised wavefunction,

\begin{equation}
    d\mathcal P(\vec r,t) = |\psi(\vec r,t)|^2d\vec r
\end{equation}

This is the probability that a position \textbf{measurement} at time $t$ will find the particle within $d\vec r$ of $\vec r$.

\end{document}
