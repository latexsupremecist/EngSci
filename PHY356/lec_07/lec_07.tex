\documentclass[12pt]{article}
\usepackage{../../template}
\title{Lecture 7}
\author{niceguy}
\begin{document}
\maketitle

\section{}

$$X|\vec r \rangle = x|\vec r \rangle$$
and similar for $Y,y$ and $Z,z$. We have
$$\langle \vec r |X|\vec r' \rangle = x\delta(\vec r - \vec r')$$
Now to find $[X,P_x]$, we take an arbitrary inner product
\begin{align*}
    \langle \vec r | [X,P_x] | \vec \psi \rangle &= \langle \vec r | XP_x - P_xX | \psi \rangle \\
                                                 &= x\langle \vec r |P_x|\psi \rangle - \frac{\hbar}{i} \del{}{x} \langle \vec r | X | \psi \rangle \\
                                                 &= x\langle \vec r |P_x|\psi \rangle + i\hbar \del{}{x} \left(x\langle \vec r|\psi\rangle\right) \\
                                                 &= i\hbar
\end{align*}

\section{}

The only possible result of the measurement of a physical quantity $\mathcal A$ is one of the eigenvalues of the corresponding observable $A$.

When the physical quantity $\mathcal A$ is measured on system in the normalised state $|\psi\rangle$, the probability $P(a_n)$ of obtaining the nondegenerate eigenvalue $a_n$ of the corresponding observable $A$ is
$$P(a_n) = |\langle u_n|\psi\rangle|^2$$

\end{document}
