\documentclass[12pt]{article}
\usepackage{../../template}
\title{Lecture 1}
\author{niceguy}
\begin{document}
\maketitle

\section{Course Information}

Office hours Fridays 14:00 - 17:00 MP408, including reading week. Problem set questions appear on quizzes.

\section{Waves and Particles}

\begin{equation}
    E = h\omega
\end{equation}

\begin{equation}
    p = \frac{h}{\lambda}
\end{equation}

The LHS of both equations are usually linked with \emph{particles}, while the RHS are linked with \emph{waves}. \\
Relativistically, we have

\begin{equation}
    E = \frac{mc^2}{\sqrt{1 - \frac{v^2}{c^2}}}
\end{equation}

\begin{equation}
    p = \frac{mv}{\sqrt{1 - \frac{v^2}{c^2}}}
\end{equation}

And the nonrelatavistic counterparts are trivial.

\subsection{Electrons}
Electrons are used to study surfaces, since they don't penetrate too deeply.

\subsection{Wavefunctions}
In 1 dimension, we have $\cos(kx - \omega t)$ or $\sin(kx - \omega t)$. We also know that these functions are complex in quantum, hence we use $e^{i(kx - \omega t)}$. \\
At $t=0$, we form a wavepacket

\begin{equation}
    \psi(x) = \frac{1}{\sqrt{2\pi}} \int_{-\infty}^\infty g(k)e^{ikx}dk
\end{equation}

A simpler expression is to sum over $k$s instead of integration.

\begin{ex}
    Let
    $$g(k) = \begin{cases} 1 & i = k_0 \\ \frac{1}{2} & i = k_0 \pm \frac{\Delta}{2} \end{cases}$$
    Then
    $$\psi(x) = \frac{e^{ik_0x}}{\sqrt{2\pi}}\left[1 + \cos\left(\frac{\Delta k_0x}{2}\right)\right]$$
\end{ex}

If we introduct time, $e^{ikx}$ becomes $e^{i(kx - \omega(k)t)}$. \\
In a nonrelatavistic case,

\begin{equation}
    E = \frac{p^2}{2m} = \frac{\hbar^2k^2}{2m}
\end{equation}

Assuming superposition,

\begin{align*}
    \psi(x,t) &= \frac{1}{\sqrt{2\pi}} \int_{-\infty}^\infty g(k)e^{i(kx - \omega(k)t)}dk \\
              &= \frac{1}{\sqrt{2\pi}} e^{i(k_0x - \omega(k_0)t)} \int_{-\infty}^\infty g(k)e^{i(k-k_0)x}e^{-i(\omega(k) - \omega_0)t} \\
              &\approx \frac{1}{\sqrt{2\pi}} e^{i(k_0x - \omega_0 t)} \int_{-\infty}^\infty g(k) e^{i(k-k_0)(x - Vt)}
\end{align*}

where we define

\begin{equation}
    V = \left(\frac{d\omega}{dk}\right)_{k_0}
\end{equation}

Letting the final integral be $F(x,t)$, we see $F$ moves at a phase velocity $V$.

For our nonrelatavistic particle,

\begin{equation}
    \left(\frac{d\omega}{dk}\right)_{k_0} = \frac{\hbar k_0}{m} = \frac{p_0}{m}
\end{equation}

For a relativistic particle,

\begin{equation}
    \frac{\omega}{k} = \omega\lambda = \frac{E}{p} = \frac{c^2}{v}
\end{equation}

\end{document}
