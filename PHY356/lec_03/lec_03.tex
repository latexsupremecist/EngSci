\documentclass[12pt]{article}
\usepackage{../../template}
\title{Lecture 3}
\author{niceguy}
\begin{document}
\maketitle

\section{Potential Barriers}

Let's say we have a long wave chain $\psi(x,0)$ that is approaching a potential barrier $V(x)$. We start by approximating $\psi$ to have only one frequency. Then we can write

\begin{equation}
    \psi(x,t) = \phi(x)e^{-i\frac{E}{\hbar}t}
\end{equation}

Substituting into Schr\"odinger's Equation, we get (see complement H$_1$ in textbook)

\begin{equation}\label{phi}
    \frac{d^2}{dx^2}\phi(x) + \frac{2m}{\hbar^2} (E - V(x))\phi(x) = 0
\end{equation}

Now if $E>V$, we put

\begin{equation}
    E - V = \frac{\hbar^2k^2}{2m}
\end{equation}

This substituted into \ref{phi} gives

\begin{equation}\label{biggerV}
    \phi(x) = Ae^{ix} + A'e^{-ikx}, |A| = |A'|
\end{equation}

In quantum mechanics, in fact there is a nonzero probability that the wave is reflected.

\begin{equation}
    \phi(x) = A_2e^{ik_2x}
\end{equation}

If $E<V$, then we put

\begin{equation}
    V - E = \frac{\hbar^2\rho^2}{2m}
\end{equation}

Plugging into \ref{phi} again, we get

\begin{equation}
    \phi(x) = Be^{\rho x} + B'e^{-\rho x}
\end{equation}

Classically, there is total reflection. However, in quantum mechanics, for $x>0$ we have

\begin{equation}
    \phi(x) = B'e^{-px}
\end{equation}

\section{Math facts of Quantum}

\textit{Note: This corresponds to the second chapter of the textbook.}

In section A of the textbook, we discuss the space of a 1 particle wave functions, where the integral of norm squared is 1. We define a vector space $\mathcal F$ with elements being functions that are square integrable and satisfies our physical assumptions (boundary conditions). This is an infinite-dimension vector space.

In sections B - F, we talk about bracket notation. We link $\psi(z) \in \mathcal F$ with $|\psi\rangle \in \mathcal E_{\mathcal F}$. We will look at more general kets later.

For kets $|\phi\rangle \neq |\psi\rangle$ in $\mathcal E$, introduce a scalar product that is a complex number. We define in $\mathacl E_{\vec r}$

\begin{equation}
    \left(|\phi\rangle, |\psi\rangle\right) = \int\phi^*(\vec r)\psi(\vec r) d\vec r
\end{equation}

This is an inner product.

We also introduce a dual space $\mathcal E^*$. Recall a linear functional $\chi$ maps from a ket to a complex number. The dual space is the set of all $\chi$s. In fact, we can prove that all $\chi$ can be written uniquely as

\begin{equation}
    \chi\left(|\psi\rangle\right) = \langle \chi | \psi \rangle
\end{equation}
\end{document}
