\documentclass[12pt]{article}
\usepackage{../../template}
\title{Chapter 3: Basics of Confinement}
\author{niceguy}
\begin{document}
\maketitle

\section{General Need for Confinement}

We need to achieve net energy output, so the Lawson Criteria has to be met, i.e. $n\tau \geq 10^{20}$. We also want to reduce plasma-wall interactions.

\subsection{Wall Erosion}
If there is no confinement,
$$\phi_{D^+} = \frac{1}{4}n\overline c, \overline c = \sqrt{\frac{8kT}{\pi m_e}}$$
For $T = 10\unit{keV}, n = 10^{20}$, we have $\phi_{D^+} \rightarrow 10^{26}$
For carbon, with an erosional yield of around 1\%, that's $10^{24}$ carbon molecules per metre squared. This number is too big.

\subsection{Plasma Contamination}

\begin{ex}[Carbon Walls]
    Carbon walls have $\frac{n_C}{n_D} \leq 1\%$. Carbon Influx is $\Gamma = y\phi_{D^+} \times 2\pi a \times 2 \pi R$.
    \begin{defn}[Impurity Confinement Time]
        $$\tau_{\text{imp}} = \frac{n_CV}{\Gamma_C}$$
        where $V$ stands for volume.
    \end{defn}
    Assume $\tau_{\text{imp}} \approx \tau_E$. Then
    $$y\phi_{D^+} \times 2\pi a \times 2\pi R = n_C \times \frac{2\pi R \times \pi a^2}{\tau_{\text{imp}}}$$
    Using generic values $n_D \approx 10^{20}, y \approx 1\%, a \approx 1\unit{m}, \tau \approx 1\unit{s}$,
    $$\phi_{D^+} \leq 10^{20}$$
    This is a million times smaller than what we conventionally have.
\end{ex}

\section{Magnetic Confinement: General}

\subsection{Maxwell's Equations}

$$\nabla \cdot \vec E = \frac{\rho}{\varepsilon_0}$$

In 1 dimensions, this simplifies to

$$\varepsilon_0 \frac{dE}{dx} = \rho = e|n_i - n_e|$$

We also know

$$\nabla \cdot \vec B = 0$$

meaning there are no magnetic "charges". The remaining laws are

$$\nabla \times \vec E = -\dot{\vec B}$$
$$\nabla \times \vec B = \mu_0\left(\vec j + \varepsilon_0\dot{\vec E}\right)$$

\subsection{Diamagnetism}

There are 3 types of magnetic materials. Diamagnetic materials (induced magnetic field in opposite direction), paramagnetic materials (induced magnetic field in same direction), ferromagnetic (no induced magnetic field). Here we ignore diamagnetic materials.

\subsection{Quasineutrality}

$$n_e \approx n_i$$

Since each negative and positive charge are generated in pairs. Now

$$\nabla \cdot \vec E = \frac{dE}{dx} = \frac{\rho}{\varepsilon_0} = \frac{en}{\varepsilon_0}$$
and
$$E = -\frac{dV}{dx} \Rightarrow V = \frac{nex^2}{2\varepsilon_0}$$
Choosing $V=0,E=0$ at $x=0$,
$$V = \frac{10^{20} \times 1.6 \times 10^{-19} \times 1}{2 \times 8.85 \times 10^{-12}} = 10^{12}\unit{V}$$

\subsection{Debye Shielding}

At the negative electrode, voltage is lower than the voltage of plasma, so $n_e \neq n_i$.

\subsection{Sheath Thickness}

$$\frac{d^2V}{dx^2} = -\frac{e(n_i-n_e)}{\varepsilon_0} \approx -\frac{en_i}{\varepsilon_0}$$

Then
$$V - V_0 = -\frac{enx^2}{2\varepsilon_0}$$

\begin{defn}[Debye Length]
    $$\frac{e\Delta V}{kT_e} = \frac{1}{2}\left(\frac{x}{\lambda_D}\right)^2$$
    Where the Debye length $\lambda_D$ is defined as above. Think of it as a scale length.
\end{defn}

$$\lambda_D = 7430 \left(\frac{T_e}{n}\right)^{1/2}$$
where $T_e$ is in eV.

\begin{ex}
    Given $T = 10\unit{keV}, n = 10^{20}$, we have $\lambda_D = 10^{-4}\unit{m}$.
\end{ex}

Rearranging, the sheath length is
$$x \approx \lambda_D \left(\frac{2e\Delta V}{kT}\right)^{1/2}$$

\begin{ex}
    If $V = 10^4\unit{V}$, then $x \approx 10^{-4}\unit{m}$.
\end{ex}

\subsection{Boltzmann Factor}

Electric force is $F = qE = -enE$. Adding this to pressure gives

$$-(p_2 - p_1)A - enEA\Delta x = 0 \Rightarrow \frac{dP}{dx} = en \frac{dV}{dx}$$

From the ideal gas law, $P = n_ekT_e$, so

$$\frac{dn}{n} = \frac{edV}{kT_e} \Rightarrow n_e = n_\infty e^{eV/kT}$$

\subsection{Shielding Again}

Far from the electrodes, $n_e = n_i = n\infty, v = 0$. Ions are fixed in space. Say $V$ goes from -10 to -1 $\frac{kT_e}{e}$. Then $x \approx 4\lambda_D$.

\begin{align*}
    \varepsilon_0 \frac{d^2V}{dx^2} &= -e(n_i - n_e) \\
                                    &= -e\left(n_i - n_\infty e^{eV/dT}\right) \\
                                    &= en\left(e^{eV/kT} - 1\right) \\
                                    &\approx \frac{e^2nV}{kT}
\end{align*}
    
\end{document}
