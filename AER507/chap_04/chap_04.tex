\documentclass[12pt]{article}
\usepackage{../../template}
\title{Chapter 4: Plasmas}
\author{niceguy}
\begin{document}
\maketitle

\section{Introduction}

Plasmas light up when energy from line radiation is in the optical range, sort of like burning metals. However, flames light up because of blackbody radiation, which is different. However, flames can still ionize air molecules up to a certain degree, like plasmas (but weaker). There are a wide range of values that they can take, e.g. $n < 1\unit{cm^{-3}}, T \approx 1\unit{K}$ to $n \geq 10^{26}\unit{cm^{-3}}, T \approx 10^{11}\unit{K}$.

Plasma is usuall created from gases. It can be produced by

\begin{itemize}
    \item Electron impact ionization
    \item Ion impact
    \item Fast neutral
    \item X-rays, lasers, etc
    \item Others, see Dolan 49, 50
\end{itemize}

It can be destroyed by

\begin{itemize}
    \item Volume Recombination
        \begin{itemize}
            \item Radiative recombination
                Where ions and electrons combine to form a neutral, releasing $E = h\nu$
            \item 3-body recombination
                Two electrons and an ion collide, producing an electron and a neutral
        \end{itemize}
    \item Surface Recombination
        Charged particles are attracted to a surface, where they recombine
\end{itemize}
                

\end{document}
