\documentclass[12pt]{article}
\usepackage{../../template}
\title{Lecture 5}
\author{niceguy}
\begin{document}
\maketitle

\section{Cyclotron Radiation}

The Lorentz Force is $F = 2e\vec v \times \vec B = 2ev^\perp B$. The centrifugal force is $\frac{m\left(v^\perp\right)^2}{R}$, and the gyro radius is $R = \frac{mv^\perp}{zeB}$. Time taken per cycle is $\tau = \frac{2\pi R}{v^\perp}$. The gyrofrequency is $\Omega = \frac{2\pi}{\tau} = \frac{zeB}{m}$.

\begin{ex}
    For $B = 5\unit{T}$ and an electron, we get
    $$\Omega = \frac{1.6\times10^{-19} \times 5}{9.1\times10^{-31}} = 10^{12} \unit{rad.s^{-1}}$$
    The mean free path is then
    $$\lambda_{cy} = \frac{2\pi c}{\Omega} \approx 2\unit{mm}$$
    Similarly, we get $\Omega = 2.4 \times 10^8 \unit{rad.s^{-1}}$ for D$^+$.
    Its mean free path is approximately 8 m.
\end{ex}

Power is

$$P_{\text{rad}} = \frac{1}{4\pi\varepsilon_0} \frac{2}{3} \frac{e^2}{c^3} a^2n$$
where
$$a = \frac{\left(v^\perp\right)^2}{R} = \frac{ev^\perp B}{m}$$
Plugging $\left(v^\perp\right)^2 \approx \frac{2kT}{m}$, we get
$$P_{\text{cy}} = \frac{1}{3\pi\varepsilon_0} \frac{e^4B^2kT}{c^2m^3} n$$
Or
$$P_{\text{cy}} = 6.21 \times 10^{-17} n_eT_eB^2$$
where $T_e$ is in keV and $n$ is in $\unit{m^{-3}}$. Another way to derive it is

$$\beta = \frac{P_P}{P_M} = \frac{n_ekT_e + n_ikT_i}{B/2\mu_0} = \frac{4nkT\mu_0}{B^2}$$
Then
$$P_{\text{cy}} = \frac{4}{3} \frac{n_e^2(kT)^2e^4\mu_0}{\pi\varepsilon_0m_e^3c^2\beta}$$

\section{Lawson Criteria}

\subsection{Including Conversion Efficiency}

We define

\begin{align*}
    \tau &= \frac{\text{total heat content}}{\text{total heating rate}} \\
         &= \frac{\int \frac{3k}{2}\left(n_eT_e + n_DT_D + n_TT_T\right)dV}{P_{\text{in}}}
\end{align*}

\begin{ex}
    For $n_D = n_T = \frac{1}{2} n_i = \frac{1}{2} n_e$, $T_e = T_D = T_T$, we have
    $$\tau = \frac{3n_ikT}{P_{\text{in}}}$$
\end{ex}

Breakeven requires $\epsilon(P_{\text{in}} + P_{\text{fusion}} = P_{\text{fusion}}$, where $\epsilon$ is the efficiency of converting heat to electricity. Continuing from the previous example,

\begin{align*}
    (1-\epsilon)P_{\text{in}} &= \epsilon P_{\text{fusion}} \\
    (1-\epsilon) \frac{3n_ikT}{\tau} &= \epsilon \frac{n_i}{2} \frac{n_i}{2} \overline{\sigma v} E_{DT} \\
    n_i\tau &= \frac{12(1-\epsilon)kT}{\epsilon\overline{\sigma v}E_{DT}} \\
    n_i\tau &= \frac{24kT}{\overline{\sigma v}E_{DT}}
\end{align*}

Where we take $\epsilon = \frac{1}{3}$.

\subsection{Specific Energy Loss Mechanisms}

Ignoring radiative effects, there is only conduction and convection. Similarly, we define

$$\tau_{CC} = \frac{3n_ikT}{P_{CC}}$$

Considering sum of power,
$$P_{\text{in}} = P_{\text{rad}} + P_{\text{cc}}$$

And we get

$$n_i\tau_{CC} = \frac{12(1-\epsilon)kT}{\epsilon\overline{\sigma v}E_{DT} - 4(1-\epsilon)C_1\sqrt{T} - 4(1-\epsilon)c_2T^2}$$
where
$$P_{\text{rad}} = P_{\text{br}} + P_{\text{cy}} = c_1n_D^2\sqrt{T} + c_2n_D^2T^2$$

Recall from thermodynamic that

$$Q = K\frac{\Delta T}{\Delta x}$$

where $K$ is the thermal conductivity. For a torus with radii $a$ and $R$ (bigger one in bigger letters), we have
$$\frac{\Delta T}{\Delta x} \approx \frac{T_0}{a}$$
Total conducted loss is then
$$Q_{TOT} = 2\pi a \times 2\pi R \times Q$$

Loss per volume is

$$P_C = \frac{Q_{TOT}}{2\pi R\pi a^2} = \frac{2Q}{a} = \frac{2KT_0}{a^2}$$

Recall

$$K = nk\chi$$
where $k$ is the Boltzmann constant, and $\chi$ is thermal diffusivity. Experimentally, $\chi$ is inversely proportional with $B$, and is around $1 \unit{m^2.s^{-1}}$ at $B = 3\unit{T}$.

$$P_C = \frac{2n\chi kT}{a^2}$$
and
$$\tau_{CC} \approx \frac{3nkT}{P_C} = \frac{3nkT}{2nkT\chi/a^2}$$
so
$$\tau_{\text{non-rad}} \approx \frac{a^2}{\chi}$$
For $\tau_{nr}$ at 1 second, we have $a$ around 1 metre and $\chi$ around 1 metre squared per second.

\section{Impurities Can Prevent Ignition}

$P_z \unit{W.m^{-3}}$

\begin{itemize}
    \item Bremsstrahlung proprtional to $z^2$
    \item Line Radiation (see tables)
\end{itemize}

$$P_z \approx n_en_zL_(T)$$
where the last term is tabulated. \\

Ignition requires $P_z < P_\alpha$. Letting $n_e = n_D + n_T + zn_z$,
$$\frac{n_D}{n_e} = \frac{1}{2}(1 - zf_z)$$
where
$$f_z = \frac{n_z}{n_e}$$
Plugging into the inequality,
\begin{align*}
    n_en_zL_z(T) &< n_Dn_T\overline{\sigma v}(T)E_\alpha \\
    f_zL_z(T) &< \frac{1}{4}(1-zf_z)^2\overline{\sigma v}(T)E_\alpha \\
    \frac{f_z}{(1-zf_z)^2} &< \frac{\overline{\sigma v}(T)E_\alpha}{4L_z(T)}
\end{align*}

Note that we can allow for more impurities as temperature goes up. Factors include more ionisation.

\subsection{Impurity Production Mechanisms}

\begin{itemize}
    \item Sputtering
    \item Chemical Erosion
    \item Evaporation
    \item Melting
\end{itemize}

\end{document}
