\documentclass[12pt]{article}
\usepackage{../../template}
\title{Lecture 3}
\author{niceguy}
\begin{document}
\maketitle

\section{Definitions}

We write down the definitions of some terms here for convenience.

\begin{table}[h]
    \begin{center}
    \begin{tabular}{|c||c|}
        \hline
        Flux Density & $\Gamma = nv$ \\
        \hline
        Reaction Rate & $r = n_bn_t\sigma v_b$ \\
        \hline
        Collision Frequency & $\nu = v_bn_t\sigma$ \\
        \hline
        Mean Free Path & $\lambda_b = \frac{1}{n_t\sigma}$ \\
        \hline
        Beam Attenuation & $\Gamma = \Gamma_0e^{-x/\lambda}$ \\
        \hline
    \end{tabular}
    \end{center}
\end{table}

\section{Thermonuclear Fusion}

The mean free path $\lambda$ is on the order of $10^7\unit{m}$. This means we need to make the particles go really fast. If velocity follows a Maxwellian Distribution. Defining $dN$ to be the particles with energy between $E$ and $E+dE$, the distribution is represented by $f(E) = \frac{dN}{dE}$ which tends to $E^{-1/2}e^{-E/kT}$.

Recall

$$r = n_Dn_T \overline{\sigma v} = n_Dn_T \overline{\sigma(E)\sqrt{\frac{2E}{m}}} = n_Dn_T \sqrt{\frac{2}{m}} \int f(E)\sigma(E)\sqrt{E}dE$$

Substituting $f(E)$, we eventually get

$$r = \frac{4}{\sqrt{\pi}} \left(\frac{M_r}{2kT}\right)^{3/2} \int_0^\infty \sigma(v)v^3e^{-M_rv^2/2kT}dv$$

Where $v$ is the relative velocity and $M_r = \frac{m_1m_2}{m_1+m_2}$ is the reduced mass.

\begin{ex}
    For monogenergetic plasma $E_D = E_T = 15\unit{keV}$, with mean approach energy around 15 keV, if we take the mean energy, we have $\sigma$ around $2 \times 10^{-30}$ and $v$ around $2.4 \times 10^6$. Then their product is $2.4 \times 10^{-24}$.

    If it's 10 keV, we get $\overline{\sigma v} = 1.09 \times 10^{-22}$.
\end{ex}

\section{Energy Distribution Among Fusion Products}

With conservation of momentum, we get

\begin{align*}
    \frac{E_1}{E_2} &= \frac{m_1v_1^2}{m_2v_2^2} \\
                    &= \frac{v_1}{v_2} \\
                    &= \frac{m_2}{m_1}
\end{align*}

where we use $m_1v_1 = m_2v_2$ in the last equality.

$$\text{D} + \text{T} \rightarrow \text{He}(3.5\unit{MeV}) + n(14.1\unit{MeV})$$

The benefits of this are that

\begin{itemize}
    \item There is no heat transfer problem for thermal electricity generation, since neutrons can travel for a long distance
    \item It also allows neutrons to reach Li for tritium breeding.
\end{itemize}

The disadvantages are that

\begin{itemize}
    \item Neutron Radiation
        \begin{itemize}
            \item There is material damage because of neutron bombardment
            \item There is induced radioactivity
            \item Transmutations produce hydrogen and helium that leads to embrittlement
        \end{itemize}
    \item Thermal cycle for electricity generation (thermodynamics)
    \item Only 20\% of the energy is available for self-heating plasma
\end{itemize}

\begin{defn}[Ignition]
    Ignition refers to self-sustaining plasma.
\end{defn}

Exotic Fuels

$$\text{D} + ^3\text{He} \rightarrow p(14.7\unit{MeV}) + ^4\text{He}(3.6\unit{MeV})$$
$$\text{H} + ^{11}B \rightarrow 3^4\text{He}$$

\section{The Need for Plasma Confinement}

\subsection{Energy Balance}

If we have a confinement time a hundredth of fusion time, energy output is around 5-6 times the energy input. Then confinement time becomes
$$\tau_C = \frac{1}{100n\overline{\sigma v}} = \frac{1}{100n\times10^{-22}}$$
Rearranging yields $n\tau_C \approx 10^{20}$. The Lawson Criteria is thus
$$n\tau_C \geq 10^{20}$$

\subsection{}

Plasma pressure is

$$P_p = n_DkT_D + n_TkT_T + n_ekT_e$$

Plasma is quasi-neutral, meaning net charge is practically 0, or $n_e = n_D + n_T$. For $T=10\unit{keV}$ and $n_D=n_T=10^{20}$, pressure is 6.4 atm.

\subsection{Electrostatics}

Net force acting on each particle due to an electric field is $eE$. Combining this with pressure,

$$\frac{\Delta p}{\Delta x} = -enE$$

From Maxwell's Equations,

$$\nabla \cdot \vec E = \frac{ne}{\varepsilon_0} \Rightarrow \frac{dE}{dx} = \frac{ne}{\varepsilon_0}$$

Substituting,

$$\frac{dp}{dx} + E\varepsilon_0\frac{dE}{dx} = 0 \Rightarrow \frac{d}{dx}\left(p + \frac{1}{2} \varepsilon_0 E^2\right) = 0$$

Then the function in the derivative is a constant. Noting that the constant has to be greater than plasma pressure at some point ($E\neq0$), then in the exterior, we need the plasma pressure to be less than $\frac{1}{2}\varepsilon_0E^2_{\text{ext}}$. The math shows that given the values of $\varepsilon_0$ and plasma pressure, we need $E \approx 10^9 \unit{V/m}$.

\subsection{Magnetic Fields}

$$P_m = \frac{B^2}{2\mu_0}$$

This gives $B = 1.6\unit{T}$, which is feasible.

\end{document}
