\documentclass[answers]{exam}
\usepackage{../../template}
\title{Assignment 4}
\author{Daniel Chua}
\begin{document}
\maketitle

% ask list
% 2: How fraction

\begin{questions}

\question{It is desired that neutral tritium atom beams be able to penetrate at least 2 m (mean free path) into a reactor plasma with $n = 10^{20} \unit{m^{-3}}$ and $T_e = T_i = 10 \unit{keV}$. What energy must the beams have? Is this practical?}

\begin{solution}
    For electron ionization, the tritium beam is essentially stationary, so we plug electron temperature in. This gives $\sigma v = 10^{-14}$.
    \begin{align*}
        \lambda &= \frac{v_T}{n\sigma v} \\
        2 &= \frac{v_T}{10^{20} \times 10^{-14}} \\
        v_T &= 2 \times 10^6
    \end{align*}
    and energy is at least
    $$\frac{1}{2}mv^2 = \frac{1}{2} \times 5.007 \times 10^{-27} \times 4 \times 10^{12} = 1.001 \times 10^{-14}\unit{J} = 62.6\unit{keV}$$
    For proton ionization, the proton moves very slowly, so we plug in beam energy instead. Then for $E$ in units of eV,
    $$\lambda = \frac{1}{n\sigma} \Rightarrow 2 = \frac{1}{10^{20}\sigma} \Rightarrow \sigma = 5 \times 10^{-21}$$
    Then
    $$\sigma v = 5 \times 10^{-21} \sqrt{\frac{2E \times 1.6 \times 10^{-19}}{m}} = 4.00 \times 10^{-17} \sqrt{E}$$
    From the graph, this gives the energy to be around $4 \times 10^5\unit{eV} = 400\unit{keV}$. Similarly, this is the minimum energy; for a greater mean free path, a higher energy is needed. Finally, for charge exchange, we similarly have interactions between protons and the electron beam. Then we get the same equation. It intersects with the curve for $\sigma v$ at $E = 10^5\unit{eV} = 100\unit{keV}$.

\end{solution}

% 1
% Fig 3D1, p56
% e ionization depends on sigma v (integrated over something)
% p ionization: one sigma, one v

\question{Neutral H atoms with 3 eV energy are incident on a hydrogen plasma with $n_e = 10^{19}\unit{m^{-3}}, T_e = T_i = 1\unit{keV}$. What processes will be significant? About how far will the atoms penetrate before having any kind of reaction? About what fraction of the atoms will cause charge exchange?}

% 2.1 cm, 80%

\begin{solution}
    At 1 keV and 3eV, according to figure 3D1 (p56) of Dolan, charge exchange and electron ionization are more significant, with $\sigma v > 10^{-14}$. Proton ionization does not occur. For charge exchange, plugging energy to be 3 eV gives $\sigma v = 1.5 \times 10^{-14}$. For electron ionization, putting $T_e = 1\unit{keV}$ gives $\sigma v = 2 \times 10^{-14}$. Total cross section is then $3.5 \times 10^{-14}$.
    Velocity of H is
    $$v = \sqrt{\frac{2E}{m}} = \sqrt{\frac{2\times 3 \times 1.6 \times 10^{-19}}{1.67 \times 10^{-27}}} = 23950$$
    hence
    $$\sigma = 4 \times 10^{-14} \div 23950 = 1.46 \times 10^{-18}$$
    and mean free path is
    $$\lambda = \frac{1}{n\sigma} = \frac{1}{10^{19}\times 8.35 \times 10^{-19}} = 0.0684 \unit{m}$$
    Atoms are expected to penetrate 6.84 cm before charge exchange occurs. \\
    To find the fraction of atoms that cause charge exchange, first note that $r\tau$ gives the reactions per volume of atoms that cause charge exhange. Dividing this by atom density gives the fraction of atoms that cause charge exchange. We can simplify this to
    \begin{align*} 
        \frac{r\tau}{n_H} &= \frac{n_Hn_i\langle\sigma v\rangle_{\text{charge exchange}}\tau}{n_H} \\
                          &= n_i\langle\sigma v\rangle_{\text{charge exchange}}\times\frac{\lambda}{v} \\
                          &= 10^{19} \times 1.5 \times 10^{-14} \times \frac{0.0684}{23950} \\
                          &= 0.429 \\
                          &= 42.9\%
    \end{align*}
\end{solution}

\question{Debye Shielding. In order for shielding to occur, ie., for a plasma to exist (as distinct from ionized matter) the particle density must be large enough for some particles to exist, on average, in a distance $\lambda_D$, the Debye length. Find what conditions this imposes on $n$ and $T$. Are fusion plasmas likely to satisfy this condition? What about the following?}

\begin{solution}
    Assume each particle occupies the space of a sphere. We want the radius of that sphere to be $\lambda_D$ or smaller. This means we want
    $$nV \geq 1$$
    Substituting,
    \begin{align*}
        \frac{4}{3} \pi \lambda_D^3 n &\geq 1 \\
        \frac{4\pi}{3} 7430^3 T_e^{3/2} n^{-1/2} &\geq 1 \\
        n &\leq 2.95 \times 10^{24} T_e^3
    \end{align*}
    For fusion plasmas, we take a conservative estimate of $n=10^{20}$ and $T = 10\unit{keV}$. Then
    $$2.95 \times 10^{24} \times 10^3 > 10^{20} = n$$
    so fusion plasmas are likely to satisfy this. \\
    For a glow discharge,
    $$2.95 \times 10^{24} \times 2^3 > 10^{16} = n$$
    so the condition is satisfied. For the ionosphere,
    $$2.95 \times 10^{24} \times 0.1^3 = 2.95 \times 10^{21} > 10^{12} = n$$
    so the condition is also satisfied. For interstellar space,
    $$2.95 \times 10^{24} \times 10^{-6} = 2.95 \times 10^{18} > 10^6 = n$$
    so the condition is also satisfied.
\end{solution}

\question{The typical distance between two electrons in a plasma is of order $n_e^{-1/3}$. Show that the potential energy associated with bring two electrons this close together is much less than their kinetic energy, so long as $n_e\lambda_D^3 > 1$}

\begin{solution}
    Potential energy is on the order of
    \begin{align*}
        U &= \frac{e^2}{4\pi\varepsilon_0r}\unit{J} \\
          &= \frac{e}{4\pi\varepsilon_0n_e^{-1/3}}\unit{eV} \\
          &= 1.438 \times 10^{-9} n_e^{1/3}
    \end{align*}
    Kinetic energy is on the order of $T_e$. Now substituting the definition of $\lambda_D$, we get
    \begin{align*}
        n_e\lambda_D^3 &> 1 \\
        n_e \times 7430^3 T_e^{3/2} n_e{-3/2} &> 1 \\
        7430^3 T_e^{3/2} &> n_e^{1/2} \\
        7430^2 T_e &> n_e^{1/3} \\
        1.438 \times 10^{-9} \times 7430^2 T_e &> 1.438 \times 10^{-9}n_e^{1/3} \\
        0.0794 T_e &> U
    \end{align*}
    Hence kinetic energy is much greater than potential energy.
\end{solution}

\question{Langmuir probe is inserted into a plasma with $n_e = 3 \times 10^{17} \unit{m^{-3}}$, and biased positive with respect to the plasma potential such that ions are repelled and electrons collected. If the probe surface area is $0.2 \unit{mm^2}$ and the current collect is 5 mA, what is the electron temperature of the plasma?}

\begin{solution}
    Electron flux density is
    $$\frac{5 \times 10^{-3}}{1.6 \times 10^{-19} \times 0.2 \times 10^{-6}} = 1.563 \times 10^{23}$$
    Then using the equation for particle flux density,
    \begin{align*}
        \Gamma &= \frac{1}{4} n \sqrt{\frac{8kT}{\pi m}} \\
        1.563 \times 10^{23} &= \frac{1}{4} \times 3 \times 10^{17} \sqrt{\frac{8kT}{\pi m}} \\
        T &= 9.69 \unit{eV}
    \end{align*}
\end{solution}

\question{}

\begin{solution}
    For $\vec B = 0\vec r + B_\phi\hat \phi + 0\hat z$,
    $$\frac{1}{\mu_0} (\vec B \cdot \nabla)\vec B = \frac{1}{\mu_0} (B_\phi\hat \phi \cdot \vec \nabla) B_\phi \hat\phi$$
    We can use the expression in Dolan for $(\vec A \cdot \nabla)\vec B$. Since $\vec B$ has no components in the $\hat r$ and $\hat z$ directions, all the partial derivative terms for $B_r$ and $B_z$ vanish. By symmetry, $\del{B_\phi}{\phi} = 0$. The term with $B_r$ also vanishes, since $\vec B$ does not have a component in that direction. The remaining term is
    $$-\frac{1}{r} B_\phi^2\hat r$$
    Putting this back into the original equation, we get
    $$-\frac{B^2_\phi}{\mu_0r}\hat r$$
    as desired. Now for
    $$-\frac{1}{2\mu_0}\nabla B^2$$
    again by symmetry, the partial derivatives with respect to $\phi$ and $z$ vanish. The remaining term is
    \begin{align*}
        -\frac{1}{2\mu_0}\nabla B^2 &= -\frac{1}{2\mu_0} \del{B^2}{r} \hat r \\
                                    &= -\frac{1}{2\mu_0} \del{B_\phi^2}{r} \hat r
    \end{align*}
    Now
    $$B_\phi(r) = \frac{\mu_0I}{2\pi r} = \begin{cases} \frac{\mu_0j_0r}{2} & r \leq a \\ \frac{\mu_0j_0a^2}{2r} & r > a \end{cases}$$
    For $r\leq a$,
    \begin{align*}
        -\frac{B_\phi^2}{\mu_0r} &= -\frac{1}{\mu_0r} \times \frac{\mu_0^2j_0^2r^2}{4} \\
                                 &= -\frac{\mu_0j_0^2r}{4}
    \end{align*}
    and
    \begin{align*}
        -\frac{1}{2\mu_0} \del{B_\phi^2}{r} &= -\frac{1}{2\mu_0} \del{}{r} \frac{\mu_0^2j_0^2r^2}{4} \\
                                            &= -\frac{1}{2\mu_0} \frac{\mu_0^2j_0^2r}{2} \\
                                            &= -\frac{\mu_0j_0^2r}{4} \\
                                            &= -\frac{B_\phi^2}{\mu_0r}
    \end{align*}
    so both contributions are the same. For $r > a$, we have
    \begin{align*}
        -\frac{B_\phi^2}{\mu_0r} &= -\frac{1}{\mu_0r} \times \frac{\mu_0^2j_0^2a^4}{4r^2} \\
                                 &= -\frac{\mu_0j_0^2a^4}{4r^3}
    \end{align*}
    and
    \begin{align*}
        -\frac{1}{2\mu_0} \del{B_\phi^2}{r} &= -\frac{1}{2\mu_0} \del{}{r} \frac{\mu_0^2j_0^2a^4}{4r^2} \\
                                            &= -\frac{1}{2\mu_0} (-2) \frac{\mu_0^2j_0^2a^4}{4r^3} \\
                                            &= \frac{\mu_0j_0^2a^4}{4r^3} \\
                                            &= -\left(-\frac{B_\phi^2}{\mu_0r}\right)
    \end{align*}
    Both contributions have the same magnitude. However, for $r \leq a$, the contributions add up (constructive), and for $r > a$, they cancel each other out (destructive).
\end{solution}
% 6
% Dolan Appendix F
% Consider both r <= a and r > a
\end{questions}
\end{document}
