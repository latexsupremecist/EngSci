\documentclass[12pt]{article}
\usepackage{../../template}
\author{niceguy}
\title{Lecture 16}
\author{niceguy}
\begin{document}
\maketitle

\section{Incompressible Fluid at Rest}
$\rho$ is a constant for incompressible fluids. Then from
$$\frac{dp}{dz} = -\rho g$$
we get
$$p_2-p_1 = -\rho g(z_2-z_1)$$
as derived in the previous lecture. 

\begin{defn}
	The specific weight is defined as
	$$\gamma := \rho g$$
\end{defn}

We define specific weight as it appears often in fluid dynamics. It is the weight per unit volume of the fluid. Moreover, since liquids are usually much denser than gases

\begin{align*}
	\rho_g &<< \rho_f \\
	\rho_g g &<< \rho_f g \\
	\left|\left(\frac{dp_g}{dz}\right)\right| &<< \left|\left(\frac{dp_f}{dz}\right)\right|
\end{align*}

Therefore, for small distances $\Delta h \approx 0$, we can take gas pressure to be constant.

\begin{ex}
	Find the pressure-elevation relationship for isothermal perfect gas.
	$$p = \rho RT$$
	Then
	\begin{align*}
		\frac{dp}{dz} &= -\rho g \\
		\frac{dp}{dz} &= -\frac{p}{RT} g \\
		\frac{dp}{p} &= -\frac{g}{RT}dz \\
	\ln\frac{p_2}{p_1} &= -\frac{g}{RT}(z_2-z_1) \\
	p_2 &= p_1e^{-\frac{g}{RT}(z_2-z_1)}
\end{align*}
\end{ex}

As a summary, assuming no shear,
$$-\vec{\nabla}p + \rho\vec{g} = \rho\vec{a}$$
where $\vec{g}$ is a body force. If it is gravity, we have
$$-\vec{\nabla}p - \rho g\hat{k} = \rho\vec{a}$$
If the fluid is also at rest, we have
$$\frac{dp}{dz} = -\rho g$$
For incompressible fluids at rest,
$$p_2 = p_1+\rho gh$$
and for compressible fluids at rest,
$$\int dp = \int \rho gdz$$
which simplifies to a constant $p$ for small ranges $\Delta h$.

\begin{ex}
	For an incompressible fluid at rest, in a gravitational field acting in the $-z$ direction, show that the free surface is horizontal. \\
	Given the conditions, we have
	$$p = -\rho gz + C$$
	We know $p = p_{atm}$ at $z = z_s$, so
	$$p_{atm} = -\rho gz_s+C$$
	Rearranging yields
	$$z_s = \frac{C-p_{atm}}{\rho g}$$
	which is an expression of constants. The $z$ component of the surface is a constant, hence the surface is horizontal.
\end{ex}

\section{Measurements of Pressure}
Pressure values are stated with respect to a reference level. The gage pressure is measured with a local atmospheric reference, while absolute pressure is compared with a vacuum reference. Hence
$$p_{abs} = p_{gage} + p_{atm}$$
Several instruments for measuring pressure including
\begin{itemize}
	\item Mercury Barometer: $P_a = \rho gh$
	\item Piezometer: $P_a = \rho gh$
	\item U-tube manometer: $P_a = \gamma_2h_2 - \gamma_1h_1$
\end{itemize}

\begin{ex}
	A closed tank contains air and water. A piezometer is connected to the tank as shown. The column height are $h_1=1$m,$h_2=0.5$m. If the pressure gage reading of the compressed air shows 10kPa, determine the height $h$ of the water in teh piezometer. \\
	$$\gamma(h-1) = 10000 \Rightarrow h = 2$$
\end{ex}
\end{document}
