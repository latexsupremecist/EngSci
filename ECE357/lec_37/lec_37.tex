\documentclass[12pt]{article}
\usepackage{../../template}
\title{Lecture 37}
\author{niceguy}
\begin{document}
\maketitle

\begin{ex}
    An air filled rectangular waveguide has cross-section $a = 5\unit{cm}, b = 2\unit{cm}$. A mode is
    $$\tilde H_z = \cos(40\pi x)e^{-j\frac{40\pi\sqrt{7}}{3}z}$$
    Since $\tilde H_z \ne 0$, it is TE. There is no attenuation along the $z$ direction, so it is propagating. Since it has no $y$ dependence, $n = 0$, and
    $$\frac{m\pi x}{a} = 40\pi x \Rightarrow m = 2$$
    This is the TE$_{2,0}$ mode. The frequency can be found by
    $$\beta = k\sqrt{1 - \left(\frac{f_c}{f}\right)^2}$$
    Cutoff frequency is
    $$f_c = \frac{1}{2\sqrt{\varepsilon\mu}} \sqrt{\left(\frac{m}{a}\right)^2 + \left(\frac{n}{b}\right)^2} = 20c_0 = 6\unit{GHz}$$
    Plugging into the formula above,
    \begin{align*}
        \frac{40\pi\sqrt{7}}{3} &= \frac{2\pi f}{3\times10^8} \sqrt{1 - \left(\frac{6\times10^9}{f}\right)^2} \\
        f &= 8\unit{GHz}
    \end{align*}
    Find the range of frequencies where the guide supports a single mode. \\
    For $(m,n)$ pairs, the two lowest are $(1,0)$ at $3\unit{GHz}$ and $(2,0)$ at $6\unit{GHz}$. Then this is the range of frequencies.
\end{ex}

\end{document}
