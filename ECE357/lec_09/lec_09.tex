\documentclass[12pt]{article}
\usepackage{../../template}
\title{Lecture 9}
\author{niceguy}
\begin{document}
\maketitle

\section{Special cases of Transmission Lines}

$$Z_{\text{in}} = Z_0 \frac{Z_L + jZ_0\tan(\beta l)}{Z_0 + jZ_L\tan(\beta l)}$$

\begin{enumerate}
    \item Matched Line: $Z_L = Z_0$. \\
        This implies $Z_{\text{in}} = Z_0$, and there is no impedance tranformation, no standing wave.
    \item $l = \frac{n\lambda}{2}, n \in \Z^+$ \\
        Then $\beta l = \frac{2\pi}{\lambda} \frac{n\lambda}{2} = n\pi$. Then its tangent vanishes, and $Z_{\text{in}} = Z_L$. This is then a half-wave line.
    \item $l = (n + \frac{1}{2}) \frac{\lambda}{2}, n \in \N$ \\
        The tangent term becomes $\tan(\beta l) = \tan\left[\frac{2\pi}{\lambda} \frac{\lambda}{2}\left(n+\frac{1}{2}\right)\right] = \tan\left(n\pi + \frac{\pi}{2}\right) = \pm\infty$. This implies $Z_{\text{in}} = \frac{Z_0^2}{Z_L}$. This is the quarter-wave transformer, or impedance inverter.
\end{enumerate}

\begin{defn}[Normalised Impedance]
    $$z = \frac{Z}{Z_0}$$
\end{defn}

The quarter-wave transformer "inverts impedance" because
$$z_{\text{in}} = \frac{Z_0}{Z_L} = \frac{1}{z_L}$$

\begin{ex}
    Use a quarter wave transformer to \textit{match} a 5$\Omega$ load to a $Z_0 = 50\unit\Omega$ transmission line. \\
    We add a transmission line between the existing one and the load, with $Z_0'$. Then
    \begin{align*}
        \frac{Z_0}{Z_0'} &= \frac{Z_0'}{R_L} \\
        \frac{50}{Z_0'} &= \frac{Z_0'}{5} \\
        Z_0' &= 5\sqrt{10} \\
             &= 158\unit{\Omega}
    \end{align*}
    where the equation comes from the results of a quarter-wave transformer.
\end{ex}

\section{Open-Circuited Transmission Line}

As $Z_L \rightarrow \infty$, we have
$$Z_{\text{in}} = \frac{1}{jY_0\tan(\beta l)}, Y_0 = \frac{1}{Z_0}$$
which is the characteristic admittance (recall admittance is purely imaginary). Also
$$Y_{\text{in}} = jY_0\tan(\beta l)$$

Since the sign of $\tan(\beta l)$ changes, it can act like an inductor or capacitor depending on its arguments. Susceptance against $\beta l$ becomes a tangent curve, so it starts out as a capacitor and alternates.

\section{Short-Circuited Transmission Line}

Obviously $Z_{\text{in}} = jZ_0\tan(\beta l)$. A similar graph can be sketched for reactance, but it starts out acting as an inductor.

\begin{ex}
    Use a short-circuited transmission line ($Z_0 = 50\unit{\Omega}$) to implement $C = 4\unit{pF}$ at $f = 2.25\unit{GHz}$. Phase velocity of the line is $0.75c \approx 2.25 \times 10^8\unit{m.s^{-1}}$. \\
    The impedance of the capacitor is
    $$Z = \frac{1}{j\omega C} = \frac{1}{j2\pi fC} = \frac{1}{j2\pi\times2.25\times10^9\times4\times10^{-12}} = -j17.684\unit\Omega$$
    We have
    $$\beta = \frac{2\pi}{\lambda} = \frac{2\pi f}{v} = \frac{4.5\times10^9\pi}{2.25\times10^8} = 20\pi$$
    Equating the impedance with $Z_0\tan(\beta l)$,
    \begin{align*}
        Z_0\tan(\beta l) &= -17.684 \\
        50\tan(20\pi l) &= -17.684 \\
        \tan(20\pi l) &= -0.354 \\
        l &= (4.46 + 5n) \unit{cm}
    \end{align*}
\end{ex}

\end{document}
