\documentclass[12pt]{article}
\usepackage{../../template}
\title{Lecture 27}
\author{niceguy}
\begin{document}
\maketitle

\begin{ex}
    A 10 Ghz aircraft radar uses a narrow-beam scannign antenna mounted on a gimbal behind a dielectric radome. Even though the radome shape is far from planar, it is approximately planar over the narrow extent of the radar be (couldn't finish copying) \\
    At 10 GHz, the wavelength in air is 3 cm, and that in the material is
    $$\lambda = \frac{\lambda_0}{\sqrt{\varepsilon_r}} = \frac{3}{3} = 1\unit{cm}$$
    From transmission lines, there will be no reflection if thickness (total length) is an integer multiple of $\frac{\lambda}{2}$, so the radome wil be stable for $d = 2.5\unit{cm}$.
\end{ex}

\section{Incidence at an angle}

Unlike in the normal case, we need to find the angles of reflection and transmission. All angles are measured with respect to the normal. The wavevectors are
\begin{align*}
    \vec k_i &= k_0n_1(\sin\theta_i\hat x + \cos\theta_i \hat z) \\
    \vec k_r &= k_0n_1(\sin\theta_r\hat x - \cos\theta_r \hat z) \\
    \vec k_t &= k_0n_2(\sin\theta_t\hat x + \cos\theta_t\hat z)
\end{align*}

We can split the $\vec E$ field into parallel and perpendicular components.

\begin{align*}
    \tilde{\vec E} &= E_0(\cos\theta_i \hat x - \sin\theta_i \hat z)e^{-jk_0n_1\sin\theta_ix}e^{-jk_0n_1\cos\theta_iz} \\
    \tilde{\vec H} &= \frac{1}{\eta_1} e^{-jk_0n_1\sin\theta_ix}e^{-jk_0n_1\cos\theta_iz} \hat y
\end{align*}

For the parallel components, the reflected waves are

\begin{align*}
    \tilde{\vec{E_r}} &= \Gamma_\parallel (\cos\theta_i\hat x + \sin\theta_i\hat z)e^{-jk_0n_1\sin\theta_i x}e^{jk_0n_1\cos\theta_iz} \\
    \tilde{\vec{H_r}} &= -\frac{\Gamma_\parallel}{\eta_1} e^{-jk_0n_1\sin\theta_ix}e^{jk_0n_1\cos\theta_iz} \hat y
\end{align*}

and the transmitted waves are

\begin{align*}
    \tilde{\vec E_t} &= \tau_\parallel(\cos\theta_t\hat x - \sin\theta_t\hat z)e^{-jk_0n_2\sin\theta_tx}e^{-jk_0n_2\cos\theta_tz} \\
    \tilde{\vec H_t} &= \frac{\tau_\parallel}{\eta_2} e^{-jk_0n_2\sin\theta_tx}e^{-jk_0n_2\cos\theta_tz}
\end{align*}

Applying boundary conditions,

$$\cos\theta_ie^{-jk_0n_1\sin\theta_ix} + \Gamma_\parallel\cos\theta_re^{-jk_0n_1\sin\theta_ix} = \tau_\parallel\cos\theta_te^{-jk_0n_1\sin\theta_ix}$$

\end{document}
