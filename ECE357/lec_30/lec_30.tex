\documentclass[12pt]{article}
\usepackage{../../template}
\title{Lecture 30}
\author{niceguy}
\begin{document}
\maketitle

\section{Brewster Angle}

It is the angle $\theta_i = \theta_B$ for which $\Gamma_t(\theta_B) = 0$. For the parallel case,

$$\sin^2\theta_B = \frac{1 - \frac{\mu_2\varepsilon_1}{\mu_1\varepsilon_2}}{1 - \left(\frac{\varepsilon_1}{\varepsilon_2}\right)^2}$$

For interfaces between non-magnetic media, $\mu_1 = \mu_2 = \mu_0$, so this simplifies to

$$\theta_B = \arcsin\frac{1}{\sqrt{1 + \frac{\varepsilon_1}{\varepsilon_2}}}$$

For the perpendicular case,

$$\sin^2\theta_B = \frac{1 - \frac{\mu_1\varepsilon_2}{\mu_2\varepsilon_1}}{1 - \left(\frac{\mu_1}{\mu_2}\right)^2}$$

The brewster angle exists only for media with different $\mu$.

\begin{ex}
    Consider a plane wave travelling from air to a medium with $\varepsilon_r = 4$. Both are non-magnetic. The incident electric field is
    $$\tilde{\vec{E_i}} = (4\hat x + 5\hat y - 4\hat z) e^{-j\sqrt{2}\pi(x+z)}$$
    From the exponential term, $\sin\theta_i = \cos\theta_i$, so $\theta_i = \frac{\pi}{4}$. Since $n_1 = 1$ for air, we have $k_0 = 2\pi$. The wavelength is then 1m, and the frequency is $300\unit{MHz}$. \\
    To find the reflected wave, we can use linear superposition. For the parallel component, $\theta_r = \theta_i$, and
    \begin{align*}
        \tilde{\vec{E_r}} &= \Gamma_\parallel E_0(\cos\theta_i\hat x + \sin\theta_i\hat z)e^{-jk_0n_1(x\sin\theta_i - z\cos\theta_i)} \\
                         &= \Gamma_\parallel (\hat x + \hat z)e^{-j\sqrt{2}\pi(x-z)}
    \end{align*}
    For the perpendicular component,
    $$\tilde{\vec{E_r}} = 5\Gamma_\perp \hat y e^{-j\sqrt{2}\pi(x-z)}$$
    Now $\eta_1 = 120\pi, \eta_2 = 60\pi$. Snell's law provides $\theta = 20.705^\circ$, which gives
    $$\Gamma_\parallel = -0.2038, \Gamma_\perp = -0.4514$$
    Substituting these yield the full result.
\end{ex}

\end{document}
