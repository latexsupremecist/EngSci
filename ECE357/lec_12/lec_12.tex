\documentclass[12pt]{article}
\usepackage{../../template}
\title{Lecture 12}
\author{niceguy}
\begin{document}
\maketitle

\section{Smith Chart Basic Operations}

\subsection{Impedance Transformation}

For $Z_0 = 50\unit{\Omega}, Z_L = 130 + j90$, we can find the normalised load $z_L = \frac{Z_L}{Z_0} = 2.6 + j1.8$. The reflection coefficient is
$$\Gamma(d=0) = \Gamma_L = \frac{z_L - 1}{z_L + 1} = 0.6e^{j21.8^\circ} \Rightarrow \Gamma_d = 0.6e^{j(21.8^\circ - 2\beta d)}$$

The standing wave ratio is $\frac{1 + |\Gamma|}{1 - |\Gamma|}$. Recall also
$$z = \frac{1 + \Gamma_d}{1 - \Gamma_d} = \frac{1 + |\Gamma|e^{j(\theta_\Gamma - 2\beta d)}}{1 - |\Gamma|e^{\theta_\Gamma - 2\beta d}}$$

When $\theta_\Gamma - 2\beta d = 2n\pi$, $z$ becomes the standing wave ratio. We can read the SWR value as the value of $r$ at the crossing of the positive real axis. \\

$$2\beta d = 2 \times \frac{2\pi}{\lambda} d = \frac{4\pi d}{\lambda}$$

For $d = 0.3\lambda$, we can rotate (clockwise) the complex $\Gamma$ along the outer scale (from 0.25) by 0.3, and this gives $z_{\text{in}} = 0.26 + j0.12$. Denormalising, $Z_{\text{in}} = 12.7 + j5.8\unit{\Omega}$. The distance to the maxima is $(0.25 - 0.22)\lambda = 0.03\lambda$, where $0.25\lambda$ is known to be the maximum, and $0.22\lambda$ is the angle of $\Gamma$ according to $z$. The minimum is a quarter wavelength after that, at $0.28\lambda$.

\textit{Fun fact: a reflection about the origin/rotation by $\pi$ gives the admittance, since}

\begin{align*}
    y &= \frac{1}{z} \\
      &= \frac{1 - \Gamma_d}{1 + \Gamma_d} \\
      &= \frac{1 + (-\Gamma_d)}{1 - (-\Gamma_d)}
\end{align*}

\end{document}
