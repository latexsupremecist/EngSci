\documentclass[12pt]{article}
\usepackage{../../template}
\author{niceguy}
\title{Lecture 1}
\begin{document}
\maketitle

\section{Introduction}

\subsection{Course Outline}

\begin{itemize}
    \item Theoretical Study and applications of E\&M fields and waves
    \item Electromagnetic wave propagation
    \item Wave interaction with natural media
    \item Waveguides (wires but for waves)
\end{itemize}

This course is not going to be similar to ECE259.

\subsection{Deliverables}

Weekly online quizzes, labs with lab reports, see syllabus.

\section{Transmission Lines}

A transmission line is composed of 2 conductors embedded in a dielectric medium. Conductors include Al, Au, Ag, Cu, and dielectrics include air, glass, plastic, etc. Dielectrics are there to prevent short circuits. Arrangements can be coaxial, parallel-plate, microstrip, etc.

\begin{ex}[Analysis of Parallel Plate Transmission]
    Let the $x$ axis be the normal, and $y$ and $z$ axis be along the sides of the rectangular plates. For the dielectric in between, permittivity is $\varepsilon = \varepsilon_0\varepsilon_r$, and permeability is $\mu = \mu_0\mu_r$. \\
    Consider a "slice" of length $\Delta z$. Treating this as a capacitor, capacitance is
    $$C = \varepsilon_0\varepsilon_r \frac{w\Delta z}{h}$$
    Per unit length, capacitance is
    $$C' = \varepsilon_0\varepsilon_r \frac{w}{h}$$
    If we treat it as an inductor, using our knowledge from ECE259,
    $$L = \frac{\mu h\Delta z}{w}$$
    and similarly
    $$L' = \frac{\mu h}{w}$$
    If we consider it as a resistor, current would be constant between $z$ and $z+\Delta z$, which is untrue. Therefore, we cannot solve these by na\"ively treating them as resistors.
\end{ex}

\end{document}
