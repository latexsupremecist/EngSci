\documentclass[12pt]{article}
\usepackage{../../template}
\title{Lecture 29}
\author{niceguy}
\begin{document}
\maketitle

\section{Recap}

For a parallel case,

\subsection{Incident}

\begin{align*}
    \tilde{\vec E}_i &= (\cos\theta_i\hat x - \sin\theta_i\hat z)e^{-jk_0n_1(x\sin\theta_i + z\cos\theta_i)} \\
    \tilde{\vec H}_i &= \frac{1}{\eta_1}e^{-jk_0n_1(x\sin\theta_i + z\cos\theta_i)} \hat y
\end{align*}

\subsection{Reflected}

\begin{align*}
    \tilde{\vec E_r} &= \Gamma_\parallel (\cos\theta_r\hat x + \sin\theta_r\hat z)e^{-jk_0n_1(x\sin\theta_r - z\cos\theta_r)} \\
    \tilde{\vec H_r} &= \frac{\Gamma_\parallel}{\eta_1} e^{-jk_0n_1(x\sin\theta_r - z\cos\theta_r)} \hat y
\end{align*}

\subsection{Transmitted}

\begin{align*}
    \tilde{\vec E_t} &= \tau_\parallel(\cos\theta_t \hat x - \sin\theta_t\hat z)e^{-jk_0n_2(x\sin\theta_t + z\cos\theta_t)} \\
    \tilde{\vec H_t} &= \frac{\tau_\parallel}{\eta_2} e^{-jk_0n_2(x\sin\theta_t + z\cos\theta_t)} \hat y
\end{align*}

\section{Total Internal Reflection}

For $n_1 > n_2$, there is $\theta_i = \theta_c$ where $\sin\theta_t = 1$. At an angle greater than that, $\sin\theta_t > 1$, and
$$\cos\theta_t = \sqrt{1 - \sin^2\theta_t} = \pm j\sqrt{\left(\frac{n_1}{n_2}\sin\theta_i\right)^2 - 1}$$

This is multiplied by $-jk_0n_2$ in the exponent. We cannot admit a solution that grows to infinity, so the only accepted solution is the negative $j$ solution. The reflection coefficient becomes complex and has a magnitude of 1, since

$$\Gamma_\parallel = \frac{\eta_2\cos\theta_t - \eta_1\cos\theta_i}{\eta_2\cos\theta_t + \eta_1\cos\theta_i} = \frac{jx - y}{jx + y}$$

\section{Perpendicular Case}

The electric field is perpendicular to the plane spanned by the wavevector and the normal.

\subsection{Incident}

\begin{align*}
    \tilde{\vec H_i} &= \frac{1}{\eta_1} (-\cos\theta_i\hat x + \sin\theta_i\hat z) e^{-jk_0n_1(x\sin\theta_i + z\cos\theta_i)} \\
    \tilde{\vec E_i} &= e^{-jk_0n_1(x\sin\theta_i + z\cos\theta_i)} \hat y
\end{align*}

\subsection{Reflected}

\begin{align*}
    \tilde{\vec H_r} &= \frac{\Gamma_\perp}{\eta_1} (\cos\theta_r \hat x + \sin\theta_r\hat z)e^{-jk_0n_1(x\sin\theta_r - z\cos\theta_r)} \\
    \tilde{\vec E_r} &= \Gamma_\perp e^{-jk_0n_1(x\sin\theta_r - z\cos\theta_r)}\hat y
\end{align*}

\subsection{Transmitted}

\begin{align*}
    \tilde{\vec H_t} &= \frac{\tau_\perp}{\eta_2} (-\cos\theta_t \hat x + \sin\theta_t \hat z)e^{-jk_0n_2(x\sin\theta_t + z\cos\theta_t)} \\
    \tilde{\vec E_t} &= \tau_\perp e^{-jk_0n_2(x\sin\theta_t + z\cos\theta_t)}\hat y
\end{align*}

Similar to the parallel case, we can use boundary conditions to find the coefficients

$$\Gamma_\perp = \frac{\frac{\eta_2}{\cos\theta_t} - \frac{\eta_1}{\cos\theta_i}}{\frac{\eta_2}{\cos\theta_t} + \frac{\eta_1}{\cos\theta_i}}$$
and
$$\tau_\perp = \frac{\frac{2\eta_2}{\cos\theta_t}}{\frac{\eta_2}{\cos\theta_2} + \frac{\eta_1}{\cos\theta_i}}$$
\end{document}
