\documentclass[12pt]{article}
\usepackage{../../template}
\title{Lecture 31}
\author{niceguy}
\begin{document}
\maketitle

\section{Waveguides}

Electromagnetic waves attenuate exponentially in conductors. Therefore, ordinary cables are unsuitable for high frequency power transfer. The alternative is to use waveguides. Normally, we have waves that propagate along the $z$ axis. We now want them to do so along the axis of the waveguide.

\subsection{Parallel plates}

Consider a wave travelling approximately along a pair of parallel plates. Due to imperfections, it travels at an angle and is reflected along each of the plates. Using the convention where $x$ is normal to the plates,

$$\vec k_i = k\cos\theta_i\hat x + k\sin\theta_i\hat z = k_x\hat x + \beta\hat z$$

Then
\begin{align*}
    \tilde{\vec{H_i}} &= H_0e^{-jk_xx}e^{-j\beta z}\hat y \\
    \tilde{\vec{H_r}} &= -\Gamma_\perp H_0e^{jk_xx}e^{-j\beta z}\hat y \\
    \Gamma_\perp &= \frac{\eta_2\cos\theta_t - \eta_1\cos\theta_i}{\eta_2\cos\theta_t + \eta_1\cos\theta_i}
\end{align*}

As $\eta_2 \rightarrow 0, \Gamma_\perp \rightarrow -1$, which gives

$$\tilde{\vec H} = H_0e^{-j\beta z}\left(e^{-jk_xx} + e^{jk_xx}\right)\hat y = 2H_0\cos(k_xx)e^{-j\beta z}\hat y$$

To find the electric field more quickly, we can use

$$\nabla \times \tilde{\vec H} = j\omega\varepsilon \tilde{\vec E}$$

Then

\begin{align*}
    \tilde{\vec E} &= -\frac{j}{\omega\varepsilon} \nabla \times \tilde{\vec H} \\
                   &= -\frac{j}{\omega\varepsilon} \left(\hat x\del{}{x} + \hat z\del{}{z}\right) \times \tilde H_y \hat y \\
                   &= \frac{2H_0e^{-j\beta z}}{\omega\varepsilon}(\beta\cos(k_xx)\hat x + jk_x\sin(k_xx)\hat z)
\end{align*}

We know that $\vec E$ has no $z$ component at the boundaries $x=0,x=a$, since there it meets a conductor, and $\vec E$ fields only penetrate perpendicularly, These conditions give $k_x = \frac{n\pi}{a}$. There is another limit, since

$$\beta^2 + k_x^2 = k^2 = \omega^2\varepsilon\mu$$

\begin{defn}[Cutoff Frequency]
    The cutoff frequency of the $n$th mode is
    $$\omega_{c,n} = \frac{n\pi}{a\sqrt{\varepsilon\mu}}$$
    The $n$th mode cannot exist below this frequency.
\end{defn}

When $\omega_{c,n} > \omega$, then
$$\beta = -jk\sqrt{\left(\frac{\omega_{c,n}}{\omega}\right)^2 - 1}$$

This is called the \textit{evanescent} mode. It attenuates with constant
$$a_n = k\sqrt{\left(\frac{\omega_{c,n}}{\omega}\right)^2 - 1}$$

If not, we have the \textit{propagating} mode. This gives a high-pass filter, since in the evanescent mode, $\beta$ becomes imaginary, so $e^{-j\beta z}$ decays. \\

For the $n = 0$ mode, $k_x = 0$, so $\tilde E_z$ vanishes, and
$$\tilde E_x = \frac{\beta}{\varepsilon\omega} 2H_0e^{-j\beta z} = 2\eta H_0e^{-j\beta z}$$
There is no cutoff. This means for frequencies below $\omega_{c,1}$, there is only one possible $k_x$.
\end{document}
