\documentclass[12pt]{article}
\usepackage{../../template}
\title{Lecture 13}
\author{niceguy}
\begin{document}
\maketitle

\section{Smith Chart on a Network Analyzer}

\section{Smith Chart as Admittance Chart}

\begin{ex}
    Find the input admitance of an open-circuited TL of characteristic impedance 300 Ohms and length $0.04\lambda$. \\
    At open circuit, conductance is 0, which is the leftmost point on the smith chart. Rotate by 0.04 towards the generator (clockwise).
    $$Y_i = j0.26 \times Y_0 = \frac{j0.26}{300} = j0.87\unit{mS}$$
\end{ex}

\section{Impedance Matching}

We want to match a line. On a Smith chart, this is equivalent to moving an arbitrary point and moving it to the origin. There are 2 simple cases for this.

\subsection{Case 1}

$$z_L = 1+jx$$
where a capacitor/inductor can be put in series to reduce $z_L$ to 1, or an open/closed circuit transmission line with suitable length. Usually open circuits are easier.

\subsection{Case 2}

$$y_L = 1 + jb$$

If we connect in parallel a reactive load with susceptance $-b$, admittance becomes 1 (recall resistance in parallel). Similarly, this can be implemented with suitable short/open circuits.

\subsection{Arbitrary Case}

With $z_L = r+jx$, on a Smith chart, we can always rotate this about the origin until it intersects the $r=1$ circle. Let this distance be $\Delta d$, then at $d = \Delta d$, we have reduced the problem to our first case. We attach a purely reactive load at this point, and we are done.

\end{document}
