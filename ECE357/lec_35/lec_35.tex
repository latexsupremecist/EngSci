\documentclass[12pt]{article}
\usepackage{../../template}
\title{Lecture 35}
\author{niceguy}
\begin{document}
\maketitle

\section{Group Velocity}

Even if we start out with a simple sinusoidal signal, it often ends up as wavepackets, where low frequency envelopes are modulated by high frequency waves. We then get
$$f(t) = g(t)\sin(\omega_0t)$$

The bandwidth $\Delta w$ of the envelope signal is much less than $\omega_0$.

\begin{ex}
    Consider 2 waves with similar frequencies $\omega_0 \pm \frac{\Delta\omega}{2}$. The faster signal propogates with
    $$\beta(\omega_+) \approx \beta(\omega_0) + \frac{\Delta\omega}{2} \beta'(\omega_0)$$
    and the same holds for $\omega_-$. For the first case,
    $$\cos(\omega_+t - \beta_+z) = \cos\left((\omega_0t - \beta(\omega_0)z) + \frac{\Delta\omega}{2}(t - \beta'(\omega_0)z)\right)$$
    and similarly for the second case
    $$\cos(\omega_-t - \beta_-z) = \cos\left((\omega_0t - \beta(\omega_0)z) - \frac{\Delta\omega}{2}(t - \beta'(\omega_0)z)\right)$$
    Using the identity
    $$\cos(x+y) + \cos(x-y) = 2\cos x\cos y$$
    their sum becomes
    $$2\cos(\omega_0t - \beta(\omega_0)z)\cos\left(\frac{\Delta\omega}{2}t - \frac{\Delta\omega}{2}\beta'(\omega_0)z\right)$$
    The second term is the low frequency envelope, and the first term is the high frequency wave, which propagates at the same velocity as the initial waves.
\end{ex}

From the above example, the velocity of the envelope itself is
$$\frac{\Delta\omega}{2} \div \frac{\Delta\omega}{2}\beta'(\omega_0) = \frac{1}{\beta'(\omega_0)}$$
The definition becomes obvious.

\begin{defn}
    The group velocity is defined as
    $$v_g = \frac{1}{\beta'(\omega_0)} = \frac{1}{\del{\beta}{\omega}(\omega_0)}$$
\end{defn}

For non-TEM waveguide modes,
$$\beta = k\sqrt{1 - \left(\frac{f_c}{f}\right)^2} = \sqrt{\varepsilon\mu}\sqrt{\omega^2-\omega_c^2}$$
The phase velocity is
\begin{align*}
    v_p &= \frac{\omega}{\beta} \\
        &= 1/\sqrt{\varepsilon\mu} \times \frac{1}{\sqrt{1 - \left(\frac{\omega_c}{\omega}\right)^2}} > \frac{1}{\sqrt{\varepsilon\mu}}
\end{align*}
and the group velocity is
\begin{align*}
    \del{\beta}{\omega} &= \sqrt{\varepsilon\mu} \frac{\omega}{\sqrt{\omega^2-\omega_c^2}} \\
                    &= \sqrt{\frac{\varepsilon\mu}{1 - \left(\frac{\omega_c}{\omega}\right)^2}} \\
    v_g &= \frac{1}{\del{\beta}{\omega}} \\
        &= \sqrt{\frac{1 - \left(\frac{\omega_c}{\omega}\right)^2}{\varepsilon\mu}}
\end{align*}
It is immediately seen that $v_gv_p = \frac{1}{\varepsilon\mu} = c^2$.

\begin{ex}
    For transmission lines, $\beta = \omega\sqrt{L'C'}$, so $v_p = \frac{1}{\sqrt{L'C'}} = v_g$. For a backwards wave line, $\beta = \pm \frac{1}{\omega\sqrt{L'C'}}$. In this case, $v_p = \omega^2\sqrt{L'C'}, v_g = -\omega^2\sqrt{L'C'}$. They have different directions! Similarly, we can have planar negative index lens.
\end{ex}

\end{document}
