\documentclass[12pt]{article}
\usepackage{../../template}
\title{Lecture 33}
\author{niceguy}
\begin{document}
\maketitle

\section{TE/TM modes in waveguides}

For guided waves, we can identify the $z$ dependance by
$$\tilde{\vec E}(x,y,z) = \tilde{\vec e}(x,y)e^{-j\beta z}$$
and similarly for $\tilde{\vec H}$. We can substitute into Maxwell's equations to solve for these unknowns.

\begin{align*}
    \vec\nabla \times \tilde{\vec H} &= j\omega\varepsilon \tilde{\vec E} \\
    \del{\tilde H_z}{y} - \del{\tilde H_y}{z} &= j\omega\varepsilon \tilde E_x \\
    \del{\tilde h_z}{y} + j\beta\tilde h_y &= j\omega\varepsilon \tilde e_x
\end{align*}

This is repeated for all 6 equations. Rearranging, $\tilde E_x, \tilde E_y, \tilde H_x, \tilde H_y$ can be rewritten as linear combinations of partial derivatives of $\tilde E_z, \tilde H_z$.

\begin{ex}
    Consider a cross section. $x \in [0,a], y \in [0,b]$, and the material outside is a perfect conductor. For rectangular waveguides in TM mode, $\tilde E_z = \tilde e_z(x,y)e^{-j\beta z}$. It has to satisfy the wave equation
    \begin{align*}
        \left(\del{^2}{x^2} + \del{^2}{y^2} + \del{^2}{z^2} + k^2\right) \tilde e_z(x,y)e^{-j\beta z} &= 0 \\
        (\del{^2\tilde e_z}{x^2} + \del{^2\tilde e_z}{y^2} - \beta^2\tilde e_z + k^2\tilde e_z)e^{-j\beta z} &= 0 \\
        \left(\del{^2}{x^2} + \del{^2}{y^2} + k^2 - \beta^2 \right) \tilde e_z(x,y) &= 0
    \end{align*}
    At the boundaries, $e_z$ has to vanish. The solution is then
    $$\tilde e_z(x,y) = \sin\left(\frac{n\pi x}{a}\right)\sin\left(\frac{m\pi y}{b}\right), (n,m) \in \N^2$$
    With
    $$\left(\frac{n\pi}{a}\right)^2 + \left(\frac{m\pi}{b}\right)^2 = k^2 - \beta^2$$
    Then the propogation constant is
    $$\beta_{n,m} = \sqrt{k^2 - \left(\frac{n\pi}{a}\right)^2 - \left(\frac{m\pi}{b}\right)^2}$$
    This can be rewritten as
    $$\beta_{n,m} = k \sqrt{1 - \frac{\omega_{c,n,m}^2}{\omega^2}}, \omega_{c,n,m} = \frac{1}{\sqrt{\varepsilon\mu}} \sqrt{\left(\frac{n\pi}{a}\right)^2 + \left(\frac{m\pi}{b}\right)^2}$$
\end{ex}

We could likewise construct the TE mode.

\end{document}
