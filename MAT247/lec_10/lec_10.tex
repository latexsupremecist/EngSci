\documentclass[12pt]{article}
\usepackage{../../template}
\author{niceguy}
\title{Lecture 10}
\begin{document}
\maketitle

\section{Orthogonal Projections}

Recall $P\in\mathcal L(V)$ is a projection iff

$$P^2 = P$$

Then $I - P$ is also a projection, and all vectors $v$ can be decomposed by

$$v = Pv + (I-P)v = v_1 + v_2$$
or
$$V = V_1 \oplus V_2$$

where

$$V_2 = \text{ran}(I-P) = \text{null}(P)$$

Conversely, for any direct sum decomposition with two subspaces $V = V_1 \oplus V_2$, we get a $P$. \\

If $V$ is an inner product space over $\R$ or $\C$, we call $P$ an orthogonal projection iff there is an orthogonal decomposition. Conversely, if $V = W \oplus W^\perp$, we get a corresponding $P = P_W$. \\

From the last lecture, given an orthogonal projection $P$, then

$$||Pv|| \leq ||v||$$

\begin{prop}
	Let $V$ be an inner product space, $P \in \mathcal L(V)$ a projection. Then $P$ is an orthogonal projection iff $||Pv|| \leq ||v|| \forall v \in V$.
\end{prop}

\begin{proof}
	We proved the "only if" part in last lecture. For "if", we want to show

	$$V = \text{ran}(P) \oplus \text{null}(P)$$

	is an orthogonal decomposition, so every $v \in \text{null}(P)$ is orthogonal to ran$(P)$. Since both have the same dimensions, it suffices to show

	$$\text{null}(P)^\perp \subseteq \text{ran}(P)$$

	Supposed $v \in \text{null}(P)^\perp$. Write

	$$Pv = v - (I - P)v$$

	This is an orthogonal decomposition, since $v \in \text{null}(P)^\perp$, $(I-P)v \in \text{ran}(I-P) = \text{null}(P)$. Therefore

	$$||Pv||^2 = ||v||^2 + ||(I-P)v||^2 \geq ||v||^2$$

	Then

	$$||v|| \geq ||Pv|| \geq ||v|| \Rightarrow ||Pv|| = ||v||$$

	i.e. $||(I-P)v|| = 0$. So $Pv = v$, and $v \in \text{ran}(P)$.
\end{proof}

\begin{prop}
	Let $V$ be an inner product space, $W \subseteq V$, where $W$ is finite dimensiona. Let $v_1,\dots,v_n$ be an orthonormal basis of $W$. Then
	$$P_w(v) = \sum_{i=1}^n \langle v,v_i \rangle v_i$$
\end{prop}

\begin{proof}
	Use $V = W \oplus W^\perp$. The formula is true for all $v \in W^\perp$ and $v \in W$.
\end{proof}

\begin{thm}\label{mythm}
	Let $V = W \oplus W^\perp$, $v \in V, w \in W$. Then
	$$P_w(v) - v || \leq ||w-v||$$
	with equality iff $w = P_w(v)$.
\end{thm}

\begin{proof}
	$$w - v = (P_w(w) - v) + (w - P_w(v))$$
	where the first term is an element of $W^\perp$ and the second term is an element of $W$. Then the Pythagorean theorem shows that
	$$||w-v||^2 = ||P_w(v) - v||^2 + ||v - P_w(v)||^2 \geq ||P_w(v) - v||^2$$
	with equality iff the dropped term is 0.
\end{proof}

\begin{ex}
	Let $V$ be continuous real functions on $[-\pi,\pi)$ with
	$$\langle f,g \rangle = \int_{-\pi}^\pi f(x)g(x)dx$$
	Last time, we showed that
	$$f_n(x) = \frac{1}{\sqrt{\pi}}\sin(nx), n \in \N$$
	are all orthonormal. Then which linear combination
	$$\sum_{i=1}^n a_if_i(x)$$
	is the best approximation to $f(x) = x$? I.e. such that
	$$||f - \sum_{i=1}^n a_if_i||$$
	is minimized. \\
	By \ref{mythm}, the best approximation is the orthogonal projection of $f$ on
	$$W = \text{span}\{f_1,\dots,f_n\}$$
	so
	$$P_w(f) = \sum_{i=1}^n \langle f,f_i \rangle f_i$$
	So the oordinate $a_i$ become
	
	\begin{align*}
		a_i &= \langle f,f_i \rangle \\
		    &= \frac{1}{\sqrt{\pi}} \int_{-\pi}^\pi x\sin(nx)dx \\
		    &= -\frac{1}{n\sqrt{\pi}} (x\cos(nx)) |_{-\pi}^\pi + \frac{1}{n\sqrt{\pi}} \int_{-\pi}^\pi \cos(nx)dx \\
		    &= \frac{2\sqrt{\pi}}{n}(-1)^{n+1}
	\end{align*}

	So
	$$P_w(f) \approx 2\sum_{i=1}^n \frac{(-1)^{i+1}}{i}\sin(ix)$$
\end{ex}

\section{Adjoint Operators}

Consider $\F^n$, where $\F$ is real or complex, and
$$\mathcal L(\F^m,\F^n) = M_{n\times m}(\F)$$
For $A \in M_{m \times n}(\F)$ define the conjugate transpose

$$A^* = \overline{A}^t$$

If $\langle , \rangle$ is the standard inner product on $\F^n, \F^m$, we have

$$\langle Av,w \rangle = \langle v,A^*w \rangle$$

For all $v \in \F^m, w \in \F^n$.

\begin{proof}
	\begin{align*}
		\langle Av,w \rangle &= (Av)^t\overline{w} \\
				     &= v^tA^t\overline{w} \\
				     &= v^t \overline{\overline{A^t}w} \\
				     &= v^t\overline{A^*w} \\
				     &= \langle v, A^*w \rangle
	\end{align*}
\end{proof}

\begin{thm}
	Let $V,W$ be inner product spaces with finite dimensions. $forall T \in \mathcal L(V,W)$ there is a unique $T^* \in \mathcal{L}(W,V)$ such that
	$$\langle Tv,w \rangle = \langle v,T*w \rangle$$
\end{thm}
\end{document}
