\documentclass[12pt]{article}
\usepackage{../../template}
\author{niceguy}
\title{Lecture 10}
\begin{document}
\maketitle

\section{Recap}

The posulate of statistical mechanics is that in a closed system, the thermodynamic equilibrium is most likely found in a state of maximum entropy defined as
$$S(E,V,N) = k\ln\Omega(E,V,N)$$
We then define
$$\left(\frac{\partial S}{\partial E}\right)_{V,N} = \frac{1}{T}$$
This makes sense intuitively. At a low temperature, a small increase in energy would lead to a greater increase in $\Omega$ than at a higher temperature. \\
We can then think of a thermodynamic "force" which is the negative gradient of $S$. Entropy can be thought as thermodynamic potential in a closed system. \\
In general, if $\frac{q}{N} >> 1$, then
$$S = kN\ln\frac{Ec}{\hbar\omega N}$$
The first derivative of $S$ with respect to $E$ is positive, but the second derivative is negative.
Now for Einstein solids,
$$U = -\mu_0BS, S = 2N_\uparrow - N$$
Note that potential is maximum at 
\end{document}
