\documentclass[12pt]{article}
\usepackage{../../template}
\author{niceguy}
\title{Lecture 12}
\begin{document}
\maketitle

\section{Ideal Gas (Cont'd)}

We derived the approximation

$$S(U,N,V) = kN\left(\ln\left[\frac{V}{N}\left(\frac{4\pi mU}{3Nh^2}\right)^{1.5}\right]+\frac{5}{2}\right)$$
And differentiating for temperature gives
$$U = \frac{3}{2} kNT$$
This is a simplification, as this gives negative entropy at $T$ when we fix density $\frac{V}{N}$. Therefore, the ideal gas law does not hold if it is too cold.

\subsection{De Broglie Wavelength}

$$\lambda = \frac{h}{p}$$
We define a \textbf{thermal de Broglie Wavelength} by
$$\lambda_{\text{th}} \propto \frac{h}{\sqrt{mkT}}$$
Then
$$S = kN\left[\ln\frac{\overline{l}^3}{\lambda_{\text{th}}^3}+C\right]$$
where $C$ is a constant, and $\overline{l}$ is the average distance between particles. Essentially, we can assume the model to be classical if the particles are not within each other's de Broglie wavelength.

\end{document}
