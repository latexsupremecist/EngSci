\documentclass[answers]{exam}
\usepackage{.../../template}
\author{niceguy}
\title{Problem Set 9}
\begin{document}
\maketitle

\begin{questions}

\question{Consider a system of two Einstein solids, $A$ and $B$, each containing 10 oscillators, sharing a total of 20 units of energy. Assume that the solids are weakly coupled, and that the total energy is fixed.}

\begin{parts}
	\part{How many different macrostates are available to this system?}
	\part{How many different microstates are available to this system?}
	\part{Assuming that this system is in thermal equilibrium, what is the probability of finding all the energy in solid $A$?}
	\part{What is the probability of finding exactly half of the energy in solid $A$?}
	\part{Under what circumstances would this system exhibit irreversible behavior?}
\end{parts}

\begin{solution}
	There are 21 macrostates and $binom{39}{19}$ microstates. \\
	The probability that all the energy is in solid $A$ is
	$$\frac{\binom{29}{9}}{\binom{39}{19}}$$
	The probability that half the energy is in solid $A$ is
	$$\frac{\binom{19}{10}^2}{\binom{39}{19}}$$
	There is no "irreversible" behaviour, as every state is equally probable. However, the process from an "unlikely" macrostate to a "likely" macrostate is semi-irreversible, as it is very unlikely to go back.
\end{solution}

\question{Use the methods of this section to derive a formula, similar to equation 2.21, for the multiplicity of an Einstein solid in the "low-temperature" limit, $q << N$.}

\begin{solution}
	asdf
\end{solution}

\end{questions}

\end{document}
