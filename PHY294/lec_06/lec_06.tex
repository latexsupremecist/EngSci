\documentclass[12pt]{article}
\usepackage{../../template}
\title{Lecture 6}
\author{niceguy}
\begin{document}
\maketitle

\section{Postulate}

The picture (not proof) behind the postulate is "true randomness" in the behaviour of large $N$ systems.

\begin{itemize}
	\item Consider an isolated gas with energy $E$. With $10^{23}$ particles, there are many ways for the energy to be distributed
	\item Collisions change the way energy is distributed, i.e. microstates
	\item We assume any such states are as likely as the other
\end{itemize}

We give up a mechanical description for statistics and probability, which is easier.

\section{Electronic Paramagnet}

We have $N$ sites, and we only care about the total magnetic spin. The states are totally determined by $\{s_1,\dots,s_n\}$, where $s_i$ takes the value of $\pm1$. We neglect interaction between spins. Now the energy is described as
$$U = -\vec{\mu}\cdot\vec{B}$$
where $U$ describes the potential energy, hence the negative sign. Total energy is hence
$$U = \sum_{i=1}^N -\mu_0s_iB = -\mu_0B\sum_{i=1}^N s_i$$
Then energy only depends on sum of spins. Now total spin
$$S = \sum_{i=1}^N s_i$$
takes all values from $-N$ to $N$ in increments of 2 (flipping a spin changes $S$ by 2). Now there are $N+1$ macrostates but $2^N$ microstates. Now, we call a microstate "accessible" to a macrostate when the microstate corresponds to the macrostate. In this case, it is all combinations of $s_i$ which add up to the same $S$, which is essentially n choose k.
\end{document}
