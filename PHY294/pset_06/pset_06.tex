\documentclass[answers]{exam}
\usepackage{../../template}
\title{Problme Set 6}
\author{niceguy}
\begin{document}
\maketitle

\begin{questions}

\question{List all intermolecular interactions that take place in each of the following kinds of molecules: Xe, SO$_2$, C$_6$H$_5$F and LiF.}

\begin{solution}
	Xe: Dispersion \\
	SO$_2$: Dispersion, dipole-dipole \\
	C$_6$H$_5$F: Dispersion, dipole-dipole \\
	LiF: Dispersion, ionic
\end{solution}

\question{The compounds Br$_2$ and ICl have the same number of electrons, yet Br$_2$ melts at -7.2$^\circ$C, whereas ICl melts at 27.2$^\circ$C.  Explain.}

\begin{solution}
	Br$_2$ is non polar, as both Br molecules are identical. ICl is polar, bound by stronger dipole-dipole forces. Therefore, ICl has a higher melting point.
\end{solution}

\question{List the types of intermolecular forces that exist between molecules or (basic units) in each of the following species:}

\begin{parts}
	\part{Benzene}
	\part{CH$_3$Cl}
	\part{PF$_3$}
	\part{NaCl}
	\part{CS$_2$}
\end{parts}

\begin{solution}
	Dispersion; dispersion, dipole-dipole; dispersion, dipole-dipole; dispersion, ionic; dispersion
\end{solution}

\question{Two water molecules are separated by $2.76\si{\angstrom}$ in air. Use equation for dipole-dipole interactions, i.e. $U = \frac{2\mu_A mu_B}{4\pi \varepsilon_0r^3}$ to calculate the dipole interaction. The dipole moment of water is 1.82 D.}

\begin{solution}
	$$U = \frac{2\left(1.823.336\times10^{-30}\right)^2}{4\pi\varepsilon_0\left(2.76\times10^{-10}\right)^2} = 3.15\times10^{-20}$$
\end{solution}

\question{Calculate the induced dipole moment of I$_2$ due to a Na$^+$ ion that is $5.0\si{\angstrom}$ away from the center of the I$_2$ molecule. The polarizability of I$_2$ is $12.5\times10^{-30}\unit{m^3}$.}

\begin{solution}
	$$\frac{\alpha q}{r^2} = \frac{12.5\times10^{-30}\times1.60\times10^{-19}}{\left(5.0\times10^{-10}\right)^2} = 8.01\times10^{-30}\si{C.m} = 2.4\si{D}$$
\end{solution}

\question{Diethyl ether (C$_2$H$_5$OC$_2$H$_5$) has a boiling point of 34.5$^\circ$C, whereas 1-butanol (C$_4$H$_9$OH) boils at 117$^\circ$C. These two compounds have the same type and  number of atoms.  Explain the difference in their boiling points.}

\begin{solution}
	Diethyl ether is nonpolar, while 1-butanol is polar. The hydrogen bonds in 1-butanol raises its boiling point.
\end{solution}

\question{If water were a linear molecule}

\begin{parts}
	\part{would it still be a polar}
	\part{would the water molecules still be available to form hydrogen bonds with one another?}
\end{parts}

\begin{solution}
	It would not be polar. However, there would still be hydrogen bonds, as the hydrogen atom and the lone pair electrons of oxygen remain.
\end{solution}

\question{Explain why ammonia is soluble in water but nitrogen trichloride is not?}

\begin{solution}
	Ammonia can form hydrogen bonds with water, hence it dissolves, as nitrogen bonded with hydrogen has lone pair electrons. However, nitrogen is not bonded with hydrogen in nitrogen trichloride, so it does not form hydrogen bonds with water.
\end{solution}

\question{Acetic acid is miscible with water, but it also dissolves in non polar solvents such as benzene or carbon tetrachloride. Explain.}

\begin{solution}
	Acetic acid can form hydrogen bonds with water (-COOH), yet it has a large hydrocarbon content, so it is miscible with water. In non polar solvents, it forms a dimer.
\end{solution}

\end{questions}
\end{document}
