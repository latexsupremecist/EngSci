\documentclass[12pt]{article}
\usepackage{../../template}
\author{niceguy}
\title{Lecture 9}
\begin{document}
\maketitle

\section{Electric Scalar Potential}

For a point charge which is not at the origin, we can generalise as
$$V = \frac{Q}{4\pi\varepsilon_0|\vec{R}-\vec{R}'|}$$
For a collection of point charges, we have
$$V = \sum_{i=1}^n \frac{Q_i}{4\pi\varepsilon_0|\vec{R}-\vec{R}'_i|}$$
For a continuous charge distribution,
$$V = \int\frac{dQ'}{4\pi\varepsilon_0|\vec{R}-\vec{R}'|}$$

\begin{ex}
	Determine the electric potential at any point on the axis of a uniformly charged disk of radius $a$.
	\begin{align*}
		V &= \int_0^{2\pi}\int_0^a \frac{\rho_srdrd\phi}{4\pi\varepsilon_0\sqrt{r^2+z^2}} \\
		  &= \frac{\rho_s}{2\varepsilon_0} \int_0^a \frac{r}{\sqrt{r^2+z^2}}dr \\
		  &= \frac{\rho_s}{2\varepsilon_0}\left(\sqrt{a^2+z^2}-|z|\right)
	\end{align*}
\end{ex}

\begin{ex}
	It is known that for a specific charge distribution it has an electric field given by
	$$\vec{E} = \begin{cases} 1\times10^{-5}r\hat{a}_r & r\leq 1\unit{cm} \\ \frac{1\times10^{-9}}{r}\hat{a}_r & r>1\unit{cm}\end{cases}$$
	Determine the volume charge density that creates this field. \\
	For the first case,
	\begin{align*}
		\vec{\nabla}\cdot\vec{E} &= \frac{\rho}{\varepsilon_0} \\
		\frac{d}{dr}\left(rE_r\right) &= \frac{\rho r}{\varepsilon_0} \\
		2\times10^{-5}r &= \frac{\rho r}{\varepsilon_0} \\
		\rho &= 2\times10^{-5}\varepsilon_0
	\end{align*}
	For the second case,
	\begin{align*}
		\frac{d}{dr}\left(rE_r\right) &= \frac{\rho r}{\varepsilon_0} \\
		0 &= \frac{\rho r}{\varepsilon_0} \\
		\rho &= 0
	\end{align*}
\end{ex}
\end{document}
