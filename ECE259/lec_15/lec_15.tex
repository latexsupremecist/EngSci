\documentclass[12pt]{article}
\usepackage{../../template}
\author{niceguy}
\title{Lecture 15}
\begin{document}
\maketitle

\section{Boundary Conditions for Electric Fields}

At a boundary, from Faraday's Law,
$$\oint_C \vec{E} \cdot d\vec{l} = 0$$
When we take the limit as the total normal distance goes to 0, for a rectangular box containing the boundary, we see the tangential components of $E$ must be equal across the boundary. Then from Gauss' Law,
$$\oiint_S \vec{D}\cdot d\vec{S} = \rho_s$$
Taking the limit as the normal distance goes to 0,
$$D_{\text{out}} - D_{\text{in}} = \rho_s$$
or in general
$$(\vec{D}_{\text{out}} - \vec{D}_{\text{in}})\cdot\hat{n} = \rho_s$$

\end{document}
