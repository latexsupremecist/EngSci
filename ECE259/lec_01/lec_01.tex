\documentclass[12pt]{article}
\usepackage{../../template}
\author{niceguy}
\title{Lecture 1}
\begin{document}
\maketitle

\section{Maxwell's Equations}

\begin{equation} \label{faraday}
	\vec{\nabla} \times \vec{E} = -\mu \frac{\partial \vec{H}}{\partial t}
\end{equation}

\begin{equation} \label{ampere}
	\vec{\nabla} \times \vec{H} = \vec{J} + \varepsilon \frac{\partial \vec{E}}{\partial t}
\end{equation}

\begin{equation} \label{gauss_electric}
	\vec{\nabla} \cdot \vec{E} = \frac{\rho_\nu}{\varepsilon}
\end{equation}

\begin{equation} \label{gauss_magnetic}
	\vec{\nabla} \cdot \vec{H} = 0
\end{equation}

Which are called Faraday's Law (\ref{faraday}), Ampere's Law (\ref{ampere}), Gauss's Law (Electric) (\ref{gauss_electric}) and Gauss's Law (Magnetic) (\ref{gauss_magnetic}).

\section{Coulomb's Law}

Charles de Coulomb discovered that a charge in the presence of another charge experiences a force, which exhibits the following properties

\begin{enumerate}
	\item Force is proportional to product of charges
	\item Force is inversely proportional to distance (or radius) squared
	\item Force acts in the direction along the line joining the two charges
	\item Like charges repulse, opposite charges attract
\end{enumerate}

\begin{align*}
	|\vec{F}_e| &\propto \frac{Q_1Q_2}{R^2} \\
		    &= k\frac{Q_1Q_2}{R^2} \\
		    &= \frac{Q_1Q_2}{4\pi\varepsilon_0R^2}
\end{align*}

\end{document}
