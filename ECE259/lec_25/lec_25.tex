\documentclass[12pt]{article}
\usepackage{../../template}
\author{niceguy}
\title{Lecture 25}
\begin{document}
\maketitle

\section{Amp\`ere's Law}
\begin{ex}
	An infinitely long metallic strip of negligible thickness and width $2a$ carries a total current of $I$. The strip lies in the $xy$-plane, is centered about the $z$-axis, and is infinitely long in the $x$-direction. The current is uniformly distributed over the width of the strip and flows in the $+x$-direction. Dtermine the magnetic field due to this current strip at a point $P(0,0,z)$. \\
	Using Biot-Savart law,
	$$d\vec{B} = \frac{\mu_0\vec{J}\times(\vec{R}-\vec{R}')dS}{4\pi|\vec{R}-\vec{R}'|^3}$$
	Plugging this into an integral,
	\begin{align*}
		\vec{B} &= \int_{-\infty}^\infty \int_{-a}^a \frac{\mu(\frac{I}{2a})\hat{a}_x\times(-x\hat{a}_x-y\hat{a}_y+z\hat{a}_z)}{(x^2+y^2+z^2)^{1.5}} dydx \\
			     &= \frac{\mu_0I}{8\pi a}\int_{-\infty}^\infty \int_{-a}^a \frac{-z\hat{a}_y-y\hat{a}_z}{(x^2+y^2+z^2)^{1.5}} dydx \\
			     &= \frac{\mu_0I}{8\pi a}\int_{-\infty}^\infty \int_{-a}^a \frac{-z\hat{a}_y}{(x^2+y^2+z^2)^{1.5}} dydx \\
			     &= -\frac{\mu_0I}{2\pi a}\tan^{-1}\left(\frac{a}{z}\right)\hat{a}_y
	\end{align*}
	Note that the $\hat{a}_z$ term is ignored. By the right hand rule, we know the resulting field is parallel to $\hat{a}_y$. Moreover, the $\hat{a}_z$ term is odd, so the integral yields a 0.
\end{ex}

The forms of \textbf{Amp\`ere's Law} are its differential form
$$\vec{\nabla}\times\vec{H} = \vec{J}$$
and its integral form
$$\ointctrclockwise_C \vec{H} \cdot d\vec{l} = \iint_S \vec{J} \cdot d\vec{S} = I$$

This means at any point in space, the magnetic field has a non zero curl if and only if a current density $\vec{J}$ is present.

\subsection{Integral Form}

In the integral form, we will use an \textbf{Amp\`erian Loop}. We choose a loop such that

\begin{itemize}
	\item $\vec{H}$ is either tangential or normal to the loop, so $\vec{H}\cdot d\vec{l}$ is $Hdl$ or $0$
	\item $\vec{H}$ has a constant value when $\vec{H}$ is tangential, so $\int Hdl = H\int dl = HL$
\end{itemize}

\begin{ex}
	Find the magnetic fields within each region of a coaxial cable. \\
	First observe that $\vec{H} = H_\phi\hat{a}_\phi$. Now for $0 < r < a$, the inner cabe,
	$$\ointctrclockwise_C \vec{H} \cdot d\vec{l} = H_\phi(2\pi r) = I_{\text{enc}} = J\pi r^2 = \left(\frac{I_0}\pi a^2\right)\pi r^2 \Rightarrow \vec{H} = \frac{I_0a^2r}{2\pi}$$
\end{ex}
\end{document}
