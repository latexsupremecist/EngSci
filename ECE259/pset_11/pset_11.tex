\documentclass[answers]{exam}
\usepackage{../../template}
\author{niceguy}
\title{Problem Set 11}
\begin{document}
\maketitle

\begin{questions}

\question{Consider a circular copper loop carrying a steady current, in air, and the following changes to the loop, one at the time: (a) change the wire material from copper to aluminum, (b) double the loop radius, (c) extend the loop so it becomes an ellipse with the major to minor axis ratio of four but the same circumference, (d) add a ferromagnetic core so that the loop encircles it, (e) double the current of the loop, and (f) reverse the direction of the loop current. Which of these changes would result in a change of the loop inductance?}

\begin{solution}
    C, changes (b)-(d) only. We know
    $$L = \frac{N\Phi}{I}$$
    This is independent of material, so (a) is false. $\vec B$ and hence $\Phi$ changes with (b) or (c), with all else constant, so both are true. For (c), $\mu$ changes, so $\vec B$ and hence $\Phi$ changes, so it is true. (e) would double $\Phi$ and $I$, where the changes cancel out. Finally, (f) does not change inductance, which is always positive.
\end{solution}

\question{If a piece of a ferromagnetic material of relative permeability $\mu_r$ is placed as a core of a wire loop, as indicated in Fig. Q7.3, the inductance of the loop, $L$, is related that, $L_0$, of the same loop with no core as follows}

\begin{solution}
    B, $L_0 < L < \mu_r L_0$. Note that $L$ is proportional to $\Phi$, which is proportional to $\vec B$, which is proportional to $\mu$. Depending on the region, i.e. air or the core, $\mu = \mu_0$ or $\mu = \mu_r \mu_0$. We can then take a weighted average and apply it to all $\vec B$ to get the same mathematical result for $L$. Since $\mu = \mu_0$ for $L_0$, the constant by which it is scaled to get $L$ is between $1$ and $\mu_r$.
\end{solution}

\question{A coil with $N$ turns of wire is wound uniformly and densely about a thin toroidal core made from a linear ferromagnetic material of relative permeability $\mu_r$. Consider the magnetic flux density, $\vec B$, inside the core and inductance, $L$, of the coil. If the diameter of the wire in the coil is halved and $N$ is doubled, while the current $I$ in the coil is kept the same, we have that}

\begin{solution}
    D, $\vec B$ doubles and $L$ quadruples. The diameter has no effect, but doubling $N$ would double $\vec B$, hence $\Phi$. Then the numerator of $L$ quadruples, with the denominator kept constant.
\end{solution}

\question{Two conducting wire contours, $C_1$ and $C_2$, in air carry slowly time-varying currents of intensities $i_1(t)$ and $i_2(t)$, respectively, as shown in Fig. Q7.4. The mutual inductance $L_{21}$ between the contours will not change if}

\begin{solution}
    D, both contour remain the same. This is because
    $$L_{12} = \frac{N_2}{I_1} \int \vec B_1 \cdot d\vec S_2$$
    The fraction is unchanged. If only $C_2$ changes, the domain of the integral changes, so $L_{12}$ need not stay constant. The same happens if only $C_1$ changes. Although it is possible, there is no guarantee mutual inductance is kept constant when both change.
\end{solution}

\question{The mutual inductance $L_{12}$ of the two contours in Fig. Q7.4 will change if}

\begin{solution}
    E, none of the above cases. If $I$ is scaled by a nonzero constant, this is reflected only in $\vec B_1$ and $I_1$ in the equation, where the constant is cancelled out. Hence no scaling of current would change mutual inductance.
\end{solution}

\question{If in Fig. Q.7.4 the orientation of the contour $C_1$ is reversed and that of $C_2$ remains the same, which of the mutual inductances $L_{12}$ and $L_{21}$ of the contours will change?}

\begin{solution}
    C, both inductances. Consider $L_{21}$, where the direction of $d\vec S_1$ is reversed. This causes $L_{21}$ to become its negative. Noting that $L_{12}=L_{21}$, we know both inductances change. Alternatively, flipping the contour would flip the direction of current, flipping the sign of $\vec B_1$ and hence $L_{12}$.
\end{solution}

\question{Out of the four mutual positions of two circular wire loops shown in Fig. Q7.5, the magnitude of the mutual inductance between the loops is largest in}

\begin{solution}
    B. $N$ and $I$ are constant, so only $\Phi$ matters. Note the direction of $\vec B$. Case (c) is out as it is perpendicular to the normal of the surface. Case (b) is out, as its $\vec B$ is not parallel to the normal, and is lesser in magnitude than in case (a), all else being equal. Finally, case (d) is also out, as it has a smaller $\vec B$. (It is inversely proportional to radius squared, which is minimised in case (a).)
\end{solution}

\question{Fig. Q7.6 shows two coils wound on a cardboard one. The mutual inductance $L_{12}$ of the coil is}

\begin{solution}
    A, positive. Without loss of generality, assume current goes clockwise, i.e. the positive voltage is on the right. Then $\vec B$ goes counterclockwise on the cardboard coil, cause by both coils. In both cases, $\vec B$ goes in the same direction as the normal of the surface, according to the right hand rule. Hence mutual inductance is positive.
\end{solution}

\question{Repeat the previous question but for two coils shown in Fig. Q7.7.}

\begin{solution}
    B, negative. Without loss of generality, we let current flow downwards. Then in this case, $\vec B$ points upwards, yet the normal of the surface points downwards for both coils. Hence mutual inductance is negative.
\end{solution}

\question{Two linear inductors of inductances $L$ and $2L$, respectively, have the same magnetic flux, $\Phi$. The magnetic energy stored in the inductor with twice as large inductance is}

\begin{solution}
    C, half of that stored in the other inductor. For convenience, since this holds for all cases, this holds for $N=1$. Then current in the inductor with the smaller inductance is twice as large. Now consider
    $$L = \frac{2W_m}{I^2}$$
    Since current is squared, $W_m$ must be doubled for the inductor with the smaller inductance to maintain the ratio of inductance.
\end{solution}

\question{Two linear inductors contain the same amounts of magnetic energy. If the magnetic field intensity ($\vec H$) at every point in the first inductor becomes twice larger, while the magnetic flux density ($\vec B$) at every point in the second inductor is halved, the energy stored in the first inductor in the new steady state is}

\begin{solution}
    D, 16 times. Note that $W_m$ is the integral of the dot product of both, and that there is a linear relationship between $\vec B$ and $\vec H$. Then the energy in the first inductor quadruples, as both are doubled. Similarly, the energy in the second inductor becomes 4 times smaller.
\end{solution}

\question{In two equally sized pieces of different ferromagnetic materials, a uniform magnetic field is first established, at the same intensity ($H_m$), and then reduced to zero ($H=0$), during which process the operating point describes the respective paths shown in Fig.Q7.10. The net magnetic energy spent in the magnetization-demagnetization of the piece in case(a) is}

\begin{solution}
    C, a half. $W_m$ is proportional to $BH$, and $B$ is half as great at the end in case (a), so net energy used is also half.
\end{solution}

\question{Refer to Example 6-16. Determine the inductance per unit length of the air coaxial tranmission line assuming that its outer conductor is not very thin but is of a thickness $d$.}

\begin{solution}
    In the outer conductor,
    $$I_\text{enc} = I \times \frac{\pi(b+d)^2 - \pi r^2}{\pi(b+d)^2 - \pi b^2} = I \times \frac{(b+d)^2-r^2}{(b+d)^2-b^2}$$
    Hence
    $$\vec B_3 = \frac{\mu_0I}{2\pi r} \times \frac{(b+d)^2-r^2}{(b+d)^2-b^2} \hat a_\phi$$
    Then the magnetic energy per length stored in the outer conductor is
    $$W_m = \frac{1}{2} \int_0^{2\pi} \int_b^{b+d} \frac{|\vec B_3|^2rdrd\phi dz}{\mu_0} = \frac{\mu_0I^2}{4\pi} \left(\frac{(b+d)^4}{((b+d)^2-b^2)^2} \ln \left(1 + \frac{d}{b}\right) + \frac{b^2-3(b+d)^2}{4((b+d)^2-b^2)}\right)$$
    Adding this to inductance found in the example, we have
    $$L = \frac{\mu_0}{2\pi}\left(\frac{1}{4} + \ln \frac{b}{a} + \frac{(b+d)^4}{((b+d)^2-b^2)^2}\ln\left(1+ \frac{d}{b}\right) + \frac{b^2-3(b+d)^2}{4((b+d)^2-b^2)}\right)$$
\end{solution}

\question{calculate the mutual inductance per unit length between two parallel two-wire transmission lines $A-A'$ and $B-B'$ separated by a distance $D$. Assume the wire radius to be much smaller than $D$ and the wire spacing $d$.}

\begin{solution}
    Note: this is a weird integral I don't quite get even after literally copying the answer key. \\
    To find inductance, we use
    $$L = \frac{N\Phi}{I}$$
    Where
    $$\Phi = \int \vec B \cdot d\vec S$$
    The surface we pick is the surface in the $\hat a_\phi$ direction, from the circle centred at $A$ touching $B$ to another concentric circle touching $B'$, i.e. from $r=D$ to $r=\sqrt{D^2+d^2}$. This is so that the surface "covers" the entirety of $B$ and $B'$ in the point of view of $A$. The same process is also done for $A'$, but the result is the same by symmetry. Moving on to the maths, note that for a long wire,
    $$\vec B = \frac{\mu I}{2\pi r} \hat a_\phi$$
    The surface as defined has area
    $$dS = Ldr = dr$$
    where $L=1$ is taken to be the unit length (in/out of page). Now
    $$\Phi = \int_D^{\sqrt{D^2+d^2}} \frac{\mu I}{2\pi r}\hat a_\phi \cdot dr\hat a_\phi = \frac{\mu I}{2\pi} \ln\frac{\sqrt{D^2+d^2}}{D}$$
    Then inductance becomes
    $$L = \frac{\mu}{\pi} \ln\frac{\sqrt{D^2+d^2}}{D}$$
\end{solution}

\question{Determine the mutual inductance between a very long, straight wire and a conducting circular loop.}

\begin{solution}
    Consider the flux of the long straight wire,
    $$B = \frac{\mu_0 I}{2\pi r}$$
    Then the flux is given by
    \begin{align*}
        \Phi &= \int BdS \\
             &= \int_{-b}^b \int_{-\sqrt{b^2-x^2}}^{\sqrt{b^2-x^2}} \frac{\mu I}{2\pi(x+d)} dydx \\
             &= \int_{-b}^b \frac{\mu I\sqrt{b^2-x^2}}{\pi(x+d)} dx \\
    \end{align*}
    The integral is analytical, but quite involved, according to WolframAlpha. According to the solution manual, which gives an integral that is most definitely wrong, we get
    $$\Phi = \mu I(d-\sqrt{d^2-b^2})$$
    Then inductance is
    $$L = \mu(d-\sqrt{d^2-b^2})$$
\end{solution}

\question{Find the mutual inductance between two coplanar rectangular loops with parallel sides. Assume that $h_1 >> h_2 (h_2 > w > d)$.}

\begin{solution}
    We ignore the shorter sides because they are shorter. Since $h_1 >> h_2$, we can ignore fringing effects, and assume
    $$B = \frac{\mu_0 I}{2\pi r}$$
    Then the flux according to the contributions of both long wires to the smaller rectangular loop is
    \begin{align*}
        \Phi &= \int \vec B \cdot d\vec S \\
             &= \int_0^{w_2} \frac{\mu_0 I}{2\pi} \left(\frac{1}{d+x} - \frac{1}{w_1+d+x}\right) h_2 dx \\
             &= \frac{\mu_0 Ih_2}{2\pi} \left(\ln\frac{d+w_2}{d} - \ln\frac{w_1+d+w_2}{w_1+d}\right)
    \end{align*}
    And
    $$L = \frac{\mu_0 h_2}{2\pi} \left(\ln\frac{d+w_2}{d} - \ln\frac{w_1+d+w_2}{w_1+d}\right)$$
\end{solution}

\question{Calculate the force per unit length on each of three equidistant, infinitely long parallel wires $0.15\unit{m}$ apart, each carrying a current of $25\unit{A}$ in the same direction. Specify the direction of the force.}

\begin{solution}
    Between each pair of wires,
    $$B = \frac{\mu I}{2\pi r}$$
    The contributions to any wire by the other two wires have an angle of $\frac{\pi}{3}$, so net $\vec B$ bisects both contributions, forming an angle of $\frac{\pi}{6}$ with each. Then total contribution becomes
    $$B = 2\cos\frac{\pi}{6} \frac{\mu I}{2\pi r} = \frac{\sqrt{3}\mu I}{2\pi r}$$
    Since
    $$F = I\vec l \times \vec B$$
    it points the wires towards the centre of the triangle. Its magnitude is
    $$F = IB = \frac{\sqrt{3}\mu I^2}{2\pi r} = 1.44\times10^{-3}$$
    per length.
\end{solution}

\question{Determine the force per unit length between two parallel, long, thin conducting strips of equal width $w$. The strips are at a distance $d$ apart and carry currents $I_1$ and $I_2$ in opposite directions.}

\begin{solution}
    From a previous problem, the magnetic flux density at height $y$ is given by
    $$B = -\frac{\mu_0 I_1}{2\pi w} \left[\arctan\frac{y}{d} + \arctan\frac{w-y}{d}\right]$$
    Then the force can be integrated as
    \begin{align*}
        F &= \int_0^w B\times\frac{I_2}{w}dy \\
          &= \frac{\mu_0I_1I_2}{2\pi w^2} \int_0^w\left[\arctan\frac{y}{d} + \arctan\frac{w-y}{d}\right]dy \\
          &= \frac{\mu_0I_1I_2}{2\pi w^2} \left[2w\arctan\frac{w}{d} - d\ln\left(1+\frac{w^2}{d^2}\right)\right]
    \end{align*}
    Where $\vec F$ points in the $\hat a_x$ direction. Note that the integral uses the identity
    $$\int \arctan x = x\arctan x - \frac{1}{2}\ln(1+x^2) + C$$
    which can be derived using integration by parts.
\end{solution}

\question{Find the force on the circular loop that is exerted by the magnetic field due to an upward current $I_1$ in the long straight wire. The circular loop carries a current $I_2$ in the counterclockwise direction.}

\begin{solution}
    \begin{align*}
        \vec F &= I\vec l \times \vec B \\
               &= \int_0^{2\pi} I_2 d\phi \hat a_\phi \times -\frac{\mu I_1}{2\pi(d+b\cos\theta)} \hat a_z \\
               &= -\frac{\mu I_1I_2}{2\pi} \int_0^{2\pi} \frac{d\phi}{d+b\cos\theta} \hat a_r \\
               &= -\frac{\mu I_1I_2}{2\pi} \int_0^{2\pi} \frac{(\cos\theta\hat a_x + \sin\theta\hat a_y)d\phi}{d+b\cos\theta}
    \end{align*}
    By symmetry, $\vec F$ has no net $y$ component, so we can ignore that. The integral becomes
    \begin{align*}
        \vec F &= -\frac{\mu I_1I_2}{\pi} \int_0^\pi \frac{\cos\theta d\theta}{d + b\cos\theta} \hat a_x \\
               &= \mu I_1I_2 \left(\frac{1}{\sqrt{1-\frac{b^2}{d^2}}} - 1\right) \hat a_x
    \end{align*}
    where the solution to the definite integral comes from the same place Praxis TAs get their marks from.
\end{solution}

\question{For the question above, assuming that the circular loop is rotated about its horizontal axis by an angle $\alpha$, find the torque exerted on the circular loop.}

\begin{solution}
    \begin{align*}
        \vec T &= \int d\vec T \\
          &= \int d\vec M \times \vec B \\
          &= I_2 \int d\vec S \times \vec B \\
          &= -I_2 \sin\alpha \int Bds \hat a_x \\
          &= -\mu I_1I_2 \sin\alpha (d-\sqrt{d^2-b^2}) \hat a_x
    \end{align*}
    Where the integral is solved the same as before.
\end{solution}

\question{Determine the mutual inductance between a very long, straight wire and a conducting equilateral triangle loop, as shown in the figure below}

\begin{solution}
    Mutual inductance is defined as
    $$L_{12} = \frac{N_2}{I_1} \int_{S_2} \vec B_1 \cdot d\vec S_2$$
    The $B$-field generated by the line current is
    $$\vec B = -\frac{\mu I}{2\pi x} \hat a_z$$
    Flux through the triangle is
    \begin{align*}
        \Phi &= \int \vec B \cdot d\vec S \\
             &= \int_0^{\frac{\sqrt{3}}{2}b} \int_{-\frac{x}{\sqrt{3}}}^{\frac{x}{\sqrt{3}}} \frac{\mu I}{2\pi (x+d)} dydx \\
             &= \frac{\mu I}{\sqrt{3}\pi} \int_0^{\frac{\sqrt{3}}{2}b} \frac{x}{x+d} dx \\
             &= \frac{\mu I}{\sqrt{3}\pi} x - d\ln(x+d) \Big |_0^{\frac{\sqrt{3}}{2}b} \\
             &= \frac{\mu I}{\sqrt{3}\pi} \left(\frac{\sqrt{3}}{2}b - d\ln\left(1+\frac{\sqrt{3}b}{2d}\right)\right)
    \end{align*}
    Inductance is then
    $$\frac{\mu}{\sqrt{3}\pi} \left(\frac{\sqrt{3}}{2}b - d\ln\left(1+\frac{\sqrt{3}b}{2d}\right)\right)$$
\end{solution}

\question{The cross section of a long thin metal plate and a parallel wire is shown in the figure below. Equal and opposite current $I$ flow in the conductors. Find the force per unit length acting on both conductors.}

\begin{solution}
    Magnetic flux at an arbitrary point on the metal plate is
    \begin{align*}
        \vec B &= -\frac{\mu I}{2\pi r} \hat a_\phi \\
               &= \frac{\mu I}{2\pi\sqrt{D^2+y^2}} \left(\frac{y\hat a_x}{\sqrt{D^2+y^2}} + \frac{D\hat a_y}{\sqrt{D^2+y^2}}\right)
    \end{align*}
    Now force per unit length is
    \begin{align*}
        \vec F &= I\vec l \times \vec B \\
               &= \int_{-\frac{W}{2}}^{\frac{W}{2}} \frac{I}{W} dy \hat a_z \times \frac{\mu I}{2\pi\sqrt{D^2+y^2}} \left(\frac{y\hat a_x}{\sqrt{D^2+y^2}} + \frac{D\hat a_y}{\sqrt{D^2+y^2}}\right) \\
               &= \frac{\mu I^2}{2W\pi} \int_{-\frac{W}{2}}^{\frac{W}{2}} \frac{ydy\hat a_y}{D^2+y^2} - \frac{Ddy\hat a_x}{D^2+y^2} \\
               &= \frac{\mu I^2}{2W\pi} -\arctan\frac{y}{D} |_{-\frac{W}{2}}^{\frac{W}{2}} \hat a_x \\
               &= \frac{\mu I^2}{2W\pi} (-2) \arctan\frac{W}{2D} \hat a_x \\
               &= -\frac{\mu I^2}{W\pi} \arctan\frac{W}{2D} \hat a_x
    \end{align*}
    The forces must balance, so force per unit length exerted by the metal plate on the wire is
    $$\vec F = \frac{\mu I^2}{W\pi} \arctan\frac{W}{2D} \hat a_x$$
\end{solution}

\question{The bar AA’ in the figure below serves as a conducting path (such as the blade of a circuit breaker) for the current $I$ in two very long (semi-infinite) parallel lines. The lines have a radius $b$ and are spaced at a distance $d$ apart. Find the direction and the magnitude of the magnetic force on the bar.}

\begin{solution}
    Flux at the bar due to the bottom line is
    \begin{align*}
        \vec B &= \frac{\mu I}{4\pi} \int_0^\infty \frac{-dx\hat a_x \times (y\hat a_y - x\hat a_x)}{(x^2+y^2)^{\frac{3}{2}}} \\
               &= -\frac{\mu I}{4\pi} \int_0^\infty \frac{ydx\hat a_z}{(x^2+y^2)^{\frac{3}{2}}} \\
               &= -\frac{\mu I}{4\pi} \frac{1}{y} \hat a_z \\
               &= -\frac{\mu I}{4\pi y} \hat a_z
    \end{align*}
    By symmetry, flux due to the top line is
    $$\vec B = -\frac{\mu I}{4\pi (d - y)} \hat a_z$$
    Then force acting on the bar is
    \begin{align*}
        \vec F &= I\vec l \times \vec B \\
               &= \int_0^d Idy \hat a_y \times \left(-\frac{\mu I}{4\pi}\right)\left(\frac{1}{y} + \frac{1}{d-y}\right)\hat a_z \\
               &= -\frac{\mu I^2}{4\pi} \int_b^(d-b) \left(\frac{1}{y} + \frac{1}{d-y}\right)dy \hat a_x \\
               &= -\frac{\mu I^2}{4\pi} \left(\ln\frac{d-b}{b} - \ln\frac{b}{d-b}\right) \hat a_x \\
               &= -\frac{\mu I^2}{2\pi} \ln\frac{d-b}{b} \hat a_x
    \end{align*}
\end{solution}

\question{A d-c current $I = 10\unit{A}$ flows in a triangular loop in the xy-plane as in the figure below. Assuming a uniform magnetic flux density $\vec B = \hat a_y 0.5\unit{T}$ in the region, find the forces and torque on the loop. All dimensions are in cm.}

\begin{solution}
    Segment 1:
    \begin{align*}
        \vec F &= I\vec l \times \vec B \\
               &= 10 \times 0.2 \hat a_x \times 0.5 \hat a_y \\
               &= \hat a_z
    \end{align*}
    Segment 2:
    \begin{align*}
        \vec F &= I\vec l \times \vec B \\
               &= \left(-\frac{10}{\sqrt{5}}\hat a_x + \frac{20}{\sqrt{5}}\hat a_y\right) \times 0.5 \hat a_y \times \frac{\sqrt{5}}{10} \\
               &= -0.5 \hat a_z
    \end{align*}
    Segment 3:
    \begin{align*}
        \vec F &= I\vec l \times \vec B \\
               &= \left(-\frac{10}{\sqrt{5}}\hat a_x - \frac{20}{\sqrt{5}} \hat a_y\right) \times 0.5 \hat a_y \times \frac{\sqrt{5}}{10} \\
               &= -0.5 \hat a_z
    \end{align*}

    Total force on the loop is then 0. Torque is
    \begin{align*}
        \vec T &= \vec m \times \vec B \\
               &= NI\vec A \times 0.5\hat a_y \\
               &= 10 \times \frac{1}{2} \times 0.2 \times 0.2 \hat a_z \times 0.5 \hat a_y \\
               &= -0.1\hat a_x
    \end{align*}
\end{solution}

\end{questions}
\end{document}
