\documentclass[12pt]{article}
\usepackage{../../template}
\author{niceguy}
\title{Lecture 18}
\begin{document}
\maketitle

\section{Capacitors}

\begin{ex}

	Consider an inner and outer cylindrical conductors. $Q = 1\unit{nC}, a = 1\unit{mm}, b = 3\unit{mm}, c = 5.5\unit{mm}, \varepsilon_{r1} = 2, \varepsilon_{r2} = 4, L = 5\unit{cm}$. Where $a,b,c$ are the radii in ascending order. Find the total stored electric potential energy and the capacitance. \\
	We can use the equation
	$$W = \frac{1}{2} \iint_S \rho_s vds$$
	or
	$$W = \frac{1}{2} \iiint_V \vec{D} \cdot \vec{E} dv$$
	From Gauss' Law,
	$$\vec{D} = \begin{cases} \frac{Q}{2\pi rL}\hat{a}_r & a < r < c \\ 0 & \text{else} \end{cases}$$
	Then
	$$\vec{E} = \begin{cases} \frac{Q}{2\pi \varepsilon_0\varepsilon_{r1} rL}\hat{a}_r & a < r < b \\ \frac{Q}{2\pi \varepsilon_0\varepsilon_{r2} rL}\hat{a}_r & b < r < c \\ 0 & \text{else} \end{cases}$$
	The second equation then gives
	\begin{align*}
		W &= \frac{1}{2} \int_0^L \int_0^{2\pi} \int_a^b \frac{Q^2}{4\pi^2 \varepsilon_0\varepsilon_{r1} r^2L} rdrd\phi dz + \frac{1}{2} \int_0^L \int_0^{2\pi} \int_b^c \frac{Q^2}{4\pi^2 \varepsilon_0\varepsilon_{r2} r^2L} rdrd\phi dz \\
		  &= \frac{Q^2}{4\pi\varepsilon_0}\left(\frac{\ln b - \ln a}{\varepsilon_{r1}} - \frac{\ln c - \ln b}{\varepsilon_{r2}}\right) \\
		  &= 0.126\unit{\mu J}
	\end{align*}
	And
	$$c = \frac{1}{2} \frac{Q^2}{W} = 3.87\unit{pf}$$
\end{ex}

\section{Some Maff}

Consider the differential form of Gauss' Law:
$$\vec{\nabla} \cdot (\varepsilon\vec{E}) = \rho_v$$
$$\vec{E} = -\vec{\nabla} V$$
Poisson's equation is then
$$\vec{\nabla}\cdot(\varepsilon\vec{\nabla}V) = -\rho_v$$
If $\rho_v = 0$, we get Laplace's equation
$$\vec{\nabla}\cdot(\varepsilon\vec{\nabla}V) = 0$$

If the material is homogeneous, i.e. $\sigma$ and $\varepsilon_r$ are independent of spatial coordinates, this simplifies to
$$\vec{\nabla}^2V = -\frac{\rho_v}{\varepsilon}$$
and
$$\vec{\nabla}^2V = 0$$
\end{document}
