\documentclass[12pt]{article}
\usepackage{../../template}
\author{niceguy}
\title{Lecture 2}
\begin{document}
\maketitle

\section{Electric Field Intensity}

Two objects are required for an electric force

\begin{itemize}
	\item Source charge $Q_1$
	\item Test charge $Q_2$
\end{itemize}

\begin{defn}
	The \emph{electric field intensity} is defined as
	$$\vec{E}_{12} = \lim_{Q_2\rightarrow 0} \frac{\vec{F}_{12}}{Q_2}$$
\end{defn}

Note that the force is experimentally determined to be

$$\vec{F}_{12} = k\frac{Q_1Q_2}{R^2}\hat{a}_{12}$$

where

$$k = \frac{1}{4\pi\varepsilon_0}$$

The relations below are then easily derived

$$\vec{E}_{12} = \frac{Q_1(\vec{R} - \vec{R'})}{4\pi\varepsilon_0|\vec{R} - \vec{R'}|^3} $$

$$\vec{F}_{12} = Q_2\vec{E}_{12}$$

The electric field has 4 key properties as mentioned in the previous lecture

\begin{enumerate}
	\item $\vec{E}$ points away from positive charges
	\item $\vec{E}$ points towards negative charges
	\item $\vec{E}$ points along the line connecting the source to the measurement point
	\item $\vec{E}$ is linear, hence superposition applies
\end{enumerate}

\section{Cylindrical Coordinates}

Note that

$$\vec{R} = r\hat{a}_r + z\hat{a}_z$$

\section{Spherical Coordinates}

Note that

$$\vec{R} = r\hat{a}_r$$

We use the notation in physics, i.e. $\hat{a}_{\phi}$ lies on the $xy$ plane, and at $\theta=0$, $\hat{a}_r = \hat{a}_z$.

\end{document}
