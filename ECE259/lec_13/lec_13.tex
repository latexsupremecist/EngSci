\documentclass[12pt]{article}
\usepackage{../../template}
\author{niceguy}
\title{Lecture 13}
\begin{document}
\maketitle

\section{Summary}
\begin{ex}
	A 3mm gap between two capacitor plates is partially filled with a dielectric of thickness 1mm. The dielectric has a relative permittivity of $\varepsilon_r = 2$, and the charge densities on the two metal pates are $\rho_s = \pm3 \mu$C/m$^2$. \\
	In terms of magnitude of electric field intensity, $1 = 3 > 2$. \\
	the polarization vector $\vec{P}$ is obviously 0 in free space. In the middle region with the dielectrics, we get
	$$\vec{P} = \varepsilon_0 \chi_e\vec{E} = \varepsilon_0(2-1)\frac{\rho_s}{2\varepsilon_0} = \frac{\rho_s}{2}$$
\end{ex}

The relative permittivity $\varepsilon_r$ of a material describes how easily it is polarized, relating to the electric susceptibility $\chi_e = \varepsilon_r - 1$.

\section{Electric Flux Density}

We can write

$$\vec{D} = \varepsilon_r\varepsilon_0\vec{E} = \varepsilon_0\vec{E} + \vec{P}$$

Then, $\vec{D}$ does not change with dielectrics, i.e. it is material independent. However, $\vec{E}$ is more associated with the field, or force needed to move charges.

\end{document}
