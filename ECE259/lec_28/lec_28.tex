\documentclass[12pt]{article}
\usepackage{../../template}
\author{niceguy}
\title{Lecture 28}
\begin{document}
\maketitle

\section{Hysteresis}

"Soft" magnetic matierials have small residual $\vec B$ values, being easily magnetised and demagnetised. \\
"Hard" magnetic materials are the opposite, having high residual $\vec B$ values. They are difficult to demagnetise, and make good permanent magnets. However, there is a greater energy loss when applied field $\vec H$ varies with time.

\section{Boundary Conditions}

From
$$\oiint_S \vec B \cdot d\vec S = 0$$
Observe that the integral is equal to $\vec B_{n1} - \vec B_{n2}$, meaning the normal component of $\vec B$ is preserved when crossing a magnetic materia. \\
Now consider a closed loop of length $\Delta l$ along $\vec H$. Then
$$\oint_ctrclockwise \vec H \cdot d\vec l = H_{t2} \Delta l - H_{t1} \Delta l = J_S\Delta l \Rightarrow \hat n_2 \times (\vec{H_1} - \vec{H_2}) = \vec{J_S}$$

\end{document}
