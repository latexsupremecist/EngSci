\documentclass[answers]{exam}
\usepackage{../../template}
\author{niceguy}
\title{Problem Set 9}
\begin{document}
\maketitle

\begin{questions}

\question{Consider an arbitrary distribution of volume currents in a vacuum
and the magnetic flux density vector due to these currents at an arbitrary point in space. If the magnitude of the current density vector is doubled everywhere, the magnetic flux density vector consider}

\begin{solution}
    A, becomes twice large in magnitude and keeps the same direction. This is because from the Biot-Savart Law, $\vec B$ is proportional to $I$.
\end{solution}

\question{A rectangular wire loop of edge lengths $a$ and $b$ in air carries a steady current of intensity $I$ ($I > 0$), as shown in Fig. Q4.3. The magnetic flux density vector $\vec B$ at the point $M$ in the figure can be represented as}

\begin{solution}
    C, $\vec B = B_z\hat a_z, B_z > 0$.
\end{solution}

\question{Three identical solenoidal coils, wound uniformly and densely with $N$ turns of thin wire, are positioned in space as shown in Fig. Q4.4. The axes of coils lie in the same plane and the permeability everywhere is $\mu_o$. Let $I_1, I2$ and $I_3$ denote the intensities of time-invariant currents in the coils. Consider the following two cases: (a) $I_1 = I_2 = I_3 = I$ and (b) $I_1 = I, I_2 = I3 = 0$. If $I > 0$, the magnetic flux density at the center of the system (the point $P$) for case (a) is}

\begin{solution}
    C, smaller than the magnetic flux density at the same point for case (b). In case (a), $\vec B$ cancels out to 0 by symmetry, while $\vec B$ is nonzero in case (b).
\end{solution}

\question{A time-invariant current of intensity $I( > 0)$ is established in a cylindrical copper conductor. The conductor is situated in air. The circulation (line integral) of the magnetic flux density vector, $\vec B$, along a contour $C$ composed from two circular and two radial parts and positioned outside the conductor, as shown in Fig. Q4.5, is}

\begin{solution}
    E, zero. The small radial components of the contour do not matter, as they are perpendicular to $\vec B$, which vanishes. In the other cases, $\vec B$ is either parallel or anti parallel to $d\vec l$, so the integral becomes $\mu_0I-\mu_0I = 0$.
\end{solution}

\question{Consider two very long metallic conductors, one of a circular and the other of square cross section. Both conductors carry steady currents of the same density. If the same circular contour $C$ is positioned inside each of the conductor, as in Fig. Q4.6, the circulation of the magnetic flux density vector along $C$ in the circular conductor is}

\begin{solution}
    B, the same as. This is because the enclosed currents are the same.
\end{solution}

\question{Consider two identical cylindrical metallic conductors carrying steady currents of the same intensity and Amperian contours positioned around each of them. The first contour is circular, while the other one has a square shape, as shown in Fig. Q4.7. The circulation of the magnetic flux density vector along the circular contour is}

\begin{solution}
    B, the same as. Same reason as above.
\end{solution}

\question{ A contour composed of eight straight segments is positioned in air near a very long wire conductor with a steady current of intensity $I$(Fig. Q4.8). The line integral of the magnetic flux density vector due to this current along the part of the contour between points $M$ and $Q$, via $N$ and $P$, equals}

\begin{solution}
    B, $\frac{\mu_0I}{2}$. Consider mirroring the circuit along the vertical axis. Then the total path integral is $\mu_0I$. Then by symmetry, it is half of that.
\end{solution}

\question{The inner and outer conductors of a coaxial cable carry steady currents of the same intensity and opposite directions. The cable dielectric and conductors are nonmagnetic, and the surrounding medium is air. If the current intensity in the inner conductor is increased, while keeping the current in the outer conductor unchanged, the magnetic flux density at every point of the outer conductor}

\begin{solution}
    A, increases. The closed path integral of $\vec B$ is equal to the enclosed current. Since enclosed current increases, $\vec B$ increases.
\end{solution}

\question{For the coaxial cable from the previous question, assume that the current intensity in the outer conductor is decreased, while keeping the current in the inner conductor unchanged. As a result, the magnetic flux density at every point of the inner conductor (not considering the points at the conductor axis)}

\begin{solution}
    C, remains the same. Same logic as above, but enclosed current remains the same.
\end{questions}

\question{An insulated metallic strip folded as in Fig. Q4.9 carries a steady current of intensity $I$. The width of the strip is $a=20d$, where $d$ is the diameter of the cylindrical cavity formed by the strip. With this, the magnitude of the $\vec B$ field at the point $P$ inside the cavity (see the figure) is}

\begin{solution}
    No asterisk no do
\end{solution}

\question{Consider the folded strip conductor from the previous question, and assume that $a$ is made twice larger, while keeping $I$ and $d$ the same. The magnitude of the $\vec B$ field at the point $P$ in the modified structure is}

\begin{solution}
    no asterisk
\end{solution}

\question{Two very long and wide conducting strips are placed in air parallel and very close to each other. Steady currents of the same intensity, $I$, and opposite directions flow through the strips, as shown in Fig. Q4.10. If the current direction is reversed in one of the strips, the magnetic field between the strip, away from the strip edges}

\begin{solution}
    B, becomes noticeably weaker. Approximate the long and wide conducting strips as wires (physics moment).
\end{solution}

\question{Consider the field pattern (showing lines of a vector field a in a part of free space) in Fig. Q4.11(a) and that in Fig. Q4.11(b), and whether each of the fields is divergence-free ($\vec \nabla \cdot \vec a = 0$) and/or curl-free ($\vec \nabla \vec a = 0$). Which of the following statements is true?}

\begin{solution}
    A, Field in Fig. Q4.11(a) is divergence-free and field in Fig. Q4.11(b) is curl-free. THis is because in (a), $\vec a = f(y)\hat a_x$, and in (b), $\vec a = f(x)\hat a_x$.
\end{solution}

\question{In a certain region in free space, there is a uniform magnetic field, with flux density $\vec{B_0}$. The volume current density in that region}

\begin{solution}
    B, is zero. This is because $\oint_ctrclockwise \vec B \cdot d\vec l = \mu I$, but the first integral is always zero for a uniform field.
\end{solution}

\question{A sphere of radius $a$ is placed in free space near a very long, straight wire carrying a steady current of intensity $I$. The distance of the sphere center from the axis is $d$. The magnetic flux through the sphere surface depends on}

\begin{solution}
    E, non of the above parameters. This is because $\oiint \vec B \cdot d\vec S = 0$.
\end{solution}

\question{Consider an imaginary open conical surface in a uniform steady magnetic field of flux density $\vec B$ = 1 T. The height (length) of the cone is $h = 20$ cm and the radius of its opening is $a = 10$ cm. The vector $\vec B$ makes an angle $\alpha = 45$ with the cone axis as in Fig.Q4.12. If $h$ is doubled (without changing $a$, $B$, and $\alpha$), the magnetic flux through the conical surface (oriented downward)}

\begin{solution}
    C.
\end{solution}

\question{Consider a vertical cylinder in
a steady magnetic field in free space. If $\Phi_1$ denotes the magnetic flux through the lower basis of the cylinder and $\Phi_2$ that through the upper basis with both surfaces oriented in the same way (upwards), we have that}

\begin{solution}
\end{solution}

\question{If both curl and divergence of a vector $\vec a$ at a point are zero, then $\vec a$ must be zero at that point.}

\begin{solution}
    Obviously no. Consider $\vec a = k \hat a_x$.
\end{solution}

\question{If a vector $\vec a$ at a point is zero, then both the curl and divergence of $\vec a$ must be zero at that point.}

\begin{solution}
    Obviously no. Consider $\vec a = (x+y)\hat a_x$.
\end{solution}

\question{An infinintely long cylindrical bar magnet of radius $a$, is permanently magnetized with a uniform magnetization, and the magnetization vector, of magnitude $M$, is parallel to the bar axis. Magnitudes of the magnetization volume and surface current density vectors, $\vec{J_m}$ and $\vec{J_{ms}}$, over the volume and surface of the magnet, respectively, are}

\begin{solution}
\end{solution}
\end{document}
