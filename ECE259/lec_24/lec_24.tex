\documentclass[12pt]{article}
\usepackage{../../template}
\author{niceguy}
\title{Lecture 24}
\begin{document}
\maketitle

\section{DC Motors}
A commutaor is when the current is off for half of the rotation, so the motor spins in one direction. 

\begin{ex}[Mendochino Motors]
	There is a source of magnetic field (magnet), current source (battery), and a commutator that makes it spin.
\end{ex}

\section{Magnetic Vector Potential}

$$\vec{\nabla} \cdot \vec{B} = 0 \Leftrightarrow \oiint_S \vec{B} \cdot d\vec{S} = 0$$
From one of the Maxwell's equations, we see all magnetic fields have no divergence (solenoidal). Since the divergence of a curl is always 0, we can think of $\vec{B}$ as a curl, namely
$$\vec{B} = \vec{\nabla} \times \vec{A}$$

We can relate this with the current through
$$\vec{\nabla}^2\vec{A} = -\mu_0\vec{J}$$
Which gives
$$\vec{A} = \frac{\mu_0}{4\pi} \int \frac{Id\vec{l}}{|\vec{R}-\vec{R}'|}$$
or
$$\vec{A} = \frac{\mu_0}{4\pi}\iiint_V \frac{\vec{J}dV}{|\vec{R}-\vec{R}'|}$$
We define $\Phi_m$ as below, and
$$\Phi_m = \oint_C \vec{A} \cdot d\vec{l} = \iint_S \vec{B} \cdot d\vec{S}$$
by Stokes'.

\section{Biot-Savart Law}

$$d\vec{B} = \frac{\mu_0Id\vec{l}\times(\vec{R}-\vec{R}')}{4\pi|\vec{R}-\vec{R}'|^3}$$

\end{document}
