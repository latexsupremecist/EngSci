\documentclass[12pt]{article}
\usepackage{../../template}
\author{niceguy}
\title{Lecture 27}
\begin{document}
\maketitle

\section{Generalized Amp\`ere's Law}

$$\vec B = \mu \vec H = \mu_0\mu_r \vec H$$
and
$$\oint_C \vec H \cdot d\vec l = I_{\text{enc}} \Rightarrow \oint_C \vec B \cdot d\vec l = \mu_0\mu_r I_{\text{enc}}$$
Similar to Gauss' Law with $\vec E$ and $\vec D$.

\begin{ex}
	Consider a very long solenoid that consists of $n$ turns per meter filled with a magnetic material with relative permeability of $\mu_r$. Find the magnetic field intensity, $\vec H$, inside the solenoid. \\
	Let the closed path be a rectangle. The only nonzero component is the part of the path parallel to $\vec B$. Then
	$$Bw = \mu_0\mu_rI_0nw \Rightarrow \vec B = \mu_0\mu_r I_0n = \frac{\mu_0\mu_rI_0N}{L}\hat a_z$$
\end{ex}

\section{Ferromagnetism}

When a material is exposed to an applied field $\vec B$,

\begin{itemize}
	\item Materials with nonzero internal moments can align:
	\begin{itemize}
		\item Strong alignment (Ferromagnetic): Field is greatly enhanced
		\item Weak alignment (Paramagnetic): Filed is moderately enhanced
	\end{itemize}
	\item Materials with net internal moments of zero:
	\begin{itemize}
		\item Due to Lenz's Law, the applied field reduced the orbital moments slightly
		\item Diamagnetic: This causes a small net reduction of the $\vec B$ field within the material
	\end{itemize}
\end{itemize}


\end{document}
