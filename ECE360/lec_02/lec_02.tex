\documentclass[12pt]{article}
\usepackage{../../template}
\title{Lecture 2}
\author{niceguy}
\begin{document}
\maketitle

\section{Diodes}

To analyse diodes, consider 2 cases:

\begin{itemize}
    \item V\textsubscript{in} $>$ 0, where the diode is replaced with a short circuit in an equivalent circuit
    \item V\textsubscript{in} $<$ 0, where the diode is replaced with an open circuit
\end{itemize}

In AC currents, the above alternates every half-cycle. Remember to verify if the assumption of the current direction is correct!

\begin{ex}
\begin{figure}[!h]
\begin{center}
    \begin{circuitikz}\draw
  (-1,4.5) to[stroke diode] ++(1.5,0)
  to[short] ++(1,0)
  to[short] ++(0,-1)
  to[short] ++(-1,0)
  (-1,3.5) to[stroke diode] ++(1.5,0)
  (1.5,3.5) to[european resistor] ++(0,-2.5)
  (1.5,3.5) to[short] ++(2,0)
  (1.5,1) node[ground]{}
;\end{circuitikz}
\caption{2 inputs on the left; 1 output on the right}
\end{center}
\end{figure}

We can build a truth table

\begin{figure}
\begin{center}
    \begin{tabular}{c c c}
        \hline
        Input 1 & Input 2 & Output \\
        \hline
        0 & 0 & 0 \\
        \hline
        5 & 0 & 5 \\
        \hline
        0 & 5 & 5 \\
        \hline
        5 & 5 & 5 \\
        \hline
    \end{tabular}
\end{center}
\end{figure}
\end{ex}

\begin{ex}
\begin{figure}[!h]
\begin{center}\begin{circuitikz}\draw
  (3,2) to[short] ++(0,1)
  to[stroke diode] ++(-1.5,0)
  (4,2) to[short] ++(-1,0)
  to[stroke diode] ++(-1.5,0)
  (3,3) to[european resistor] ++(0,2.5)
;\end{circuitikz}\end{center}
\caption{2 inputs; 1 output; top voltage set at 5}
\end{figure}

\begin{figure}
\begin{center}
    \begin{tabular}{c c c}
        \hline
        Input 1 & Input 2 & Output \\
        \hline
        0 & 0 & 0 \\
        \hline
        5 & 0 & 5 \\
        \hline
        0 & 5 & 5 \\
        \hline
        5 & 5 & 5 \\
        \hline
    \end{tabular}
\end{center}
\end{figure}
\end{ex}

\begin{ex}
\begin{figure}[!h]
\begin{center}\begin{circuitikz}\draw
  (-0.5,3.5) node[ground]{}
  (-0.5,3.5) to[short] ++(1.5,0)
  (1,3.5) to[stroke diode] ++(0,-1.5)
  (1,2) to[european resistor] ++(0,-2)
  (1,2) to[short] ++(2.5,0)
  (3.5,3.5) to[stroke diode] ++(0,-1.5)
  (3.5,3.5) to[short] ++(1.5,0)
  (3.5,3.5) to[european resistor] ++(0,2)
;\end{circuitikz}\end{center}
\caption{Bottom at -10V; top at 10V, bottom resistor at 10k$\Omega$; top resistor at 5k$\Omega$; 1 output on right}
\end{figure}

If both are on, KCL fails when you consider the current through the bottom resistor. If both are off, net current goes up the bottom resistor without an escape.
\end{ex}


\end{document}
