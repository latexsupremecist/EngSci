\documentclass[12pt]{article}
\usepackage{../../template}
\title{Lecture 5}
\author{niceguy}
\begin{document}
\maketitle

\section{Small Signal Model}

Let's say there is some fluctuation in voltage. In a circuit diagram, we can consider the voltage source to be a DC source and AC source (with amplitude $\Delta V$) connected in series.

If we have a linear system, we can

\begin{itemize}
    \item Use superposition to solve the circuit
    \item Use linear algebra
\end{itemize}

For the above equation, if $\Delta V(t)$ is small, we can "scale" current up and get

$$i_D(t) \approx I_D\left(1 + \frac{\Delta V(t)}{V_T}\right)$$

Where we take the Taylor approximation of the solution

$$i_D(t) = I_D\exp\left(\frac{\Delta V(t)}{V_T}\right)$$

Where $I_D$ is the current with only the DC source.

\subsection{Resistor Model}

We can also use the resistor model, where we only turn one of the voltage sources on for superposition. If $\Delta V$ is small enough, then we can treaat the diode as a resistor, where resistance is given by voltage and current solutions from the DC current. This is technically a Taylor approximation, since

$$i_D(v) = i_D(V_D + \Delta V) \approx I_D(V_D) + \Delta V \frac{di}{dv} \Big |_{v = V_D}$$

\end{document}
