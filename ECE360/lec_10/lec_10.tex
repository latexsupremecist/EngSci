\documentclass[12pt]{article}
\usepackage{../../template}
\title{Lecture 10}
\author{niceguy}
\begin{document}
\maketitle

\section{Drift-Diffusion Current}

$$J_n = q\mu_nNE - qD_n\frac{dn}{dx}$$
and similar for $p$. This is the sum of drift and diffusion current, where $\frac{dn}{dx}$ is the gradient.

\section{PN Junctions}

When n type and p type materials love each other very much, a pn junction is born. We find a carrier depletion region formed by diffusion near the junction. However, an internal electric field is formed to preclude further diffusion, called a barrier voltage.

\section{Maxwell-Boltzmann Distribution}

The distribution is proportional to $\exp\left(\frac{E}{kT}\right) = \exp\left(\frac{qV}{kT}\right)$. The free electrons that overcome the potential barrier due to an applied voltage is then proportional to $\exp\left(\frac{qV}{kT}\right)-1$ as derived. A more detailed derivation can be found by consulting an EngSci ECE.
\end{document}
