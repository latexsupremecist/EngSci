\documentclass[12pt]{article}
\usepackage{../../template}
\title{Lecture 1}
\author{niceguy}
\begin{document}
\maketitle

\section{Rates}

\begin{defn}[Effective Rate]
    Say we have an effective rate $r_{\text{EFF}} = 10\%$ per year. This means in a year, the total interest is exactly 10\% of the principal.
\end{defn}

\begin{defn}[Rate]
    If we say the interest rate is 10\% but compounded semi-annually, the effective rate is 5\% per half-year.
\end{defn}

\begin{rem}
    The effective annual rate in the example above is $(1 + 5\%)^2 - 1 = 10.25\%$.
\end{rem}

\begin{ex}[Problem Set Q6]
    \begin{tabular}{|c|c|c|}
        \hline
        Given & Calculate & Answer \\
        \hline\hline
        Effective rate of 8\% & Monthly rate & $(1+8\%)^{1/12} - 1$ \\
        \hline
        Rate of 3.5\% compounded per trading day & Effective rate & $\left(1+\frac{3.5\%}{252}\right)^{252} - 1$ \\
        \hline
        Rate of 4\% compounded per quarter & Continuously compounding rate & $\ln\left(1+\frac{4\%}{4}\right)^4$
        \hline
    \end{tabular}
\end{ex}

\section{Converting to effective rates}

\begin{equation}
    i_{\text{eff}} = \left(1 + \frac{r_{n/m}}{n_m}\right)^{n_y} - 1
\end{equation}

Where $n_m$ is the number of compounding periods $n$ in time $m$, and $n_y$ is the same per year.

\begin{ex}
    Let the rate be 10\% per 3 years with quarterly compounding. Effective rate is then
    $$\left(1 + \frac{10\%}{12}\right)^4 - 1 = 2.375\%$$
    If we instead had 4\% per quarter compounded every 3 years, the effective rate is
    $$(1 + 4\% \times 12)^{1/3} - 1$$
\end{ex}

\begin{ex}
    Let's say our mortgage rate is 5\% compounded semi-annually. Your loan is 100k. What is the monthly interest? \\
    Interest rate per month is
    $$\left(1 + \frac{5\%}{2}\right)^{1/6} - 1 = 0.4124\%$$
    Interest is then
    $$100,000 \times 0.4124\% = \$412.4$$
    Do note that inflation exists. A simple way to get that out of the way is to account it in the interest rate (i.e. divide 1 + interest rate by 1 + inflation rate).
\end{ex}
    

\end{document}
