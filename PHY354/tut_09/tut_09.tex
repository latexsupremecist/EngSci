\documentclass[12pt]{article}
\usepackage{../../template}
\title{Tutorial 9}
\author{niceguy}
\begin{document}
\maketitle

Consider a cylinder $r = a$ rolling in a semicircular surface $y = \sqrt{R^2 - x^2}$. Let $\phi$ be the angle between the vertical and the line joining the centres of the circle (surface) and cylinder. Kinetic energy from translational motion can easily be found.

\begin{align*}
    T_{tr} &= \frac{1}{2}m(\dot r^2 + r^2\dot\phi^2) \\
           &= \frac{1}{2}m(R-a)^2\dot\phi^2
\end{align*}

For rotational motion, we need angular velocity and moments of inertia. The $z$ component suffices. Consider these two expressions which both give the velocity of the centre of mass.

\begin{align*}
    \vec\Omega \times \vec r &= (R-a)\dot\phi \\
    |\vec\Omega| &= \frac{R-a}{a}\dot\phi
\end{align*}

Rotational kinetic energy is then

\begin{align*}
    T_{rot} &= \frac{1}{2} I\Omega^2 \\
            &= \frac{1}{2} \times \frac{1}{2}m a^2 \times \frac{(R-a)^2}{a^2}\dot\phi^2 \\
            &= \frac{1}{4}m (R-a)^2\dot\phi^2
\end{align*}

and total kinetic energy is

$$T = \frac{3}{4}m(R-a)^2\dot\phi^2$$

\end{document}
