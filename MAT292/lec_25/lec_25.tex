\documentclass[12pt]{article}
\usepackage{../../template}
\author{niceguy}
\title{Lecture 25}
\begin{document}
\maketitle

\section{Laplace Transform}

Recall
$$F(s) = \int_0^\infty e^{-st}f(t)dt$$
Given the Laplace Transform of $f(t)$, how does it change when $f(t)$ itself is changed?
\begin{ex}
	$$f(t) \rightarrow e^{ct}f(t)$$
	Then
	\begin{align*}
		\mathcal{L}\{e^{ct}f(t)\}(s) &= \int_0^\infty e^{-st}e^{ct}f(t)dt \\
					  &= \int_0^\infty e^{-(s-c)t}f(t)dt \\
					  &= F(s-c)
	\end{align*}
	Which is defined when $s>a+c$.
\end{ex}

\begin{ex}
	$$f(t) \rightarrow f'(t)$$
	Then
	\begin{align*}
		\mathcal{L}\{f'(t)\}(s) &= \int_0^\infty e^{-st}f'(t)dt \\
					&= e^{-st}f(t) \Big |_0^\infty + s\int_0^\infty e^{-st}f(t)dt \\
					&= sF(s) - f(0)
	\end{align*}
	assuming $f$ and $f'$ are of exponential order.
\end{ex}

Doing this twice,
$$\mathcal{L}\{f''(t)\}(s) = s^2F(s) - sf(0) - f'(0)$$
assuming $f$, $f'$ and $f''$ are of exponential order, and are defined on $[0,\infty)$. Furthermore, using induction, one can show

$$\mathcal{L}\{f^{(n)}(s)\} = s^nF(s) - \sum_{i=0}^{n-1} s^if^{(n-1-i)}(0)$$

\begin{ex}
	$$f(t) \rightarrow t^nf(t)$$
	Then
	\begin{align*}
		\mathcal{L}\{f(t)\}(s) &= \int_0^\infty e^{-st}f(t)dt \\
		\frac{d^n}{ds^n}\mathcal{L}\{f(t)\}(s) &= \int_0^\infty (-1)^nt^ne^{-st}f(t)dt \\
		(-1)^n\mathcal{L}\{t^nf(t)\}(s) &= F^{(n)}(s) \\
		\mathcal{L}\{t^nf(t)\}(s) &= (-1)^nF^{(n)}(s)
	\end{align*}
\end{ex}

Putting $f(t) = 1$, we get
\begin{align*}
	\mathcal{L}\{t^n\} &= (-1)^n\frac{d^n}{ds^n} \frac{1}{s} \\
			   &= \frac{n!}{s^{n+1}}
\end{align*}

\section{Applications of the Laplace Transform}

\begin{ex}
	$$y'' + 2y' + 5y = e^{-t}, y(0) = 1, y'(0) = -3$$
	Applying the Laplace Transform on both sides,
	\begin{align*}
		s^2Y(s) - s + 3 + 2sY(s) - 2 + 5Y(s) &= \frac{1}{s+1} \\
		Y(s) &= \frac{s^2}{(s+1)(s^2+2s+5)}
	\end{align*}
\end{ex}
\end{document}
