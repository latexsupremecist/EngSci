\documentclass[12pt]{article}
\usepackage{../../template}
\author{niceguy}
\title{Lecture 29}
\begin{document}
\maketitle

\section{Discontinuous Forcing Functions}

\begin{ex}
	$$y''(t) + \pi^2y(t) = f(t), y'(0) = 0, y''(0) = 0$$
	where $f(t)$ is the square wave. \\
	Note that the window function of $f(t)$ is
	$$f_2(t) = \begin{cases} 1, & t \in[0,1) \\ 0, t \in [1,2) \end{cases}$$
	Applying the Laplace Transform,
	\begin{align*}
		\mathcal{L}\{f(t)\} &= \frac{\mathcal{L}\{f_2(t)\}}{1-e^{-2s}} \\
				    &= \frac{\int_0^2e^{-st}f_2(t)dt}{1-e^{-2s}} \\
				    &= \frac{\int_0^1e^{-st}dt}{1-e^{-2s}} \\
				    &= \frac{1}{s(1+e^{-s})} \\
		\mathcal{L}\{y''(t) + \pi^2y(t)\} &= s^2Y(s) - sy(0) - y'(0) + \pi^2Y(s) \\
						  &= s^2Y(s) + \pi^2Y(s) \\
		Y(s) &= \frac{1}{s(s^2+\pi^2)(1+e^{-s})}
	\end{align*}
	Define
	$$H(s) = \frac{1}{s(s^2+\pi^2)}$$
	Using partial fractions,
	$$H(s) = \frac{1}{\pi^2}\left(\frac{1}{s} - \frac{s}{s^2+\pi^2}\right)$$
	Using the lookup table,
	$$h(t) := \mathcal{L}^{-1}\{H(s)\} = \frac{1}{\pi^2}(1-\cos\pi t)$$
	We know that
	$$\frac{1}{1-x} = \sum_{n=0}^\infty x^n$$
	Therefore
	$$\frac{1}{1+e^{-s}} = \sum_{k=0}^\infty(-1)^ke^{-ks}$$
	Therefore
	\begin{align*}
		\mathcal{L}^{-1}\{Y(s)\} &= \mathcal{L}^{-1}\left\{\frac{H(s)}{1+e^{-s}}\right\} \\
					 &= \mathcal{L}^{-1}\left\{\sum_{k=0}^\infty(-1)^kH(s)e^{-Ks}\right\} \\
					 &= \sum_{k=0}^\infty(-1)^k\mathcal{L}^{-1}\left\{e^{-ks}H(s)\right\} \\
					 &= \sum_{k=0}^\infty(-1)^ku_k(t)h(t-K) \\
					 &= \frac{1}{\pi^2}\sum_{k=0}^\infty(-1)^k[1-\cos(\pi(t-k))]u_k(t)
	\end{align*}
	Plugging natural numbers,
	$$y(n) = \begin{cases} -\frac{n}{\pi^2}, & n \text{ is even} \\ \frac{n+1}{\pi^2}, & n \text{ is odd}\end{cases}$$
	This is an oscillating function whose amplitude increases linearly.
\end{ex}

\section{Impulse Function}

Consider

$$\delta_\epsilon(t) = \frac{u_0(t) - u_\epsilon(t)}{\epsilon} = \begin{cases} \frac{1}{\epsilon}, & t \in [0,\epsilon) \\ 0, & t \in (-\infty, 0) \cup [\epsilon, \infty) \end{cases}$$
We want to take the limit as $\epsilon \rightarrow 0$. Note that this is not a function.

\begin{ex}
	$$y'' + y = \alpha\delta_\epsilon(t), y(0) = 0, y'(0) = 0$$
	Using the Laplace Transform,
	\begin{align*}
		\mathcal{L}\{\alpha\delta_\epsilon\} &= \alpha \int_0^\epsilon e^{-st} \times \frac{1}{\epsilon}dt \\
						     &= \frac{\alpha}{\epsilon} \times \frac{1-e^{-\epsilon s}}{s} \\
		Y(s) &= \frac{\alpha}{\epsilon}\left(\frac{1-e^{-\epsilon s}}{(s^2+1)s}\right)
	\end{align*}
	Using the lookup table,
	$$y_\epsilon = \frac{\alpha}{\epsilon}[u_0(t)(1-\cos t) - u_\epsilon(t)(1-\cos(t-\epsilon)] = \begin{cases} 0, & t \in (-\infty, 0) \\ \frac{\alpha}{\epsilon}(1-\cos t), & t \in [0,\epsilon) \\ \frac{\alpha}{\epsilon}(\cos(t-\epsilon) - \cos t), & t \in [\epsilon, \infty) \end{cases}$$
	Taking the limit as $\epsilon \rightarrow 0$, the second and third cases go to 0.
\end{ex}

We then define the Direc Delta function $\delta_0(t)$.

\begin{defn}
	The Dirac Delta function is the "function" where
	$$\delta_0(t-t_0) = 0 \forall t \neq t_0$$
	and $\forall f$ continuous on the interval $[a,b]$ containing $t_0$,
	$$\int_a^b f(t)\delta_0(t-t_0)dt = f(t_0)$$
\end{defn}

The intuition behind the second property is that

\begin{align*}
	\lim_{\epsilon \rightarrow 0} \int_0^\infty f(t)\delta_\epsilon(t-t_0)dt &= \lim_{\epsilon \rightarrow 0} \int_0^\epsilon f(t)\delta_\epsilon(t-t_0)dt \\
										 &= \lim_{\epsilon \rightarrow 0} \int_0^\epsilon \frac{f(t)}{\epsilon} dt \\
										 &= \lim_{\epsilon \rightarrow 0} \int_0^\epsilon \frac{f(\epsilon)}{\epsilon} dt \\
										 &= \lim_{\epsilon \rightarrow 0} f(\epsilon) \\
										 &= f(0)
\end{align*}

\begin{ex}
	Find the Laplace Transforms of the Dirac Delta function.
	\begin{align*}
		\mathcal{L}\{\delta_0(t)\} &= \int_0^\infty e^{-st}\delta_0(t)dt \\
					   &= e^0 \\
					   &= 1 \\
		\mathcal{L}\{\delta_0(t-t_0)\} &= e^{-st_0}
	\end{align*}
\end{ex}

\begin{ex}
	Solve the IVP
	$$2y''(t) + y'(t) + 2y(t) = \delta_0(t-5), y(0) = 0, y'(0) = 0$$
	\begin{align*}
		\mathcal{L}\{\delta_0(t-5)\} &= e^{-5s} \\
		\mathcal{L}\{2y'' + y' + 2y\} &= 2s^2Y(s) + sY(s) + 2Y(s) \\
		Y(s) &= \frac{e^{-5s}}{2s+2+s+2} \\
		     &= \frac{e^{-5s}}{2}\times\frac{1}{\left(\frac{s+1}{4}\right)^2 + \frac{15}{16}}
	\end{align*}
\end{ex}
\end{document}
