\documentclass[12pt]{article}
\usepackage{../../template}
\author{niceguy}
\title{Lecture 16}
\begin{document}
\maketitle

\section{Improving Euler's Method}
Recall Euler's method can be thought as approximating an integral
$$\int_{t_n}^{t_{n+1}} f(t,y)dt$$
We can improve Euler's method by finding a better approximation for $f$ on the interval $t \in [t_n,t_{n+1}]$. Our original approximation is that $f$ is approximately $f(t_n,y(t_n))$. A better approximate is to take the average value of $f$, i.e.
$$f(t,u(t)) \approx \frac{f(t_n,y(t_n)) + f(t_{n+1}, y(t_{n+11}))}{2}$$
However, we do not have access to $y(t_{n+1})$. Instead, we will use the approximated value of $y(t_{n+1})$ using Euler's method, i.e.
$$f(t,y(t)) \approx \frac{f(t_n,y(t_n)) + f(t_{n+1}, y_n + (t_{n+1}-t_nf(t_n,y_n))}{2}$$
We then have
$$y_{n+1} = y_n + \frac{f(_n,y_n) + f(t_{n+1}, y_n + hf(t_n,y_n))}{2} \times h$$
where $h = t_{n+1} - t_n$. \\
However, there is a downside to this, as more calculations have to be performed for each successive approximation. \\
The Error comparisons are as follows
\begin{center}
\begin{tabular}{|c||c|c|c|}
	\hline
	Method & Local Truncation Error & Global Truncation Error & Function Evaluations \\
	\hline \hline
	Euler & $h^2$ & $h$ & 1 \\
	\hline
	Improved Euler & $h^3$ & $h^2$ & 2 \\
	\hline
\end{tabular}
\end{center}
However, we can still improve on this. We first start with the slope $f_1 = f(t_n,y(t_n))$. Extending this to $t_n + \frac{h}{2}$, we get the slope $f_2$. Now put $f_2$ at $t_n$, and extend it to get a second approximation for the slope at $t_n + \frac{h}{2}$, which is $f_3$. Finally, $f_3$ is placed at $t_n$ which is then extended to approximate the slope at $t_{n_1}$, which is $f_4$. We take a weighted average with $f_1, f_2, f_3, f_4$ being given weights of $1, 2, 2, 1$ respectively.
$$y_{n+1} = y_n + \frac{s_{n1} + 2s_{n2} + 2s_{n3} + s_{n4}}{6} \times h$$
where
\begin{align*}
	s_{n1} &= f(t_n,y_n) \\
s_{n2} &= f(t_n + \frac{h}{2}, y_n + \frac{1}{2}hs_{n1}) \\
s_{n3} &= f(t_n + \frac{h}{2}, y_n + \frac{1}{2}hs_{n2}) \\
s_{n4} &= f(t_n + h, y_n + hs_{n3})
\end{align*}
There are some shortcomings to using a constant step size. There might be certain bounds where a smaller step size (more oscillations) is desired, while there are other bounds where a larger step size is desired (approximately constant). Assuming we want the maximum truncation error to be $\epsilon > 0$. If the error is greater than that, we decrease the step size, and vice versa. We estimate the local truncation error by
$$e_{n+1} \approx |y_{n+1} - z_{n+1}|$$
where $z$ is a better approximation than $y$.
\end{document}
