\documentclass[12pt]{article}
\usepackage{../../template}
\author{niceguy}
\title{Lecture 10}
\begin{document}
\maketitle

\section{Continued from the last lecture...}

$$\begin{pmatrix} x' \\ y' \end{pmatrix} = \begin{pmatrix} -2 & 1 \\ 0.5 & -2 \end{pmatrix} \begin{pmatrix} x \\ y \end{pmatrix} + \begin{pmatrix} 200 \\ 20 \end{pmatrix}$$
with initial conditions
$$\begin{pmatrix} x(0) \\ y(0) \end{pmatrix} = \begin{pmatrix} 200 \\ 80 \end{pmatrix}$$

The characteristic polynomial is
\begin{align*}
	(-2-\lambda)^2 - \frac{1}{2} &= \lambda^2 + 4\lambda + 4 - \frac{1}{2} \\
				     &= \lambda^2 + 4\lambda + \frac{7}{2} \\
\end{align*}
where the quadratic formula gives us
$$\lambda = \frac{-4\pm\sqrt{2}}{2}$$
Now
$$A - \lambda_1I = \begin{pmatrix} \frac{1}{\sqrt{2}} & 1 \\ \frac{1}{2} & \frac{1}{\sqrt{2}} \end{pmatrix}$$
Where the eigenvector is
$$\begin{pmatrix} \sqrt{2} \\ -1 \end{pmatrix}$$
and similarly
$$\vec{v_2} = \begin{pmatrix} \sqrt{2} \\ 1\end{pmatrix}$$
The solution to the homogeneous system is
$$\vec{\phi}(t) = c_1e^{-\frac{4+\sqrt{2}}{2}t} \begin{pmatrix} \sqrt{2} \\ -1 \end{pmatrix} + c_2e^{-\frac{4-\sqrt{2}}{2}t} \begin{pmatrix} \sqrt{2} \\ 1 \end{pmatrix}$$
Adding the equilibrium solution, we have
$$\vec{x}(t) = c_1e^{-\frac{4+\sqrt{2}}{2}t} \begin{pmatrix} \sqrt{2} \\ -1 \end{pmatrix} + c_2e^{-\frac{4-\sqrt{2}}{2}t} \begin{pmatrix} \sqrt{2} \\ 1 \end{pmatrix} + \begin{pmatrix} 120 \\ 40 \end{pmatrix}$$
Plugging the initial conditions gives us
$$\vec{x}(0) = c_1\begin{pmatrix} \sqrt{2} \\ -1 \end{pmatrix} + c_2 \begin{pmatrix} \sqrt{2} \\ 1 \end{pmatrix} + \begin{pmatrix} 120 \\ 40 \end{pmatrix} = \begin{pmatrix} 200 \\ 80 \end{pmatrix}$$
which simplifies to
$$\begin{pmatrix} \sqrt{2} & \sqrt{2} \\ -1 & 1 \end{pmatrix} \begin{pmatrix} c_1 \\ c_2 \end{pmatrix} = \begin{pmatrix} 80 \\ 40 \end{pmatrix}$$
Taking the inverse of the matrix gives us
$$\begin{pmatrix} c_1 \\ c_2 \end{pmatrix} = \begin{pmatrix} \frac{40}{\sqrt{2}} - 20 \\ \frac{40}{\sqrt{2}} + 20 \end{pmatrix}$$
$c_1$, $c_2$ can be substituted to yield the general solution.

\section{Behaviour of System}
If $A$ has 2 real eigenvalues, we can characterise the solutions by the eigenvalues and eigenvectors of $A$. How do the phase potraits behave?

\begin{ex}
	$$\frac{d\vec{x}}{dt} = \begin{pmatrix} -13 & 6 \\ 2 & -2 \end{pmatrix} \vec{x}$$
	The eigenvalues are given by
	\begin{align*}
		(-13-\lambda)(-2-\lambda) - 12 &= \lambda^2 + 15 \lambda + 26 - 12 \\
					       &= \lambda^2 + 15\lambda + 14 \\
					       &= (\lambda+1)(\lambda+14)
	\end{align*}
	$$A-\lambda_1I = \begin{pmatrix} 1 & 6 \\ 2 & 12 \end{pmatrix}$$
	whose eigenvector is $$\vec{v_1} = \begin{pmatrix} -6 \\ 1 \end{pmatrix}$$
	And the second eigenvector is $$\vec{v_2} = \begin{pmatrix} 1 \\ 2 \end{pmatrix}$$
	The general solution is then
	$$\vec{\phi}(t) = c_1e^{-14t} \begin{pmatrix} -6 \\ 1\end{pmatrix} + c_2 e^{-t} \begin{pmatrix} 1 \\ 2 \end{pmatrix}$$
	As $t$ goes to $\infty$, the solution tends to 0. As $t$ goes to $-\infty$< the solution behaves like its first term.
	If one were to draw a phase potrait (remind me to add one after the annotated slides are out), there would be an equilibrium at the point $(0,0)$. However, non equilibrium solutions can never reach the point; they only tend to it. If this were a nonhomogeneous system, the equilibrium would be shifted by $\vec{x_{eq}}$.
\end{ex}

\begin{ex}
	$$\frac{d\vec{x}}{dt} = \begin{pmatrix} 0 & 1 \\ -6 & 5 \end{pmatrix}\vec{x}$$
	The general solution is
	$$\vec{\phi}(t) = c_1e^{2t} \begin{pmatrix} 1 \\ 2\end{pmatrix} + c_2e^{3t} \begin{pmatrix} 1 \\ 3\end{pmatrix}$$
	The phase potrait diverges away from the equilibrium. This gives us an unstable equilibrium.
\end{ex}
\end{document}
