\documentclass[12pt]{article}
\title{Actor-Network Assignment}
\author{Daniel Chua}
\begin{document}
\maketitle

\section{Brief Description}

The diagram above shows the actor network of me as a Discord user. Discord is a social media which allows users to make friends and form groups called servers, which are further divided into channels. 
% At the centre of the diagram is me. I am connected to my friends and the servers I am in, which is where most of my activities on Discord happen. The servers link me with different groups, such as Second Year students in Engineering Science, students in the same clubs, friend groups, etc. They can be broadly divided into academic servers, where academics and courses are discussed, interest groups, where people with the same interest gather and share, and friend groups, where friends maintain and develop their interpersonal relationships. 

\section{Analysis}

\subsection{Stability}

The segment of my actor network revolving academics has maintained stability for over a year. It happens on the 2T5 Discord Server, where academics is discussed in different channels for different courses. This organisation is encouraged by the server, which by default comes with different channels participants can selectively mute. This combined with enforcement by moderators limit the topic of discussion, which maintains stability of the network. This leads to punctualisation, where the combination of Discord design philosophy, administrative planning (creation and naming of channels), moderation, and our abidance to the rules result in this network being viewed as an individual channel. As a result of punctualisation, my resistance against over-classifying discussion topics has been hidden. I believe it is more natural to let conversation flow without human interference, yet punctualisation causes me to conform to the rules, making it difficult for me to highlight concepts shared by several courses, or simply to raise issues that do not seem to fit any channel. Despite that, I believe this network has enduring stability because participation is entirely voluntary. There are no exclusive announcements, and students can form their own study groups in real life outside of Discord, hence participation is not encouraged but merely allowed. As stated above, stability depends on conformity to the rules and practice as set by the moderators, in turn influenced by Discord settings. Since human actors voluntarily form this network, they generally follow such rules. In addition, the organisation of the server in the form of channels has its merits. Given the majority of students are taking 6 courses concurrently, organisation is necessary for discussion to be feasible. Considering that many active students on the server are already familiar with Discord's approach to organising servers, it follows that this system would meet relatively less resistance.

\subsection{Change}

I recently joined a server with my friends in Hong Kong. This led to a translation in the way I use Discord. Previously, I mostly used Discord to communicate with friends and groups in Toronto. English was the lingua franca, and I quickly learnt how to express myself in English in a casual context. This includes reactions, emotions, or even emojis. Obviously, these did not apply in Cantonese. More importantly, I realised my style of communication, tone and language in English are all slightly different from that in Cantonese. I developed a resistance against communicating in other languages on Discord. This led to the depunctualisation of an actor I was not even aware of: my style of communication. This is something that affects my every interaction on Discord, and influences the conversations I join, the friends I make and the people I associate with on the platform. This actor can be depunctualised into my tone, thinking style, content, and language, which are all interconnected. Since I use English and Cantonese in different contexts, the topic of conversation may determine the language I use. I prefer English for academic discussion, but Cantonese for banter in general. Depending on the language I use, my tone changes. This led me to realise I should not take the way I express myself for granted. I began to experiment with different styles of communication, such as code-switching, to learn how different styles impact my coherence.


\subsection{Non-human Actors}

A significant non-human actor is my online persona which affects how I meet people and act on servers. For example, I am more interested in science and mathematics academically, implying I am more likely to be active on related channels or servers, and vice versa. It also defines my character, such as my tone, activity, and responses. For example, I am more serious on larger servers, such as club servers, where I use a more formal tone. I also avoid being too active, as I do not feel too "involved". On the contrary, on smaller servers, I am more active, more likely to offer help, and more casual. The different online personas I adopt across servers or even channels encourage me to act in certain ways. I give up my power to do as I please in exchange for stability in my online social life. Any sudden changes in my actions would in turn affect my online persona, leading to unexpected results from other human actors who act as mediators. For example, I may rightly receive complaints if I joke with strangers in the same fashion I joke around with my friends. People whom I do not know personally may exclude or simply ignore me if I abruptly became more active and opinionated on their servers. Moreover, I treasure my relationships with other online, and I do not wish to risk damaging them. If I were to be as passive with my close friends as I am on bigger servers, it would be more difficult to maintain relationships. Similarly, I fear my friends' reactions to my sudden change in behaviour. My resistance to being "defined" or "classified" as a certain character was overcome due to my fear of unknown change.
\end{document}
