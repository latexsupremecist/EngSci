\documentclass[12pt]{article}
\usepackage[margin=0.5in]{geometry}
\usepackage{../../template}
\title{Peer Review}
\author{Daniel Chua}
\begin{document}
\maketitle

\section{First Read Through}

The abstract is a bit too long, and it should be around half of its current length, or less. There are some details that can be left to the introduction. The report itself is on the longer end at 8 pages, but this can be easily fixed by efficiently using space for figures, e.g. by scaling/sketching them to fit page width, or putting them side-by-side. The introduction is detailed, making it easy even for physicists in another field to follow. The methods section is similarly detailed, but there is no need to include information specific to the exact apparatus, as these are irrelevant or even misleading for those using similar but different apparatus. In results/analysis, clear description of the experiment has been provided, increasing reproducibility. The conclusion summarized the theory and the experiment highlighting deviations from theory, with hypotheses explaining such effects, and further steps. As for citations and references, it would be better to cite the references in the first occurrence, e.g. citation$^1$.

\section{Second Read Through}

The title reflects the topic of the experiment, He-Ne lasers. It can be slightly reworded, as "set-up" may be a bit vague. Consider titles such as "Producing a He-Ne Laser Beam". The abstract more than summarizes the report. Numbers such as 633 nm, 99.99\%, etc., which are taken for granted and aren't experimental findings, need not be in the abstract for succinctness. The latter half of the abstract is focused on experimental procedures, which should belong to the relevant sections (Methods, Results/Analysis). Simply stating experimental results would suffice.

The introduction does a thorough review of theory behind the experiment. The depth and scope are appropriate; some equations tangential to the experiment are simply given instead of derived, and the variables/constants involved are clearly defined. It assumes no prerequisites in mathematics or physics, giving a complete background of the experiment. All references are appropriately documented.

The methods section provides sufficient information to replicate the experiment. Equipment used is listed, and clear diagrams are used to describe the set-up and arrangement. The exact motions (e.g. which knobs to use) are included, though that may be excessive, since it may not be applicable for different apparatus. It would be more concise to cite the manual, e.g. "the desired adjustment can be performed following the instructions of...". The procedures do follow the scientific method. Figures used are original, and their placements and captions make their purpose and content obvious to the reader. For example, it is a good idea to include a figure of the set-up below the paragraph describing the set-up.

Results/Analysis was detailed and readable. The exact steps taken were documented ("We made small adjustments... So, we turned off..."), making it easier for a reader to follow and even replicate. The difficulties encountered, and possible reasons are clearly indicated, so one can understand the results even if they do not fully match theory. There are no comments for data analysis since the experiment does not involve (numerical) data collection.

The conclusions briefly summarize the experiment. As in the abstract, specific numbers and information can be removed in the first paragraph, since theory has been covered in the introduction, and readers expect to see experimental results in this section instead. There is also excessive details that are mentioned previously in the analysis, such as steps taken. It is enough to state the divergence between theory and experimentation, and suggestions for further steps to verify the new hypotheses.

There are also minor grammatical errors, but those can be easily fixed with spellchecking tools.

\end{document}
