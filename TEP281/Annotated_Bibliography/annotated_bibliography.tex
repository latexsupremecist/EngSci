\documentclass[12pt]{article}
\usepackage{../../template}
\author{Daniel Chua}
\title{Annotated Bibliography}
\begin{document}
\maketitle

\section{Classroom code-switching: three decades of research}

\subsection{Brief Summary}

From the abstract, this paper reviews past research on classroom code-switching, raises difficulties and problems in such research, and concludes with future direction for research.

The author presented three purposes of code-switching

\begin{itemize}
    \item Ideational Functions
    \item Textual Functions
    \item Interpersonal Functions
\end{itemize}

In simple terms, ideational functions refer to using L1 languages to better explain concepts described in L2 terms. For example, a teacher may use a shared native language to "explain, elaborate or exemplify" scientific terms in English. Textual functions refer to topic shifts. For example, a math teacher in Hong Kong may use English to start the lesson, Cantonese to deal with late-comers, then switch back to English to continue the lesson. Finally, interpersonal functions "signal a shift in role-relationships" and "appeal to shared cultural values or institutional norms".

\subsection{Detailed Discussion}
\end{document}
