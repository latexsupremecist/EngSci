\documentclass[12pt]{article}
\usepackage{../../template}
\author{Daniel Chua}
\title{Annotated Bibliography: How does code-switching affect expression?}
\begin{document}
\maketitle

\section{Classroom code-switching: three decades of research~\cite{classroom}}

\subsection{Brief Summary}

From the abstract, this paper reviews past research on classroom code-switching, raises difficulties and problems in such research, and concludes with future direction for research.

The author presented three purposes of code-switching

\begin{itemize}
    \item Ideational Functions
    \item Textual Functions
    \item Interpersonal Functions
\end{itemize}

In simple terms, ideational functions refer to using L1 languages to better explain concepts described in L2 terms. For example, a teacher may use a shared native language to "explain, elaborate or exemplify" scientific terms in English. Textual functions refer to topic shifts. For example, a math teacher in Hong Kong may use English to start the lesson, Cantonese to deal with late-comers, then switch back to English to continue the lesson. Finally, interpersonal functions "signal a shift in role-relationships" and "appeal to shared cultural values or institutional norms".

\subsection{Detailed Discussion}

The functions as described above give us the main purposes of code-switching in the classroom context, and how code-switching can be used to express different things, such as academic content, topic, or interpersonal relationships and cultural values. This signifies code-switching can be as simple as using a more familiar tongue to explain abstract concepts, or as complicated as building a sense of identity. An example given in the text is code-switching in Tamil and English, which "defies both the Tamil-only ideology in the public domains and institutions, and the English-only ideology from the ESL/TESOL pedagogical prescriptions from the West".

Apart from providing an insight as to how code-switching is used, this paper is also relevant because it discusses the stigma of code-switching, which can be seen as an inferior command of one or both languages. It discusses the challenges faced by ideological forces, where researchers are "working against the grain of dominant theories of the field". Therefore, they feel the constant need to justify and prove themselves. However, research has shown that in an educative setting, code-switching at least does no harm, and there are suggestions that it may even boost learning and understanding when measured in scores.

\section{ELF couples and automatic code-switching~\cite{couples}}

\subsection{Brief Summary}

This paper focuses on on ELF couples, who use English as a lingua franca. The article focuses on code-switching in the private sphere, with the conclusions "that sometimes ELF couples code-switch automatically, without noticeable awareness of switching", and "ELF couples allow a relaxed attitude towards language mixing within the couple’s shared range".

\subsection{Detailed Discussion}

This paper is relevant because it describes how code-switching is occasionally used automatically, without explicit thought behind it. This contrasts with the other articles cited in this annotated bibliography, where code-switching is seen as an intentional act. It is important to note however, that this does not discount the role of code-switching. In addition, this paper explicitly focuses on couples and not families. Unlike in other environments, education or preservation of culture throughout generations is not a major concern in language choice.

In the provided transcripts, the context surrounding each incidents of code-switching were thoroughly investigated, to show how it is done naturally without thought, as suggested by the topic of conversation, intonation, the partner's response, etc. In fact, for couples who share several fluent languages, they commonly incorporate code-switching between those languages, such as how a Finnish/Icelandic couple switched between variants of "yes" in different languages.

\section{Code-Switching~\cite{cs}}

\subsection{Brief Summary}

This book serves as an introduction to code-switching, and discusses it in a broad sense, unconstrained by any specific lenses unlike the other sources cited in this bibliography. It summarises the various characteristics of code-switching, and highlights the common features that hold in different contexts and different kinds of code-switching.

\subsection{Detailed Discussion}

The first main conclusion of the variation within code-switching. It varies between communities, individuals, and sociolinguistic factors, such as the relatedness of languages. A heavy influence on code-switching is the belief that there is a grammatically "correct" way of speaking, and that code-switching is wrong. This influences code-switching in different contexts; one is more likely to remain monolingual in a formal setting, or when "being right" is desired, such as when talking to a child, or one who is learning the language(s).

Another point to note is the "fuzziness" of the definition of code-switching. There are certainly different types or code-switching, ranging from the borrowing of a word or a phrase, to the switching of languages between clauses, and even building inflections on foreign words. Linguistically, it can both be a way to converge languages and a way to preserve distinctiveness. In studies, the exact definition of code-switching would impact its results. Code-switching to one might be borrowing to another, or community bilingual choices. The monolingual assumption, which states humans communicate monolingually, encourages researchers to fit a "base language" into their model, and conduct research based on a distinct pair of "primary" and "secondary" languages.

\section{Duelling Languages, Duelling Values: Codeswitching in bilingual intergenerational conflict talk in diasporic families~\cite{conflict}}

\subsection{Brief Summary}

This paper investigates "the bilingual intergenerational conflict talk in diasporic families". This analyses the social factor of code-switching, where both parties appear to deal with a specific conflict. However, behind that is a conflict in culture, values, family status, language, etc. Unlike the general approaches as seen in the papers above, this is an analysis of a specific activity, where code-switching performs a greater interpersonal function, as both languages represent the two different cultures and worldviews.

\subsection{Detailed Discussion}

This paper is valuable, as its narrow scope allows for deeper analysis and specific conclusions on the role played by code-switching. This also deals with immigrant families, the major social setting where code-switching occurs in the West. Moreover, this involves familial roles, which greatly influence power and dynamics in the conflict, and provides an insight of how code-switching is used strategically.

An obvious source of conflict is the use of "they code" versus "we code". In this study which deals with Chinese immigrants in the UK, parents view Chinese as a shared language which represents closeness and solidarity. Their children who were brought up in an English speaking society see English as the language they can best express themselves. In both examples provided in the study, it is interesting to note how the mothers used Chinese to make their own points, and English to address their children's points. The parent may also use Chinese to express their dissatisfaction, and to signify their authority. It is interesting to note the differences in cultural perspectives. The parent would use a Chinese "we", suggesting she viewed the family as a unit, while the daughter would use "I" and "you", separating them into individuals. Another example is that the mothers saw the word "you" as directed and offensive, instead opting for the polite/formal form of the same word in Chinese.


\section{To be, or not to be...Black: The effects of racial codeswitching on perceived professionalism in the workplace~\cite{race}}

\subsection{Brief Summary}

In this study, researchers examined how code-switching of Black people affect perceived professionalism. It concludes that racial code-switching consistently enhances perceived professionalism, and Black respondents viewed the non code-switching employee as more professional compared with White respondents.

\subsection{Detailed Discussion}

This study deals with the workspace, where unlike the different settings in the previous papers, one is more likely to explicitly act in a way where they would be perceived as more professional. Therefore, this also ties into the self-perception of language use. This is also a workplace context, where personal relationships are not as strong as between couples and families, which were analysed previously. Moreover, this is an example of racial code-switching, where unlike all previous papers, the different codes belong to the same named language, English. There is no concern that code-switching would affect comprehension. Instead, there is a heavier focus on cultural and racial implications.

The study concluded that code-switching is consistently preferred in terms of professionalism. This consistency implies code-switching has a real effect on how people perceive each other, holding other factors constant. The differences in responses between Black and White respondents also hint how the use of language may enhance solidarity and a sense of community. It describes how code-switching may be perceived, and highlights the difference in perception based on racial groups. Interestingly, it also touches on the social effects of code-switching, and suggests how language choices can shape social norms.

\bibliography{references}{}
\bibliographystyle{IEEEtran}

\end{document}
