\documentclass[12pt]{article}
\usepackage{../../template}
\author{Daniel Chua}
\title{ESC301 Literature Review Assignment}
\begin{document}
\maketitle

\section{Literature Review}

\subsection{Research Methods and Context}

All of the articles were found using Google Scholar, with the keywords "nanosheet transistor". Generative AI was not used.

\subsection{Review}

The topic of this literature review is nanosheet transistors as the next generation of transistors. The works chosen are "Stacked nanosheet gate-all-around transistor to enable scaling beyond FinFET"~\cite{nanosheet}, "Device and Circuit Exploration of Multi-Nanosheet Transistor for Sub-3 nm Technology Node"~\cite{3nm}, and "A critical review on performance, reliability, and fabrication challenges in nanosheet FET for future analog/digital IC applications"~\cite{review}. These works are selected because they cover a wide range of topics and provide a comprehensive overview of why nanosheets were considered, what benefits they offer, and how they compare with existing technologies. These papers will be synthesized to provide a whole picture; all of the points above are explained and elaborated on by at least one paper.

The first article attempts to demonstrate why "nanosheet structure is a good candidate for the replacement of FinFETs at the 5nm technology node and beyond". To support this claim, it explains how nanosheets improve effective width, reduce parasitic capacitance, and support a higher relative frequency, even when compared with aggressively scaled FinFETs. These results are backed by actual devices fabricated for this purpose. It shows that device fabrication is possible, and the devices have these desirable characteristics. In addition, possible concerns such as "the deformation or bending of the sheets" are not observed. There is yet to be data and analysis on the integration of these devices, but the above is sufficient evidence for considering nanosheets as a strong candidate for replacing FinFETs.

The second article goes through detailed analysis of the characteristics of technology nodes under 3mm. It starts with the structural information of all mNS-FET devices. Due to the extremely small dimensions used in the newest devices, conventional approximations for large devices do not apply, and drift-diffusion simulation was performed. The article shows the current-voltage (IV) characteristics, subthreshold slope (SS) and drain-induced barrier lowering (DIBL). As for circuit characteristics, analysis is performed by benchmarking a model circuit. Given these settings, "performance improvement in terms of power consumption and speed of the circuit" are demonstrated. These are verified "through segmentation of the resistance and capacitor components", and the analysis supports the results above. Finally, the effects of the position of the bottom oxide is investigated. The advantages and disadvantages of "top", "centre", and "bottom" structures are discussed. The same circuit analysis is repeated, which shows again that the "bottom" structure has the worst performance in terms of delay time, resistance, capacitance, and hence power consumption.

The third article "reviews the fabrication challenges, emerging materials (wafer, high-k oxide, gate metal, channel materials), dimensional influences, thermal effects, growth techniques utilized, performance, and reliability concerns involved in Nanosheet FET". It starts with fabrication challenges, including the choice of substrate material, gate oxide, gate metal, channel material, and factors that influence these choices. It then compares nanosheets with other common technologies such as nanowires and FinFETs, based on gate length scaling, sheet dimensions, and temperature. The paper ends with comments on reliability concerns, and suggests future research directions. It concludes that at least for analog/RF applications, nanosheets have superior performance.

These articles are all complementary in providing an overview of nanosheets. All articles mention in the introduction the background of nanosheets; Moore's Law~\cite{moore} states that the number of transistors in an integrated circuit doubles every two years. For it to hold for roughly the same design, transistor size needs to be reduced by half. However, this has been made increasingly difficult for FinFETs (the current technology) due to reasons including manufacturing difficulties and physical barriers such as quantum mechanical phenomenon at smaller scales. However, the first article provides further motivation. Through testing actual fabricated samples, it confirms that nanosheets can in fact address these edge effects and limitations that trouble FinFETs as they are aggressively scaled. This also addresses manufacturing difficulties, simply because the nanosheet sample itself can be produced according to specifications, as measured. The second article supports this by analyzing a multitude of designs and simulation for various characteristics. In addition, it performs circuit analysis to show that performance improvements are inherited when these designs are applied in a more practical scenario. However, the second article does not provide support as strong, since simulations depend on assumed physical properties and phenomena. It is possible that at these small scales, unknown or ignored effects will begin to dominate, which impact validity of simulation results. Regardless, but paper provide sufficient motivation in terms of a detailed list of predicted improvements, as well as actual verifiable improvements for a certain structure.

All articles provide comparisons between nanosheets and other technologies, e.g. FinFETs. However, the first and second articles merely stated several characteristics where nanosheets had superior performance. The final article, being written the latest (2022 vs 2017, 2021), provided analysis on the widest range of design choices, which are used in a more in-depth comparison involving ballistic current, electrostatic control, SS, and DIBL. The uncertainties of these values based on material imperfection is also considered. More data and information was available when the paper was written, allowing it to support its claims with more evidence. With more research having been done on nanosheets (when it was written), its conclusions regarding the performance improvements of nanosheets have more authority than the other papers, which either relied only on simulated data, or on one specific design.

The above review generally supports the introduction of nanosheets to replace FinFETs, especially at smaller dimensions. This is insightful as manufacturers plan to begin production of nanosheet transistors in 2025~\cite{tsmc}. Key shortcomings of this review is insufficient articles and timeliness. Three articles are not enough to provide an in-depth review of the nanosheet technology. Ideally, every claim or evidence should come from multiple sources, so it is possible to confirm findings are not due to flawed methodology or any other unique factor. Next steps could be to produce more nanosheet samples to further verify predictions for a variety of designs, and to implement them on circuits to find performance improvements for the system as a whole.

\section{Personal Reflection}

In preparing for this literature review, I gained more practice in skimming research papers. Most papers are long enough that it is not feasible to fully read the entire article to extract necessary information, or to determine if it is a suitable article to include in the review. From this experience, I learnt that it is very helpful to start with a goal in mind. With that, I could use keywords and filter functions to put more relevant papers in the top search results, which greatly reduces time. Secondly, I could skim through the abstract, introduction, and conclusion with this in mind, which helps me determine if the paper is relevant or not. For example, when I was looking for papers which implemented specific nanosheet models, I looked for numbers and units, which would tell me how specific the analysis is. In addition to techniques, I have also learnt more about the structure of nanosheets and the specific areas where it performs better than FinFETs and other technologies.

These skills are valuable in my future aspirations. Currently, I am interested in working in the semiconductor industry. This necessarily implies researching and learning new information through articles, since new developments are being made in the field every day. As a student, I struggle with papers since unlike textbooks or lectures, it is impossible to fully understand concepts, and there is no structured syllabus designed to help me. Often, I have to branch out my research to include foundational concepts that are assumed in these articles. Throughout this assignment, I have gained experience in identifying foundational background knowledge for each new concept. It is difficult and impractical to learn a new field just for a paper, but I have gotten better at understanding what minimum prerequisites are required. I think I am the strongest in searching for sources, since I have found it relatively easy to locate keywords and search specifically for content I want. I feel I could use further practice on synthesis; it has been difficult to combine the papers and see the big picture they sketch. To address this, I can start by reading literature reviews. There are in fact several journals which only contain reviews in their respective fields.

\bibliographystyle{plain}
\bibliography{refs.bib}
\end{document}
