\documentclass[12pt]{article}
\usepackage[margin=1in]{geometry}
\usepackage{../../template}
\title{Alumni Interview Assignment}
\author{Daniel Chua}
\begin{document}
\maketitle

\section{Reflection}
I interviewed Dr. Lee Liu, a "postdoctoral research associate working on infrared spectroscopy of buffer gas cooled fullerenes"~\cite{lee}. During the 30-minute interview, I asked him a list of questions (from Section \ref{qs}). I started by asking him about his current research, to see if it is related to my current or previous work in labs. It turned out that I had recently worked on something similar, so it could be helpful to contact him in the future, if I wanted to dive deeper into the field. \\
I then asked him about his journey going into research. Being in the same major, I thought it would be likely that he had gone through the same experiences, though processes, and even doubts, and it could be helpful for me to gain insights from him. He shared about how his summer research experiences in second year inspired him to pursue research post-graduation. Initially, he was interested in specific topics and fields instead of research for research's sake. However, as he went through different research areas, he began to develop an interest in research in general as well. He enjoyed the process, admitting that he was at times more interested in the problem-solving aspect over the science behind it. \\
Afterwards, I asked about the transition process from Engineering Physics to academia. I had concerns that Engineering Physics, as an Engineering program, would not prepare me as well as a Physics program. According to Lee, Engineering Science prepares one for the fundamentals in the first 2 years of the program. These are useful to fall back on when learn new material. Derivation from first principles, something emphasized in in the program, is often helpful. He also highlighted how engineering skills are directly applicable to laboratory work, and in this sense, Engineering students perform better than Physics students. Ultimately, his advice for me was to not worry about being behind Physics students in terms of theory; there is always something you don't know, and it is totally fine to learn things as you go. He encouraged me to focus on my thesis in 4th year, since it can inform me if I am interested in the topic or not. \\
The interview has been very informative. The interviewee provided me with actionable advice (e.g. focus on labs and thesis) that could help me figure out if I am interested in research, and if so, which field. He also let me know that it is not as important to catch up on physics knowledge to gauge my interest, which greatly saves time and stress. At this point, it is still too early to know for sure my future career, but I can now develop strategies to explore my compatibility with research. \\
As for my workview, I think my values regarding a healthy work-life balance are reflected in the career of research. Researchers generally have flexible work hours as compared to industry.

\section{Interview Questions} \label{qs}

This is a list of interview questions prepared prior. All questions were asked during the interview.
\begin{itemize}
    \item Can you share about your current research?
    \item When did you start to consider research as a career? When were you committed to this decision?
    \item How did Engineering Physics prepare you for a career in research?
    \item Is there anything outside of school that I should learn about?
    \item In Engineering Physics, we have to take Advanced Physics Laboratories (APL). Did these lab courese prepare you for research in the field?
    \item If you could do undergrad again, would you have picked Physics over Engineering?
    \item As an Engineering (not Physics) student, is there anything you think I should catch up on?
\end{itemize}

\bibliography{refs.bib}
\bibliographystyle{IEEEtran}

\end{document}
