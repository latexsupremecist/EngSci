\documentclass[12pt]{article}
\usepackage{../../template}
\title{Lecture 33}
\author{niceguy}
\begin{document}
\maketitle

Recall diffeomorphisms. For $g:A\rightarrow B$, where $g$ is bijective, $C^r$ and where $A,B$ are open subsets of $\R^n$, with $D(g) \neq 0 \forall x \in A$, we call $g$ a \textit{diffeomorphism}. Its inverse $g^{-1}$ is also a diffeomorphism.

\begin{lem}
    Diffeomorphisms preserve sets of measure zero. If $g:A \rightarrow B$, $S \subseteq A$ has measure zero, then $g(S) \subseteq B$ also has measure zero.
\end{lem}

\end{document}
