\documentclass[12pt]{article}
\usepackage{../../template}
\title{Lecture 50}
\author{niceguy}
\begin{document}
\maketitle

\section{Permutations}

This is a bijective function $\sigma:\{1,\dots,n\} \rightarrow \{1,\dots,n\}$. An elementary permutation is one that swaps 2 adjacent numbers and keeps everything the same, where

\begin{itemize}
    \item $e_i(j) = j$ if $j \ne i, j \ne i + 1$
    \item $e_i(i) = i+1$
    \item $e_i(i+1) = i$
\end{itemize}

Note that $e_i$ is its own inverse. We can easily prove that any permutation $\sigma$ can be expressed as the composition of finitely many switches. (Proof: by induction, show that a finite number of switches can be used to make $\sigma(1)$ be equal to an arbitrary $i$. This holds for $\sigma(j)$ by induction.)

\begin{defn}[Sign of Permutation]
    For any permutation $\sgima$ consider the number of pairs $(i,j)$ with $i < j, \sigma(i) > \sigma(j)$. $\sigma$ is odd if the number of such pairs is odd, and even if it is not odd. Odd permutations have a sign of -1, and even permutations a sign of 1.
\end{defn}

\begin{prop}
    Let $\sigma$ be expressed as $e_{i1} \circ \dots \circ e_{il}$. Then its sign is $(-1)^l$.
\end{prop}

\begin{proof}
    It suffices to show that a permutation changes sign when it is composed with an elementary operation. Consider the set of all pairs $(i,j)$ as described above. If the elementary operation is $e_k$, then any pair involving at most one of $k, k+1$ remains in the set. If $(k,k+1)$ is in the set, it will be removed, and vice versa. Then the cardinality of the set changes by 1, so the sign changes. It is obvious that the identity has a sign of 1. Then letting $\sigma = e_{i1} \circ \dots \circ e_{il} \circ 1$ makes it obvious that its sign is $(-1)^l$. As a consequence, it is trivial to show that
    $$\text{sgn}(\sigma \circ \tau) = \text{sgn}(\sigma) \times \text{sgn}(\tau)$$
    and
    $$\text{sgn}(\sigma^{-1}) = \text{sgn}(\sigma)$$
    If we swap any two random indices separated by $n$, note that $2n-1$ swaps are needed, so the sign switches.
\end{proof}

With permutations, we can define

$$f^\sigma(v_1,\dots,v_k) = f(v_{\sigma(1)}, \dots, v_{\sigma(k)})$$
We can then treat $\sigma$ as an operation on tensors. It is an isomorphism! (It is obviously injective, and setting $g = f^\tau$ with $\tau = \sigma^{-1}$ shows that it is surjective) \\
Fun fact:
$$(f^\sigma)^\tau = f^{\sigma \circ \tau}$$

We could generalise symmetry by saying $f$ is symmetric when $f^{e_i} = f$, and antisymmetric when $f^{e_i} = -f$.
\end{document}
