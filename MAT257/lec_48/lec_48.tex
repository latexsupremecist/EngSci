\documentclass[12pt]{article}
\usepackage{../../template}
\title{Lecture 48}
\author{niceguy}
\begin{document}
\maketitle

\section{Multilinear Maps}

\begin{lem}
    The dimension of $\mathcal L^1(V) = V^*$ is the same as that of $V$.
\end{lem}

\begin{proof}
    We proved this previously. For finite-dimensional $V$, $\phi_i(v) = \langle v, e_i\rangle$ forms a basis (check it!). For infinite-dimensional $V$, we can show that any list of linearly independent $v_i$ provides us with a list of linearly independent $\phi_i(v) = \langle v, v_i\rangle$ of the same length. Since the length of $v_i$ can be made arbitrarily long, we have arbitrarily many linearly independent $\phi_i$, so $V^*$ is infinite-dimensional.
\end{proof}

Extending this to multilinear maps,

\begin{align*}
    T(v_1,\dots,v_k) &= T(\lambda_{1j}v_j,v_2,\dots,v_k) \\
                     &= \sum_{j1} \lambda_{1j_1} T(v_{j1},v_2,\dots,v_k) \\
                     &= \sum_j \prod_i \lambda_{ij_i} T(v_{j1},\dots,v_{jk})
\end{align*}

This suggests that the dimension of $\mathcal L^k(V)$ has at most dimension $n^k$, where for any tuple $\sigma$ (does not have to be injective) out of all $n^k$ possibilities, there is a basis vector
$$\phi(\sigma(1),\dots,\sigma(k)) = 1, \phi(\sigma'(1),\dots,\sigma'(k)) = 0, \sigma \neq \sigma'$$

\begin{proof}
    We know from the previous lecture that these $\phi$ span the space. What remains is to show that they are linearly independent. Denote maps by $\phi_\sigma$. Then for an arbitrary tuple $\tau$,
    \begin{align*}
        a_i\phi_i &= 0 \\
        a_i\phi_i(\tau(1),\dots,\tau(k)) &= 0(\tau)
    \end{align*}
    By definition, exactly one of $\phi_i$ returns 1, and all other functions return zero. Then the corresponding $a_i = 0$. Since this is arbitrary, we can pick suitable $\tau$ to force all $a_i$ to vanish, hence $\phi_i$ are linearly independent.
\end{proof}

\end{document}
