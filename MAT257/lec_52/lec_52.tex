\documentclass[12pt]{article}
\usepackage{../../template}
\title{Lecture 52}
\author{niceguy}
\begin{document}
\maketitle

\section{Recap}

Tensors of rank $k$ over a vector space $V$ is a multilinear map
$$T:V^k \rightarrow \R$$

Given a basis of $V$ $e_1,\dots,e_n$, we have a basis for $V^k$ composed of $\Phi_I(e_{j1},\dots,e_{jk})$ which is equal to one if $(j1,\dots,j_k) = I$, where $I$ is an ordered list (that possibly repeats). The dimension is then $n^k$. \\

Alternating $k$ tensors are tensors whose sign changes when any two vectors are swapped, i.e.
$$T(v_1,\dots,v_i,v_{i+1},\dots,v_k) = -T(v_1,\dots,v_{i-1},v_{i+1},v_i,v_{i+2},\dots,v_k)$$

They form a linear subspace. Since the sign of permutations are well-defined, we can alternatively define alternating tensors as
$$T^\sigma = \text{sgn}(\sigma)T$$

\section{Altnernating tensors}

We are looking for a basis for $\mathcal A^k(V)$, the set of alternating tensors. Consider a set of ordered $k-tuples$. Unlike the $\phi_I$ as above, we do not need to know the order of the permutation, since it is determined (by definition). Now given a permutation $\bar I$ with increasing indices, e.g. $(v_1,v_2,v_3,v_5,\dots)$, we can similarly define a $\phi_{\bar I}$, which has value $\pm1$ on permutations of $\bar I$ depending on the sign. We want to say the set of this $\phi$ forms a basis. It is obvious that they are linearly independent. For an arbitrary tensor $f$, letting $a_I$ be $f(v_{i1},\dots,v_{ik})$ with $I$ being increasing, consider $g = \sum_I a_I\phi_I$. For an arbitrary $w \in V^k$, we can write $w$ as a sum of basis vectors. For both $f,g$, the function is only (possibly) nonzero on basis vectors where there are no repeating indices. For the rest, the fact that $f,\phi$ are alternating ensures $f(w) = g(w)$. Since $w$ is arbitrary, we can say that $\phi_I$ spans $\mathcal A^k(V)$, completing the proof.

\begin{ex}
    Determinants are alternating tensors.
\end{ex}

\section{Wedge Product}

We want an operation
$$\bigwedge:\mathcal A^k(V) \times \mathcal A^l(V) \rightarrow \mathcal A^{k+l}(V)$$
We know we can take the tensor product $w_1 \otimes w_2$, which is alternating except (possibly) for switches between the first $k$ and last $l$ entries. We then desire an operaton to make it alternating.

\end{document}
