\documentclass[12pt]{article}
\usepackage{../../template}
\title{Lecture 64}
\author{niceguy}
\begin{document}
\maketitle

\section{Pull Backs}

Recall $\alpha^*\omega \in \Omega^l(U)$ such that
$$\alpha^*\omega(x;v_1,\dots,v_l) = \omega(x;\alpha_*(x;v_1),\dots,\alpha_*(x;v_l)) = \omega(x;D\alpha(x)\cdot v_1,\dots,D\alpha(x)\cdot v_l)$$

This definition shows that

\begin{align*}
    \alpha^*dx_I(x;v_1,\dots,v_l) &= dx_{i1} \wedge \dots \wedge dx_{il}(x;D\alpha(x)\cdot v_1,\dots,D\alpha(x)\cdot v_l) \\
                                  &= D\alpha_{i1}(x)\cdot v_1 \wedge \dots \wedge D\alpha_{il}(x)\cdot v_l \\
    \alpha^*dx_I &= d\alpha_{i1} \wedge \dots \wedge d\alpha_{il}
\end{align*}

By linearity, we see $\alpha^*$ and $d$ commute. The next challenge is to integrate forms on manifolds. We define the integral of the form $\omega = fdx_I$ be
$$\int_A\omega = \int_A f$$

Letting $\alpha$ be the coordinate chart from $A$ to $Y$,
$$\int_Y \omega = \int_A \alpha^*\omega = \int_A (f\circ\alpha)\det\left(\del{\alpha_I}{x}\right)$$

For the general $\omega = f_Idx_I$, summation suffices.

\end{document}
