\documentclass[12pt]{article}
\usepackage{../../template}
\title{Lecture 8}
\author{niceguy}
\begin{document}
\maketitle

\section{Differentiation for Functions of Many Variables}

We can rewrite $f:U \rightarrow \R^n$ as

$$f(x) = \begin{pmatrix} f_1(x) \\ f_2(x) \\ \vdots \\ f_n(x) \end{pmatrix}$$

The derivative of a function $f$ at $x_0 \in U$ can be thought of as a matrix or a linear map. This means

$$f(x) = f(x_0) + g(h) + R(h), R(h) = o(h)$$

where $g$ is linear and $R$ is a correction term that tends to 0 faster than $h$. Now this linear map can be represented by a unique matrix (up to basis).

Now define

$$Df(x_0)_{ij} = \del{f_i}{x_j}(x_0)$$

We can prove that this matrix is the desired linear map. For simplicity, assume $f:\R^m\mapsto\R$. Then using that fact that $\del{f}{x_i} = Df\cdot e_i$, we see that $Df$ is the derivative. We can generalise this by applying the theorem on each row of $f$, ie each dimension of its range.

\begin{lem}
    $$f'(a;u) = Df(a)\cdot u$$
    where $u$ is a unit vector.
\end{lem}

\begin{proof}
    We assumed differentiability for the LHS to make sense. Then letting $B$ be its derivative,
    \begin{align*}
        \lim_{t\rightarrow0} \frac{f(a+tu) - f(a) - Btu}{|tu|} &= 0 \\
        \lim_{t\rightarrow0} \frac{f(a+tu)-f(a)}{t} - Bu &= 0
    \end{align*}
\end{proof}

\begin{lem}
    If $f$ is differentiable at $a$, then
    $$Df(a) = \begin{pmatrix} D_1f(a) & D_2f(a) & \dots & D_mf(a) \end{pmatrix}$$
\end{lem}

\begin{proof}
    Note that $Df(a)e_i = f'(a;e_i)$. Then letting $Df(a)_j = \lambda_j$, note that
    $$D_jf(a) = f'(a;e_j) = Df(a)e_j = \lambda_j$$
\end{proof}

\end{document}
