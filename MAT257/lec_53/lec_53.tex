\documentclass[12pt]{article}
\usepackage{../../template}
\title{Lecture 53}
\author{niceguy}
\begin{document}
\maketitle

\section{Wedge Product}

We will prove that given the conditions, the wedge product is unique. We can also show that the wedge product of ascending linear functionals form a basis, i.e.
$\bigwedge_{j=1}^k \phi_{ij}$ where $i1 < i2 < \dots < ik$ form a basis of $A^k(V)$.

\begin{ex}[One Tensors]
    Let $V$ be the space of vectors in $\R^n$ based at the origin. Fix a $C^1$ function on $\R^n$, where $f:\R^n \rightarrow \R$, then define the tensor
    $$w(v) = Df(v)$$
    Choosing $f(a_ix_i) = a_1$, we get $w = \phi_1$.
\end{ex}

\section{Tangent Space of a Manifold}

The tangent space of $\R^n$ of a point $p$ is the space of vectors $v \in \R^n$ based at $p$. Every tangent vector is denoted as $(x;v)$, i.e. a vector (arrow) $v$ that starts at (point) $x$. For a coordinate chart $\alpha$, define
$$\alpha_*(x;v) = (\alpha(x);D(\alpha) \cdot v)$$
\end{document}
