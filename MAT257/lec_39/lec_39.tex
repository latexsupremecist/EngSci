\documentclass[12pt]{article}
\usepackage{../../template}
\title{Lecture 39}
\author{niceguy}
\begin{document}
\maketitle

Consider linear subspaces $Y^k \subseteq \R^n$. We want to define notions of objects and functions \textbf{only} over $V^k$. A special case is $V^k = \R^k \times \{0\}^{n-k} \subset \R^n$. What do we do beyond that? We need isometries.

\section{Isometries}

\begin{defn}[Isometry]
    $\Phi:\R^n \rightarrow \R^n$ is an isometry when
    $$||\Phi(x) - \Phi(y)|| = ||x - y||$$
    In other words, isometries preserve \textit{distance}.
\end{defn}

We can simply have such a $\Phi$ where $\Phi(V^k) = \R^k \times \{0\}^{n-k}$.

\begin{ex}
    If such a map is a linear transformation, it is unitary by definition.
\end{ex}

Linear transformations are isometries iff they are unitary (vectors in matrix form orthonormal basis). It suffices to choose an orthonormal basis of $V^k$.

\end{document}
