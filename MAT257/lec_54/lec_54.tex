\documentclass[12pt]{article}
\usepackage{../../template}
\title{Lecture 54}
\author{niceguy}
\begin{document}
\maketitle

\section{Tangent Spaces}

Loosely speaking, a tangent space is defined by the derivative at a point. More formally, the tangent space of the point $x = \alpha^{-1}(p)$ on $\R^k$ is $(x;v)$, and that of the corresponding manifold is $(p;D\alpha(x) \cdot v)$. The tangent space on the manifold has dimension $2k$ ($k$ dimensions for each of $x$ and $v$).

\begin{defn}[Tangent Bundle]
    The tangent bundle of a manifold $M^k \subseteq \R^n$ is the collection of all tangent spaces, based at all points $p \in M^k$.
    $$\bigcup_{p\in M^k} TM^k|_p$$
    Equivalently, it can be defined as
    $$\bigcup_{\alpha,x,v}(\alpha(x),D\alpha(x)\cdot v)$$
\end{defn}

\begin{defn}
    A vector field along a manifold $M^k$ is a continuous function $V:M^k \rightarrow TM^k$ such that $V(p) \in TM^k|_p$.
\end{defn}

\begin{ex}
    Consider a 1-dimensional manifold defined by $\alpha:(0,1)\rightarrow\R^3$. The corresponding vector field is a line.
\end{ex}

\end{document}
