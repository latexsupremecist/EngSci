\documentclass[12pt]{article}
\usepackage{../../template}
\title{Lecture 47}
\author{niceguy}
\begin{document}
\maketitle

\section{Midterm}

The median and average is 8.5/15, 30\% scored 10 or above, many people got less than 5 (i.e. in trouble).

\section{Integrals}
Recall integrals of continuous functions $f$ defined over compact manifolds $M^k \subseteq \R^n$ if entirely paramtrizable, is
$$\int_{M^k} f = \int_U (f\circ\alpha) V[D\alpha]$$
For compact manifolds, there are finetely many $U$, so it can be simply written as a sum of integrals with corresponding $U$ and $\alpha$. What remains is to show that this definition is unique. Consider 2 different of such integrals with finite partitions $V_i$ and $V'_i$. Showing that the integral agrees on every element of the refined partition suffices, i.e. the integral has to agree on arbitrary $V = V_i \cap V'_j$. Now on this set, we denote the restricted functions (on the domain) also by $\alpha$ and $\beta$. Then we have to be able to find a diffeomorphism between $\alpha^{-1}(V)$ and $\beta^{-1}(V)$. Change of variables show that both integrals are equal. \\

To define integrals over a general manifold, we need a partition of unity for $M^k$. Multiplying by the partition of unity, we can reduce the integral to a finite sum of integrals of $f\phi_i$ over compact sets, which is well defined. The textbook shows how the value of the integral is independent of the partition of unity. \\

Now we're done with classical multivariable calculus and numbers in general.

\section{}

Take $V$ to be a $k$ dimensional vector space. Recall a linear functional $a$ on $V$ is a linear function $a:V \rightarrow \R$. A multilinear functional over $V$ is $a: V^l \rightarrow \R$ where it is linear in each input. We call $a$ an $l$ tensor over $V$.

\begin{ex}
    The inner product is a 2 tensor. In 3D, $det(v_1,v_2,v_3)$ is a 3 tensor.
\end{ex}

Now note that the set of all $k$ tensors form a vector space. The proof is trivial.

\begin{lem}
    Basis vectors uniquely determine multilinear functionals. Let $v_i = c_{ij}e_j$. Then
    \begin{align*}
        f(v_1,\dots,v_n) &= c_{1j}f(e_j,v_2,\dots,v_n) \\
                         &= c_{1j_1}f(e_{j_1},v_2,\dots,v_n) \\
                         &= \left(\prod_i c_{ij_i}\right) f(e_{j_1},\dots,e_{j_n})
    \end{align*}
    Note that $j_1,\dots,j_n$ need not be distinct or in any order. Then if the last term is known for all possible $j_i$ lists, any arbitrary $f(v_1,\dots,v_n)$ can be determined.
\end{lem}

\begin{lem}
    If $V$ is finite dimensional, so is $\mathcal L(V)$, its dual space.
\end{lem}

\begin{proof}
    It suffices to construct a basis. Letting $e_i,\dots,e_k$ be a basis of $V$. Then any $v \in V$ can be uniquely written as $c_ie_i$. Using the same notation, let $\phi_i(v) = c_i$. For an arbitrary linear functional $\phi$, if we know $\phi(e_1),\dots,\phi(e_n)$, we know $\phi(v)$ by linearity. Letting $\phi(e_i) = d_i$, it is obvious that $\phi(v) = d_i\phi_i$. We now know that $\phi_i$ spans $\mathcal L(V)$. To show that they are linearly independent, simply note that the only linear combination that adds up to zero is the trivial one.
\end{proof}

\end{document}
