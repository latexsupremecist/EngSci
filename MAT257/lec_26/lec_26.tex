\documentclass[12pt]{article}
\usepackage{../../template}
\title{Lecture 26}
\author{niceguy}
\begin{document}
\maketitle

\section{Rectifiable Sets}

\textit{Note: what we call a rectifiable set in this course is different from what they mean conventionally. We use textbook naming for consistency.}

\begin{defn}[Rectifiable Set]
    A bounded set $S \subseteq \R^n$ is rectifiable if  its boundary has measure zero.
\end{defn}

\begin{rem}
    $S$ is rectifiable iff
    $$\chi_S(x) = \begin{cases} 1 & x \in S \\ 0 & x \notin S \end{cases}$$
    is integrable.
\end{rem}

Define for $Q \supseteq S$
$$V(S) = \int_S 1 = \int_Q \chi_S$$

Recall we will exclusively be considering continuous functions for now on.

\subsection{Properties of Rectifiable Sets}

\begin{enumerate}
    \item If $S_1, S_2$ are rectifiable, then $S_1 \cup S_2, S_1 \cap S_2$ are rectifiable
    \item If $f:S \rightarrow \R$ is a bounded continuous function, then $f$ is integrable and $S$ is rectifiable
    \item If $S_1,S_2$ are rectifiable, then
        $$v(S_1 \cup S_2) = v(S_1) + v(S_2) - v(S_1 \cap S_2)$$
\end{enumerate}

\begin{proof}
The first is easy to prove. For the union, the boundary of the union is a subset of the union of the boundaries. The union of sets with measure zero has measure zero, and a subset of a set with measure zero also has measure zero.

For the second, consider the extension of $f_S(x)$ over a rectangle $Q \supset S$ and show that the set of discontinuities has measure zero. The set of discontinuities of $f_S$ is contained in the boundary of $S$.
\end{proof}

We will be considering continuous functions over open sets which need not be bounded. Our next goal is to define integrability for such functions. Consider $f(x) = \frac{1}{x^p}, p > 0$ fixed. Define integrability by studying

$$\lim_{\epsilon\rightarrow0} \int_\epsilon^1 \frac{1}{x^p} dx$$
Integrability wants that the limit exists and is finite. Then define
$$\int_0^1 f(x)dx = \sup_{\epsilon>0} \int_\epsilon^1 f(x)dx = \lim_{n\rightarrow\infty} \int_{\frac{1}{n}}^1 f(x)dx$$

To define integrability of continuous but unbounded functions over an opne set $S$, we seek to construct:

Consider the set $\mathcal D$ of all rectifiable, compact sets, where all elements $D$ satisfy

$$D \subseteq S$$

We take $\sup_{D \in \mathcal D} \int_D f$.

\end{document}
