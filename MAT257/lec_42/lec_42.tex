\documentclass[12pt]{article}
\usepackage{../../template}
\title{Lecture 42}
\author{niceguy}
\begin{document}
\maketitle

\section{Manifolds}

A $k$ dimensional manifold in $\R^k$ is a set $M^k \in \R^n$ with the property that $\forall p \in M^k \exists \alpha: U \rightarrow \R^n$ wwith $\alpha(U)  = V \subseteq M^k$ with open $U \subseteq \R^k, V \subseteq M^k$ satisfying

\begin{enumerate}
    \item $\alpha \in C^r$
    \item $\alpha$ is one-to-one
    \item $\alpha^{-1}$ is continuous
    \item The rank of $D\alpha$ is $k$
\end{enumerate}

We call $\alpha$ a coordinate chart.

\begin{ex}
    Consider the map $\pi$ from the surface of a sphere to the $\R^2$ plane by means of a line connecting the north pole to said point and finding its intersection with the $\R^2$ plane. This is \textit{not} a coordinate chart, but $\pi^{-1}$ is. Then $\mathbb{S}^2$ is a manifold without boundary.
\end{ex}

\textit{Note: If you think $\alpha$ preserves distance, you need to seek medical help.}

\begin{ex}[Ellipsoid]
    Is $\{\frac{x^2}{a^2} + \frac{y^2}{b^2} + \frac{z^2}{c^2} = 1\} \subset \R^3$ an open manifold? It is, if you use tangent planes.
\end{ex}

\begin{ex}[Parabaloid]
    Is $\{z = 5x^2 + 7y^2\} \subset \R^3$ a manifold? It is. Define
    $$\alpha(x,y) = (x,y,5x^2 + 7y^2)$$
    It is obvious that $\alpha$ is continuous. Its inverse is also continuous (preimage of open set is open). Finally,
    $$D\alpha = \begin{pmatrix} 1 & 0 \\ 0 & 1 \\ 10x & 14y \end{pmatrix}$$
    which has rank 2, since its (only) two columns are linearly independent.
\end{ex}

Consider any $C^r$ function $f: U \rightarrow \R$, where $U \subseteq \R^k$ is open. Similar to the example above, the graph of $f$ is a $k$ dimensional manifold in $\R^{k+1}$. \\

Now we see why every condition is needed. The rank being $k$ means that there are $k$ "directions" locally, and that they are all smooth. If not, the image of $U \subseteq \R^k$ under $\alpha$ has less than $k$ dimensions, so it doesn't make sense to call it a $k$ dimensional manifold. Without $C^r$, a cone can be a manifold, which is weird because of its vertex.
\end{document}
