\documentclass[12pt]{article}
\usepackage{../../template}
\title{Lecture 63}
\author{niceguy}
\begin{document}
\maketitle

\section{Dual Transformations}

Recall for $T:V\rightarrow W, b \in \Omega^l(W)$, we define

$$T^*b(v_1,\dots,v_l) = b(Tv_1,\dots,Tv_l)$$

Similarly,
$$\alpha_*(x;v) = (\alpha(x),D\alpha\cdot v)$$

This gives the general transformation

$$\alpha^*\omega(x;v_1,\dots,v_l) = \omega(\alpha(x);\alpha_*(x,v_1),\dots,\alpha_*(x,v_l))$$

This preserves linear and wedge structure, as one can easily verify. Now we want to show that this commutes with $d$. By linearity, we need only show this for elementary $dx_i$ or $dx_I$. For 1-forms,

\begin{align*}
    \alpha^*(dx_i)(x;v) &= dx_i(x)(\alpha_*(x;v)) \\
                        &= D\alpha(x)_i\cdot v \\
                        &= \sum_j D_j\alpha_i(x)v_j \\
                        &= \sum_j \del{\alpha_i}{x_j}dx_j(v) \\
    \alpha^*(dx_i) &= \sum_j \del{\alpha_i}{x_j}dx_j \\
                   &= d\alpha_i
\end{align*}
\end{document}
