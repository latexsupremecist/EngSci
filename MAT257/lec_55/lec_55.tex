\documentclass[12pt]{article}
\usepackage{../../template}
\title{Lecture 55}
\author{niceguy}
\begin{document}
\maketitle

\section{Tensor Fields}

First we define tensor fields on $\R^n$. Recall $\forall p \in \R^n \exists$ a tangent space of $\R^n$ based at $p$, i.e. $T_p\R^n$. The tensor field is the tensor $T(p)$, which takes $k$ vectors and outputs a real. In fact, $T(p) \in \mathcal L^k(T_p\R^n)$.

\begin{ex}[1-tensor field over $\R^n$]
    Consider any $C^1$ function $f:\R^n\rightarrow\R$. We will see that $Df$ is a 1-tensor field. We know that $Df$ is a function that maps from points $x \in \R^n$ into the span of $1 \times n$ row vectors, which is isomorphic with $\R^n$. Then
    $$T(p;\vec v) = D_{\vec v}f(p) = \langle Df(p), \vec v\rangle$$
    Now the input of $T$ can be thought of an element of the tangent space, and we know that it is linear in $\vec v$, since $Df(p)$ is a constant matrix, which is linear by definition.
\end{ex}

The above example is vacuously antisymmetric, since there is only one entry. We call that a one-form.

\begin{ex}[One forms]
    $$T(\vec v, \vec w) = \langle \vec v, \vec w \rangle$$
\end{ex}

\end{document}
