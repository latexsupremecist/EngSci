\documentclass[12pt]{article}
\usepackage{../../template}
\title{Lecture 15}
\author{niceguy}
\begin{document}
\maketitle

\section{Inverse Function Theorem}

We finish the proof of the theorem today. So far, we have reduced it to

\begin{thm}
    If $f: A \subseteq \R^n \rightarrow B \subseteq \R^n$ is one to one, with $A$ and $B$ open, then $[Df(x)]$ being invertible means $g=f^{-1}$ is also $\mathcal C^r$.
\end{thm}

Today we show that $g^{-1}$ is differentiable. By chain rule, this implies

$$[Dg(b)] = [Df(a)]^{-1}, f(a) = b$$

which prove the theorem. We start by considering the linear map $E = Df(a)$. We want to show that

$$\lim_{|k|\rightarrow0} \frac{|g(b+k) - g(b) - E^{-1}k|}{|k|} = 0$$

We have established that $g$ is continuous; hence $g(b+k) - g(b) \rightarrow 0$ as $|k| \rightarrow 0$.

$$\frac{g(b+k) - g(b)}{|k|} \leq M \Leftrightarrow \frac{g(b+k) - g(b)}{|b+k-b|} \leq M \Leftrightarrow \frac{|h|}{|f(a+h) - f(a)|} \leq M \Leftrightarrow \frac{|f(a+h)-f(a)|}{|h|} \geq M^{-1}$$

We have proven the rightmost statement, so the leftmost statement holds. Then

$$\frac{(g(b+k) - g(b) - E^{-1}(k)}{|k|} = E^{-1} \frac{Eh - k}{|g(b+k) - g(b)|} \frac{|g(b+k) - g(b)|}{|k|} = E^{-1} \frac{Eh - f(a+h) + f(a)}{|h|} \frac{g(b+k) - g(b)}{|k|}$$

For the rightmost term, the first term is constant, the second vanishes, and the third is bounded above. Hence it tends to zero, and $g$ is differentiable. If $f$ is $\mathcal C^r$, then each entry in $Df$ has to be $\mathcal C^{r-1}$. Hence each entry in $Dg = Df^{-1}$ is $\mathcal C^{r-1}$, and $g$ is $\mathcal C^r$.

\end{document}
