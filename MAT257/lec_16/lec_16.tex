\documentclass[12pt]{article}
\usepackage{../../template}
\title{Lecture 16}
\author{niceguy}
\begin{document}
\maketitle

\section{Some Linear Algebra...}

From linear algebra, recall that if there are $n+k$ unknowns and $n$ equations, the solution space has to have a "dimension" of $k$. We use quotes because the solution space need not be a vector subspace. It doesn't have to be closed. Note that if we have a \textit{fixed} solution

$$Ax^* = b$$

then we see that the solution forms an affine subspace. Alternatively, all solutions $x$ are such that $x - x^*$ is a \textbf{vector subspace} with dimension $k$.

\section{Back to Implicit Function Theorem}

We can treat $f:\R^{n+k}\rightarrow \R^n$ as a system of $n$ equations with $n+k$ unknowns. If we have a solution $(x^*, w^*$, where $x^* \in \R^{n+k}, w^* = \R^n$, then consider $Df(x^*)$, which has dimensions $n \times (n+k)$. If it has rank of $n$, then we expect that for every fixed $y$, the solution space is an affine subset of dimension $k$. If we fix the first $k$ entries in fact, and approximate the function with its derivative at a point, then we have a mapping from $\R^n$ to $\R^n$.

\end{document}
