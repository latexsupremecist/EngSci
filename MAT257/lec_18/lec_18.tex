\documentclass[12pt]{article}
\usepackage{../../template}
\title{Lecture 18}
\author{niceguy}
\begin{document}
\maketitle

\section{Integration}

For now, we only consider functions $f:R \in \R^n \rightarrow\R$ where $R$ is a rectangle. It is complicated to integrate over other domains, even if is a simple disk.

\subsection{Formulation}

Recall how we defined integrals. We can define a partition $P_n = \{a = a_0, a_1,a_2,\dots,a_n = b\}$ where $a_i > a_j$ if $i > j$. Then we can define the Riemann sum

$$\mathcal R_{P_n}[f] = \sum_{i=1}^n f(a_i)(a_{i+1}-a_i)$$

Taking the limit as $n\rightarrow\infty$ gives us the integral. For multi-variable calculus, we need to first define integrability and the integral itself. First we consider a bounded function $f:I_1 \times I_2 \times \dots \times I_ \rightarrow \R$. A partition of $R$ is the union of partitions of each $I_n$. We can then define the lower sum

$$L(f,P) = \sum_{R \in P} \inf_R(f)v(R)$$

where $R$ is a rectangle in the partition, and $v(R)$ is its volume (area). We similarly define the upper sum using the supremum. Note that the infimum and supremum exist, since $f$ is bounded.

\begin{lem}
    Take $R$. Take $P, P'$ to be 2 partitions of $R$. Then $P'' = P \cup P'$ is a partition and refinement of $P$.
\end{lem}

\begin{lem}
    Given $P'$ is a refinement of $P$, we have
    $$L(f,P) \leq L(f,P') \leq U(f,P') \leq U(f,P)$$
\end{lem}

\begin{proof}
    First we prove the first inequality. We do this by induction. Let $|P'| = |P| + 1$. Then $a_q$, between $a_k$ and $a_{k+1}$. Then there is a bijection between each $R\in P$ and $R \in P'$, except for $R_k = [a_k,a_q] \times S$ and $R_{k+1} = [a_q,a_{k+1}] \times S$. Note that both of these are subsets of $R' = [a_k,a_{k+1}] \times S$. Since $R'$ is the union of the sets $R_k$ and $R_{k+1}$, the infimum of $R_k$ and $R_{k+1}$ cannot be less than that of $R'$, else that (or something lower) will be the infimum. Without loss of generality, let $m_k$ be the infimum of $R_k$, and $m$ be the infimum of $R'$. Since $m_k$ is an infimum, anything greater than it is not a lower bound of $R_k$, hence it is not a lower bound of $R$. An infimum itself is a lower bound, so $m$ is a lower bound, hence $m$ cannot be greater than $m_k$, same for $m_{k+1}$. Then the sums for the "new" rectangles will be $m_kv(R_k) + m_{k+1}v(R_{k+1}) \geq mv(R_k) + mv(R_{k+1}) = mv(R')$. This proves the first inequality. Then we can simply perform induction. \\
    The second inequality is obvious, since the supremum cannot be less than the infimum. The third inequality is the same as the second; it can be proven by swapping infimum with supremum.
\end{proof}

\end{document}
