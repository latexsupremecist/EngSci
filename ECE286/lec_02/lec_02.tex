\documentclass[12pt]{article}
\usepackage{../../template}
\author{niceguy}
\title{Lecture 2}
\begin{document}
\maketitle

\section{Counting}

\begin{ex}
	\begin{itemize}
		\item Coin flip: $\{H\}$ has 1 element
		\item Die toss: {even} has 3 elements
		\item 2 dice: $6 \times 6 = 36$
	\end{itemize}
\end{ex}

\begin{ex}
	We can multiply options. For 3 appetizers, 4 mains, and 2 desserts, we have $3\times4\times2=24$ choices. If not ordering is also a choice, we have $4\times5\times3=60$ choices.
\end{ex}

\begin{ex}
	To choose a president and vice president out of $n$ people, we have $n(n-1)$ choices.
\end{ex}

\begin{ex}
	How many 4 digit numbers can we make from $\{0,1,2,5,6,9\}$? \\
	$$5\times4\times3 + 2\times4\times4\times3 = 156$$
\end{ex}

\section{Permutations}

Where order matters.

\begin{ex}
	Given $n$ items, there are $n!$ permutations.
\end{ex}

We may also pick $r$ out of $n$ items.

\begin{ex}
	2 letters out of {a,b,c,d,e,f} gives $6 \times 5 = 30$ permutations.
\end{ex}

The general formula for the number of permutations is

$$\frac{n!}{(n-r)!}$$

\begin{ex}
	How many 5 card hands are there from a normal deck?
	$$\frac{52!}{47!} = 311875200$$
\end{ex}

\begin{ex}
	Permutations with identical items: {a,b,b} \\
	There are $\frac{3!}{2!} = 3$ permutations.
\end{ex}

\section{General Formula}
For $n$ items, $m$ types each with $n_m$ entries, the total number of permutations is
$$\frac{n!}{\prod_{i=1}^m n_i!}$$

\begin{ex}
	How many distinct reorderings of ATLANTIC there are?
	$$\frac{8!}{2!2!} = 10080$$
\end{ex}

\begin{ex}
	Flip 10 coins in a row. How many sequences have 4 heads?
	$$\frac{10!}{4!6!} = 210$$
	The probability is then
	$$\frac{210}{2^{10}} \approx \frac{1}{5}$$
\end{ex}


\end{document}
