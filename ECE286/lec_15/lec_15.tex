\documentclass[12pt]{article}
\usepackage{../../template}
\author{niceguy}
\title{Lecture 15}
\begin{document}
\maketitle

\section{Normal Distribution}

\begin{ex}
	$X$ is normal with $n(x;5,2)$. Find $P(-1 \leq X \leq 4)$. \\
	Using the cumulative distribution function
	$$\Phi(x) = \int_{-\infty}^x n(t;0,1)dt$$
	Set $z = \frac{X-5}{2}$. Then $Z$ has $n(z;0,1)$, so
	$$P(-1 \leq X \leq 4) = P\left(-3 \leq Z \leq -\frac{1}{2}\right) = \Phi\left(-\frac{1}{2}\right) - \Phi(-3) = 0.3072$$
\end{ex}

We know that the binomial distribution converges to the Poisson distribution when $n\rightarrow\infty,p\rightarrow0,np=\lambda$. \\
If we have the mean $\mu = np$ and variance $\sigma^2 = np(1-p)$.  Letting
$$Z = \frac{X-np}{\sqrt{np(1-p)}}$$
Then as $n\rightarrow\infty$, then the distribution of $Z$ approaches $n(z;0,1)$.

\section{Gamma Distribution}

\begin{defn}
	The Gamma function is defined as
	$$\Gamma(\alpha) = \int_0^\infty x^{\alpha-1}e^{-x}dx, \alpha \geq 0$$
	Fun facts:
	\begin{itemize}
		\item $\Gamma\left(\frac{1}{2}\right) = \sqrt{\pi}$
		\item $\Gamma(n) = (n-1)!$ for $n \in \Z^+$
	\end{itemize}
\end{defn}

The Gamma Distribution is
$$F(x;\alpha,\beta) = \begin{cases} \frac{1}{\beta^\alpha\Gamma(\alpha)}x^{\alpha-1}e^{-\frac{x}{\beta}} & x\geq0 \\ 0 & x < 0 \end{cases}$$
The mean is $\mu = \alpha\beta$ and the variance is $\sigma^2=\alpha\beta^2$.

\subsection{Chi-Squared Distribution}

With a parameter $v \in \Z^+$,
$$f(x;v) = \begin{cases} \frac{1}{2^{\frac{v}{2}}\Gamma\left(\frac{v}{2}\right)}x^{\frac{v}{2}-1}e^{-\frac{x}{2}} & x\geq0 \\ 0 & x<0 \end{cases}$$

\subsection{Exponential Distribution}

$$f(x;\beta) = \begin{cases} \frac{1}{\beta}e^{-\frac{x}{\beta}} & x\leq0 \\ 0 & x<0 \end{cases}$$
Which is the Gamma distribution with $\alpha = 1$, so its mean is $\mu=\beta$ and its variance is $\sigma^2=\beta^2$.


\end{document}
