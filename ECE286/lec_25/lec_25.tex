\documentclass[12pt]{article}
\usepackage{../../template}
\author{niceguy}
\title{Lecture 25}
\begin{document}
\maketitle

\section{Tolerance Limits}
For a normal IID sample, with sample mean
$$\overline{x} = \frac{1}{n}\sum_ix_i$$
and sample variance
$$s^2 = \frac{1}{n-1} \sum_i (x_i-\overline{x})^2$$

Then the tolerance limits are in the form
$$\overline{x} \pm ks$$
We choose $k$ so that $(1-\alpha)$ of the population is within $[\overline{x}-ks,\overline{x}+ks]$. This does not shrink with $n$
$$\lim_{n\rightarrow\infty}P(\overline{X}-kS \leq X \leq \overilne{X}+kS) = 1 - \alpha$$

\begin{ex}[Heights]
	The confidence interval gets narrower as $n \rightarrow \infty$, but the tolerance limits don't really go anywhere.
\end{ex}

\section{Two Samples}

Suppose with have two samples with $n_1,n_2$, means $\mu_1,\mu_2$, variances $\sigma_1^2,\sigma_2^2$. Then we want to estimate $\mu_1-\mu_2$. Then consider the statistic $\overline{X_1}-\overline{X_2}$. By the Central Limit Theorem, both are approximately normal, hence

$$Z = \frac{\overline{X}_1 - \overline{X}_2 - (\mu_1 + \mu_2)}{\sqrt{\sigma_1^2/n_1 + \sigma_2^2 / n_2}}$$
has a normal disstribution. This can be used to make a confidence interval for $\mu_1-\mu_2$.

\begin{ex}[Swimmers]
	For $n_1=40,n_2=43, \overline{x_1}=60,\overline{x_2}=58, s_1^2=1,s_2^2=2$. Then since $n>30$ in both cases, we can use the Central Limit Theorem, taking the sample variance as true variance,
	$$z = \frac{2-(\mu_1+\mu_2)}{0.27}$$
	Now
	\begin{align*}
		0.95 &= P(-z_{0.025} \leq Z \leq z_{0.025}) \\
		     &= P(\overline{X_1} - \overline{X_2} - 0.27z_{0.0025} \leq \mu_1 - \mu_2 \leq \overline{X_1} - \overline{X_2} + 0.27z_{0.0025}) \\
		     &= P(1.47 \leq \mu_1 - \mu_2 \leq 2.53)
	\end{align*}
\end{ex}

For some sample $X_1,\dots,X_n$, and a statistic $W$ with a probability density function $f(w)$, we define
$$w_{\frac{\alpha}{2}} = -F^{-1}\left(\frac{\alpha}{2}\right)$$

\section{Two Samples with unknown Variance}
If $n_1,n_2 < 30$ with unknown $\sigma_1=\sigma_2$. We use a pooled estimate of variance
$$s_p^2 = \frac{(n_1-1)s_1^2+(n_2-1)s_2^2}{n_1+n_2-2}$$
Then for the statistics
$$T = \frac{\overline{X_1}-\overline{X_2}-(\mu_1-\mu_2)}{s_p\sqrt{\frac{1}{n_1}+\frac{1}{n_2}}}$$
Then $T$ has $t$ distribution, and we can make a confidence interval as usual.
\end{document}
