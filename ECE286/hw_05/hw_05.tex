\documentclass[answers]{exam}
\usepackage{../../template}
\author{niceguy}
\title{Homework 5}
\begin{document}
\maketitle

\begin{questions}

\question{Given a continuous uniform distribution, show that}

\begin{parts}
	\part{$\mu = \frac{A+B}{2}$}
	\part{$\sigma^2 = \frac{(B-A)^2}{12}$}
\end{parts}

\begin{solution}
	The mean is
	\begin{align*}
		\mu &= \int_{-\infty}^\infty xf(x)dx \\
		    &= \int_A^B \frac{x}{B-A} dx \\
		    &= \frac{B^2-A^2}{2(B-A)} \\
		    &= \frac{A+B}{2}
	\end{align*}
	The variance is
	\begin{align*}
		\sigma^2 &= \int_{-\infty}^\infty (x-\mu)^2f(x)dx \\
			 &= \int_A^B \frac{x^2 - (A+B)x + \frac{(A+B)^2}{4}}{B-A}dx \\
			 &= \frac{B^3-A^3}{3(B-A)} - \frac{(A+B)(B^2-A^2)}{2(B-A)} + \frac{(A+B)^2(B-A)}{4(B-A)} \\
			 &= \frac{A^2+AB+B^2}{3} - \frac{(A+B)^2}{2} + \frac{(A+B)^2}{4} \\
			 &= \frac{4A^2+4AB+4B^2-6A^2-12AB-6B^2+3A^2+6AB+3B^2}{12} \\
			 &= \frac{A^2-2AB+B^2}{12} \\
			 &= \frac{(B-A)^2}{12}
	\end{align*}
\end{solution}

\question{A bus arrives every 10 minutes at a bus stop. It is assumed that the waiting time for a particular individual is a random variable with a continuous uniform distribution.}

\begin{parts}
	\part{What is the probability that the individual waits more than 7 minutes?}
	\part{What is the probability that the individual waits between 2 and 7 minutes?}
\end{parts}

\begin{solution}
	$$P[X>7] = \frac{10-7}{10} = 0.3$$
	$$P[2\leq X\leq 7] = \frac{7-2}{10} = 0.5$$
\end{solution}

\question{Given a standard normal distribution, find the value of $k$ such that}

\begin{parts}
	\part{$P(Z>k) = 0.2946$}
	\part{$P(Z<k) = 0.0427$}
	\part{$P(-0.93<Z<k) = 0.7235$}
\end{parts}

\begin{solution}
	$$0.54,-1.72,1.28$$
\end{solution}

\question{The loaves of rye bread distributed to local stores by a certain bakery have an average length of 30 centimeters and a standard deviation of 2 centimeters. Assuming that the lengths are normally distributed, what percentage of the loaves are}

\begin{parts}
	\part{longer than 31.7 centimeters?}
	\part{between 29.3 and 33.5 centimeters in length?}
	\part{shorter than 25.5 centimeters?}
\end{parts}

\begin{solution}
	$$P(X>31.7) = P(Z>0.85) = 19.8\%$$
	$$P(29.3<X<33.5) = P(-0.35<Z<1.75) = 59.7\%$$
	$$P(X<25.5) = P(Z<-2.25) = 1.22\%$$
\end{solution}

\question{f a set of observations is normally distributed,
what percent of these differ from the mean by}

\begin{parts}
	\part{more than 1.3$\sigma$?}
	\part{less than 0.52$\sigma$?}
\end{parts}

\begin{solution}
	$$P(|Z|>1.3) = 19.4\%$$
	$$P(|Z|<0.52) = 39.7\%$$
\end{solution}

\question{A coin is tossed 400 times. Use the normal curve
approximation to find the probability of obtaining}

\begin{parts}
	\part{between 185 and 210 heads inclusive}
	\part{exactly 205 heads}
	\part{fewer than 176 or more than 227 heads}
\end{parts}

\begin{solution}
	$$Z = \frac{X-np}{\sqrt{np(1-p)}} = \frac{X-200}{10}$$
	Then
	$$P(184.5\leq X\leq210.5) = P(-1.55\leq X\leq 1.05) = 0.793$$
	$$P(204.5<X<205.5) = P(0.45<X<0.55) = 0.0352$$
	$$P(X<175.5) + P(X>227.5) = P(Z<-2.45) + P(Z>2.75) = 0.0101$$
\end{solution}

\question{A pair of dice is rolled 180 times. What is the
probability that a total of 7 occurs}

\begin{parts}
	\part{at least 25 times?}
	\part{between 33 and 41 times inclusive?}
	\part{exactly 30 times?}
\end{parts}

\begin{solution}
	The probability that a total of 7 occurs is $\frac{1}{6}$. Thus
	$$Z = \frac{X-np}{\sqrt{np(1-p)}} = \frac{X-30}{5}$$
	Then
	$$P(X>24.5) = P(Z>-1.1) = 0.864$$
	$$P(32.5<X<41.5) = P(0.5<Z<2.3) = 0.298$$
	$$P(29.5<X<30.5) = P(-0.1<Z<0.1) = 0.0797$$
\end{solution}

\question{Use the gamma function with $y=\sqrt{2x}$ to show that $\Gamma\left(\frac{1}{2}\right)=\sqrt{\pi}$.}

\begin{solution}
\begin{align*}
		\Gamma\left(\frac{1}{2}\right) &= \int_0^\infty x^{\frac{1}{2}-1}e^{-x}dx \\
					       &= \int_0^\infty \frac{e^{-x}}{\sqrt{x}}dx \\
					       &= \sqrt{2} \int_0^\infty e^{-\frac{y^2}{2}}dy \\
					       &= \sqrt{2} \sqrt{\frac{\pi}{2}} \\
					       &= \sqrt{\pi}
\end{align*}
\end{solution}

\question{Suppose that the service life, in years, of a hearing aid battery is a random variable having a Weibull distribution with $\alpha = \frac{1}{2}$ and $\beta = 2$.}

\begin{parts}
	\part{How long can such a battery be expected to last?}
	\part{What is the probability that such a battery will be operating after 2 years?}
\end{parts}

\begin{solution}
	The mean is
	$$\mu = \alpha^{-\frac{1}{\beta}}\Gamma\left(1+\frac{1}{\beta}\right) = \sqrt{2}\Gamma(1.5) = \sqrt{2}\int_0^\infty \sqrt{x}e^{-x}dx = \frac{1}{\sqrt{2}}\int_0^\infty \frac{e^{-x}}{\sqrt{x}}dx = \sqrt{\frac{\pi}{2}}$$
	The probability it operates after 2 years is
	\begin{align*}
		P(X>2) &= \int_2^\infty \alpha\beta x^{\beta-1}e^{-\alpha x^\beta} dx \\
		       &= \int_2^\infty xe^{-\frac{x^2}{2}}dx \\
		       &= e^{-2}
	\end{align*}
\end{solution}

\question{Let $X$ be a random variable with probability
	$$f(x) = \begin{cases} \frac{1}{3} & x=1,2,3 \\ 0 & \text{elsewhere} \end{cases}$$
Find the probability distribution of the random variable $Y=2X-1$.}

\begin{solution}
	$$g(y) = \begin{cases} \frac{1}{3} & y=1,3,5 \\ 0 & \text{elsewhere} \end{cases}$$
\end{solution}

\question{Let $X_1$ and $X_2$ be discrete random variables with joint probability distribution
	$$f(x_1,x_2) = \begin{cases} \frac{x_1x_2}{18} & x_1=1,2;x_2=1,2,3 \\ 0 & \text{elsewhere} \end{cases}$$
Find the probability distribution of the random variable $Y=X_1X_2$.}

\begin{solution}
	$$g(y) = \begin{cases} \frac{1}{18} & y=1 \\ \frac{2}{9} & y=2,y=4 \\ \frac{1}{6} & y=3 \\ \frac{1}{3} & y=6 \end{cases}$$
\end{solution}

\question{Let $X$ be a random variable with probability distribution
	$$f(x) = \begin{cases} \frac{1+x}{2} & -1 < x < 1 \\ 0 & \text{elsewhere} \end{cases}$$
	Find the probability distribution of the random variable $Y=X^2$.
}

\begin{solution}
	\begin{align*}
		G(y) &= P(Y\leq y) \\
		     &= P(-\sqrt{y}\leq X\leq \sqrt{y}) \\
		     &= \int_{-\sqrt{y}}^{\sqrt{y}} f(t)dt
	\end{align*}
	Then
	\begin{align*}
		g(y) &= \frac{d}{dy}G(y) \\
		     &= f(\sqrt{y})\times\frac{1}{2\sqrt{y}} - f(-\sqrt{y})\times\frac{-1}{2\sqrt{y}} \\
		     &= \frac{1+\sqrt{y}}{2}\times\frac{1}{2\sqrt{y}} + \frac{1-\sqrt{y}}{2}\times\frac{1}{2\sqrt{y}} \\
		     &= \frac{1}{2\sqrt{y}}
	\end{align*}
\end{solution}
\end{questions}
\end{document}
