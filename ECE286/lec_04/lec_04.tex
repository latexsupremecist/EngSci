\documentclass[12pt]{article}
\usepackage{../../template}
\author{niceguy}
\title{Lecture 4}
\begin{document}
\maketitle

\section{Recall}

$$P(A)\in[0,1]\forall A\subseteq S$$

\begin{ex}
	Consider throwing darts. The probability of landing in region $A$, given a random throw, is
	$$P(A) = \frac{\text{Area of }A}{\text{Area of }S} = \frac{\int_Adx}{\int_Sdx}$$
\end{ex}

\section{Additive Rule}

$$P(A\cup B) = P(A) + P(B) - P(A\cap B)$$

\begin{ex}
	Consider 2 dice. The probability of rolling 7 ($A$) or at least one two ($B$) is
	$$P(A) + P(B) - P(A\cap B) = \frac{1}{6} + \frac{11}{36} - \frac{1}{18} = \frac{5}{12}$$
\end{ex}

Expanding, we have
\begin{align*}
	P(A\cup B\cup C) &= P(A) + P(B\cup C) - P(A\cap(B\cup C)) \\
			 &= P(A) + P(B) + P(C) - P(B\cap C) - P((A\cap B)\cup(A\cap C)) \\
			 &= P(A) + P(B) + P(C) - P(B\cap C) - P(A\cap B) - P(A\cap C) + P(A\cap B\cap C)
\end{align*}

\section{Conditional Probability}

For $A,B\subseteq S$, $P(B|A)$ is the probability of $B$ given $A$ occurred.

\begin{ex}
	Rolling two dice
	\begin{itemize}
		\item $P(7) = \frac{1}{6}$
		\item $P(7|\text{first roll is a 2}) = \frac{1}{6}$
	\end{itemize}
\end{ex}

\begin{defn}
	$$P(B|A) = \frac{P(A\cup B)}{P(A)}, P(A) > 0$$
\end{defn}

Consider being a pro athlete. The probability of that is $10^{-4}$. Given the probability of beinga pro who starts at 3 is $9\times10^{-5}$, and the probability of starting at 3 is $0.01$. Then the probability of being a pro given one starts at 3 is then

$$\frac{9\times10^{-5}}{0.01} = 0.009$$

\end{document}
