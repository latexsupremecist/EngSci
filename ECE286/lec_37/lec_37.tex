\documentclass[12pt]{article}
\usepackage{../../template}
\author{niceguy}
\title{Lecture 37 (Review Lecture)}
\begin{document}
\maketitle

\section{Central Limit Theorem}

We have a sample with data $x_1,\dots,x_n$, which are \textbf{actual numbers}. There are also random variables $X_1,\dots,X_n$, whose realisations are the data. They are independent and identically distributed. So they have the same (unknown) distribution.

\section{Mean}

The numeric value is
$$\overline x = \frac{1}{n} \sum_{i=1}^n x_i$$
The random variable is
$$\overline X = \frac{1}{n} \sum_{i=1}^n X_i$$
The true mean is $\mu$, and the true standard deviation is $\sigma$. Then let
$$Z = \frac{\overline X - \mu}{\sigma/\sqrt{n}}$$
The Central Limit Theorem states that as $n\rightarrow\infty$, the distribution of $Z$ approaches the normal distribution.

\section{Confidence Interval}

Letting $Z$ be a normal distribution, we find a range where there is a probability of $x\%$ that the true mean lies in said region. Then defining
$$z_{\frac{\alpha}{2}} = -\Phi^{-1}\left(\frac{\alpha}{2}\right)$$
we can rearrange the terms to get
\begin{align*}
    1 - \alpha &= P\left(-z_{\frac{\alpha}{2}} \leq Z \leq z_{\frac{\alpha}{2}}\right) \\
               &= P\left(\overline X - z_{\frac{\alpha}{2}} \frac{\sigma}{\sqrt{n}} \leq \mu \leq \overline X + z_{\frac{\alpha}{2}}\right)
\end{align*}
The confidence interval is then
$$\left(\overline X - z_{\frac{\alpha}{2}}, \overline X + z_{\frac{\alpha}{2}}\right)$$

\section{T-Distribution}

Given a \textbf{normal} population, with an unknown $\sigma$, which is estimated from the sample, then it is the \textbf{exact distribution}. The same procedure (using a $t$ instead of a $z$) follows. We estimate the standard deviation by
$$s^2 = \frac{1}{n-1}\sum_{i=1}^n (x_i-\overline x)^2$$

Now if $n < 30$, and we do not know the variance, then we have no good solution.

\section{$\chi^2$ Distribution}

The $\chi^2$ distribution measures the distribution of the sample variance, given a normal population. 

\section{Bayes}

$$P(A|B) = \frac{P(A\cap B)}{P(B)}$$

\end{document}
