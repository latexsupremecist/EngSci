You may require accessibility services. If you do, \href{https://studentlife.utoronto.ca/service/accessibility-services-registration-and-documentation-requirements/}{register} by 17:00 on Friday, July 14, 2023 (check the website if it's 2024+). If you are unsure whether you are eligible, email accessibility services to check. Don't do this last minute, because they receive high volume emails and take forever to reply. In general, always register if you are hesitating, because you don't have to use your accommodations if you get them (e.g. volunteering notes, extra writing time, extensions, etc.), but getting registered during the term is time-consuming and frustrating.

\section{Example}

The person who wrote this is registered with them for migraines, anxiety/depression, and so on. Now, not everyone with migraines are registered with accessibility services, since their migraines do not prevent them from completing daily tasks. However, since they affect my vision, I am provided with accommodations.

\section{Reasons for Accommodations}

You can receive accommodations even with undiagnosed or suspected disabilities.

\section{I feel like I don't really need it...}

You shouldn't stop yourself from receiving accommodations just because you think you are taking advantage. You do the same work as everyone else by the time you graduate, so I don't really see how that is the case. Furthermore, if accessibility services think your conditions don't require accommodations, they won't provide them. So don't worry about taking advantage of the system.
