% Options for packages loaded elsewhere
\PassOptionsToPackage{unicode}{hyperref}
\PassOptionsToPackage{hyphens}{url}
%
\documentclass[
]{article}
\usepackage{amsmath,amssymb}
\usepackage{iftex}
\ifPDFTeX
  \usepackage[T1]{fontenc}
  \usepackage[utf8]{inputenc}
  \usepackage{textcomp} % provide euro and other symbols
\else % if luatex or xetex
  \usepackage{unicode-math} % this also loads fontspec
  \defaultfontfeatures{Scale=MatchLowercase}
  \defaultfontfeatures[\rmfamily]{Ligatures=TeX,Scale=1}
\fi
\usepackage{lmodern}
\ifPDFTeX\else
  % xetex/luatex font selection
\fi
% Use upquote if available, for straight quotes in verbatim environments
\IfFileExists{upquote.sty}{\usepackage{upquote}}{}
\IfFileExists{microtype.sty}{% use microtype if available
  \usepackage[]{microtype}
  \UseMicrotypeSet[protrusion]{basicmath} % disable protrusion for tt fonts
}{}
\makeatletter
\@ifundefined{KOMAClassName}{% if non-KOMA class
  \IfFileExists{parskip.sty}{%
    \usepackage{parskip}
  }{% else
    \setlength{\parindent}{0pt}
    \setlength{\parskip}{6pt plus 2pt minus 1pt}}
}{% if KOMA class
  \KOMAoptions{parskip=half}}
\makeatother
\usepackage{xcolor}
\usepackage[margin=1in]{geometry}
\usepackage{color}
\usepackage{fancyvrb}
\newcommand{\VerbBar}{|}
\newcommand{\VERB}{\Verb[commandchars=\\\{\}]}
\DefineVerbatimEnvironment{Highlighting}{Verbatim}{commandchars=\\\{\}}
% Add ',fontsize=\small' for more characters per line
\usepackage{framed}
\definecolor{shadecolor}{RGB}{248,248,248}
\newenvironment{Shaded}{\begin{snugshade}}{\end{snugshade}}
\newcommand{\AlertTok}[1]{\textcolor[rgb]{0.94,0.16,0.16}{#1}}
\newcommand{\AnnotationTok}[1]{\textcolor[rgb]{0.56,0.35,0.01}{\textbf{\textit{#1}}}}
\newcommand{\AttributeTok}[1]{\textcolor[rgb]{0.13,0.29,0.53}{#1}}
\newcommand{\BaseNTok}[1]{\textcolor[rgb]{0.00,0.00,0.81}{#1}}
\newcommand{\BuiltInTok}[1]{#1}
\newcommand{\CharTok}[1]{\textcolor[rgb]{0.31,0.60,0.02}{#1}}
\newcommand{\CommentTok}[1]{\textcolor[rgb]{0.56,0.35,0.01}{\textit{#1}}}
\newcommand{\CommentVarTok}[1]{\textcolor[rgb]{0.56,0.35,0.01}{\textbf{\textit{#1}}}}
\newcommand{\ConstantTok}[1]{\textcolor[rgb]{0.56,0.35,0.01}{#1}}
\newcommand{\ControlFlowTok}[1]{\textcolor[rgb]{0.13,0.29,0.53}{\textbf{#1}}}
\newcommand{\DataTypeTok}[1]{\textcolor[rgb]{0.13,0.29,0.53}{#1}}
\newcommand{\DecValTok}[1]{\textcolor[rgb]{0.00,0.00,0.81}{#1}}
\newcommand{\DocumentationTok}[1]{\textcolor[rgb]{0.56,0.35,0.01}{\textbf{\textit{#1}}}}
\newcommand{\ErrorTok}[1]{\textcolor[rgb]{0.64,0.00,0.00}{\textbf{#1}}}
\newcommand{\ExtensionTok}[1]{#1}
\newcommand{\FloatTok}[1]{\textcolor[rgb]{0.00,0.00,0.81}{#1}}
\newcommand{\FunctionTok}[1]{\textcolor[rgb]{0.13,0.29,0.53}{\textbf{#1}}}
\newcommand{\ImportTok}[1]{#1}
\newcommand{\InformationTok}[1]{\textcolor[rgb]{0.56,0.35,0.01}{\textbf{\textit{#1}}}}
\newcommand{\KeywordTok}[1]{\textcolor[rgb]{0.13,0.29,0.53}{\textbf{#1}}}
\newcommand{\NormalTok}[1]{#1}
\newcommand{\OperatorTok}[1]{\textcolor[rgb]{0.81,0.36,0.00}{\textbf{#1}}}
\newcommand{\OtherTok}[1]{\textcolor[rgb]{0.56,0.35,0.01}{#1}}
\newcommand{\PreprocessorTok}[1]{\textcolor[rgb]{0.56,0.35,0.01}{\textit{#1}}}
\newcommand{\RegionMarkerTok}[1]{#1}
\newcommand{\SpecialCharTok}[1]{\textcolor[rgb]{0.81,0.36,0.00}{\textbf{#1}}}
\newcommand{\SpecialStringTok}[1]{\textcolor[rgb]{0.31,0.60,0.02}{#1}}
\newcommand{\StringTok}[1]{\textcolor[rgb]{0.31,0.60,0.02}{#1}}
\newcommand{\VariableTok}[1]{\textcolor[rgb]{0.00,0.00,0.00}{#1}}
\newcommand{\VerbatimStringTok}[1]{\textcolor[rgb]{0.31,0.60,0.02}{#1}}
\newcommand{\WarningTok}[1]{\textcolor[rgb]{0.56,0.35,0.01}{\textbf{\textit{#1}}}}
\usepackage{graphicx}
\makeatletter
\def\maxwidth{\ifdim\Gin@nat@width>\linewidth\linewidth\else\Gin@nat@width\fi}
\def\maxheight{\ifdim\Gin@nat@height>\textheight\textheight\else\Gin@nat@height\fi}
\makeatother
% Scale images if necessary, so that they will not overflow the page
% margins by default, and it is still possible to overwrite the defaults
% using explicit options in \includegraphics[width, height, ...]{}
\setkeys{Gin}{width=\maxwidth,height=\maxheight,keepaspectratio}
% Set default figure placement to htbp
\makeatletter
\def\fps@figure{htbp}
\makeatother
\setlength{\emergencystretch}{3em} % prevent overfull lines
\providecommand{\tightlist}{%
  \setlength{\itemsep}{0pt}\setlength{\parskip}{0pt}}
\setcounter{secnumdepth}{-\maxdimen} % remove section numbering
\ifLuaTeX
  \usepackage{selnolig}  % disable illegal ligatures
\fi
\IfFileExists{bookmark.sty}{\usepackage{bookmark}}{\usepackage{hyperref}}
\IfFileExists{xurl.sty}{\usepackage{xurl}}{} % add URL line breaks if available
\urlstyle{same}
\hypersetup{
  pdftitle={Lecture 1},
  pdfauthor={niceguy},
  hidelinks,
  pdfcreator={LaTeX via pandoc}}

\title{Lecture 1}
\author{niceguy}
\date{2023-07-26}

\begin{document}
\maketitle

\hypertarget{course-goals}{%
\subsection{Course Goals}\label{course-goals}}

\begin{itemize}
\tightlist
\item
  R, Functional Programming
\item
  Visualise, Process, and Analyse tabular data
\item
  Predictive Modeling
\end{itemize}

\hypertarget{introduction}{%
\subsection{Introduction}\label{introduction}}

I'm assuming you have previous coding experience, so I'll skip the basic
stuff.

\hypertarget{comments}{%
\subsubsection{Comments}\label{comments}}

start with \#

\begin{Shaded}
\begin{Highlighting}[]
\CommentTok{\# Hi, I\textquotesingle{}m a comment}
\end{Highlighting}
\end{Shaded}

\hypertarget{print}{%
\subsubsection{Print}\label{print}}

Use \texttt{cat} to print statements

\begin{Shaded}
\begin{Highlighting}[]
\FunctionTok{cat}\NormalTok{(}\StringTok{"Hello World!"}\NormalTok{) }\CommentTok{\# no semicolon at the end}
\end{Highlighting}
\end{Shaded}

\begin{verbatim}
## Hello World!
\end{verbatim}

\hypertarget{variables}{%
\subsubsection{Variables}\label{variables}}

\begin{Shaded}
\begin{Highlighting}[]
\CommentTok{\# We use \textless{}{-} (ALT {-} ) for assign, instead of =}

\NormalTok{engsci\_adj }\OtherTok{\textless{}{-}} \SpecialCharTok{{-}}\DecValTok{12}
\NormalTok{x }\OtherTok{\textless{}{-}} \DecValTok{97} \SpecialCharTok{+}\NormalTok{ engsci\_adj}
\CommentTok{\# These are displayed in the Environment}
\end{Highlighting}
\end{Shaded}

To run a line, use CTRL ENTER, to run a code chunk, use CTRL SHIFT
ENTER.

\hypertarget{functions}{%
\subsubsection{Functions}\label{functions}}

Let's say we want to see if

\[ax^2+bx+c=0\]

has real roots or not.

\begin{Shaded}
\begin{Highlighting}[]
\NormalTok{no\_roots }\OtherTok{\textless{}{-}} \ControlFlowTok{function}\NormalTok{(a, b, c)\{}
  \CommentTok{\# Boolean of whether or not discriminant is negative}
\NormalTok{  b}\SpecialCharTok{**}\DecValTok{2} \SpecialCharTok{{-}} \DecValTok{4}\SpecialCharTok{*}\NormalTok{a}\SpecialCharTok{*}\NormalTok{c }\SpecialCharTok{\textless{}} \DecValTok{0}
  \CommentTok{\# return not needed}
\NormalTok{\}}
\end{Highlighting}
\end{Shaded}

\hypertarget{conditionals}{%
\subsubsection{Conditionals}\label{conditionals}}

\begin{Shaded}
\begin{Highlighting}[]
\NormalTok{exam\_grade }\OtherTok{\textless{}{-}} \DecValTok{94}

\CommentTok{\# C{-}like syntax}

\ControlFlowTok{if}\NormalTok{(exam\_grade }\SpecialCharTok{\textgreater{}=} \DecValTok{95}\NormalTok{)\{}
  \FunctionTok{cat}\NormalTok{(}\StringTok{"Yay"}\NormalTok{)}
\NormalTok{\}}\ControlFlowTok{else} \ControlFlowTok{if}\NormalTok{(exam\_grade }\SpecialCharTok{\textgreater{}=} \DecValTok{94}\NormalTok{)\{}
  \FunctionTok{cat}\NormalTok{(}\StringTok{"OK"}\NormalTok{)}
\NormalTok{\}}\ControlFlowTok{else}\NormalTok{\{}
  \FunctionTok{cat}\NormalTok{(}\StringTok{"Drop out"}\NormalTok{)}
\NormalTok{\}}
\end{Highlighting}
\end{Shaded}

\begin{verbatim}
## OK
\end{verbatim}

\end{document}
